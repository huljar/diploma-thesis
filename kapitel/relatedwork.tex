Many concepts that are explored and implemented in this thesis have been the subject of intensive research in recent years. This chapter aims to
summarize the important and relevant works and papers that have been published concerning these topics. Beginning with security in \gls{noc}
environments, many different attack scenarios and the proposed corresponding defenses are highlighted. Afterwards, advances in network coding with
respect to its application in \glspl{noc} are presented. Following this, the work preceding this thesis is introduced, which gives an overview of
state-of-the-art lightweight cryptography. Finally, the relevance of these works and their relation to this thesis is discussed.

\section{Network-On-Chip Security}\label{sec:nocsecurity}
Research on new and efficient ways to interconnect components on a single chip has been an important field of research for decades. The concept of
general-purpose on-chip networks has been introduced in the early 2000s
\cites{dally01routepacketsnotwires}{kumar02networkonchip}{benini02nocparadigm} and has quickly gained a lot of traction in the research community
\cite[e.g.][]{ivanov05nocintroduction}. 

However, with rising popularity, the interest of adversaries to compromise such systems also grows. In recent research, many different attack vectors
on \gls{noc} architectures have been explored, and a variety of countermeasures have been proposed to mitigate attacks. The following sections
explore them in detail.

\subsection{Security Zones}\label{subsec:securityzones}
% Security zones
The necessity of security measures as part of the design was already recognized early on in \gls{noc} research. A popular approach is to
divide the \gls{noc} into security zones with different levels of protection, where sensitive information is handled exclusively inside a
high-security zone \cites(e.g.)(){gebotys03securityframework}{fernandes16nocrouting}{kapoor13nocauthenc}.

Published in 2003, \citeauthor{gebotys03securityframework} \cite{gebotys03securityframework} were among the first to propose a \enquote{general security architecture}
\cite[1]{gebotys03securityframework} to impede attacks \enquote{at both the network level […] and at the core level}
\cite[1]{gebotys03securityframework}. At the network level, they differentiate between secure cores and other cores, thus establishing two security
zones. The secure cores are capable of encrypting and authenticating network
traffic, and are thus designed to handle sensitive user information. In addition, there is a dedicated key-keeper core that handles key distribution
amongst the secure cores. At the core level (also referred to as application level), the authors propose to use a modified implementation of elliptic curve
cryptography to facilitate encryption and authentication. Aiming to provide resistance against side channel power attacks, their modifications conceal
the power traces of the different algebraic computations during the cryptographic operations. This hinders adversaries from extracting key bits based
on power spikes.

\citeauthor{kapoor13nocauthenc} \cite{kapoor13nocauthenc} have pursued a similar approach. They also separate the cores into secure and non-secure
ones and propose to implement authenticated encryption in the network interfaces. While the secure
cores employ permanent keys to communicate with each other, the non-secure ones may use plaintext messages. Additionally, in order to allow
communication between the security zones, sessions may be established with individually generated session keys. Furthermore, the authors employ
traffic limitations in the network interfaces to prevent \gls{dos} attacks by malicious cores, while access rights tables prohibit unauthorized
memory accesses.

Another work that revolves around security zones in \glspl{noc} was published by \citeauthor{fernandes16nocrouting} \cite{fernandes16nocrouting}.
In contrast to the papers presented above, they deem the underlying hardware to be secure. Assuming that \enquote{software attacks account
for 80\% of security incidents in embedded systems} \cite[1]{fernandes16nocrouting}, their work focuses on the defense against software-based
\gls{dos} and timing attacks.

The goal of the authors is to ensure that packets from one security zone are not routed through nodes of a different zone, if possible. To achieve
this, they have adapted the routing algorithm of the network to prioritize paths that do not cross zone boundaries. In addition, they have
refined the algorithm to guarantee deadlock freedom despite the constraints imposed by the new path restrictions.

\subsection{Hardware Trojans}
% HT Survey & building trustworthy hardware
\citeauthor{bhunia14hardwaretrojans} \cite{bhunia14hardwaretrojans} have looked thoroughly into the threat of hardware trojans and possible protection
approaches. In a survey-like paper, they provide a detailed summary of attack scenarios, countermeasures, and detection paradigms. Similarly,
\citeauthor{sethumadhavan15trustworthyhardware} \cite{sethumadhavan15trustworthyhardware} analyze the challenge of building systems from untrusted
hardware components. They explain in detail how the hardware design and fabrication chain can be adapted to significantly lower the probability of
integrating malicious components. The methods in both works are not specific to \gls{noc} architectures, but are applicable to them nonetheless.

% Fort-NoCs
\citeauthor{ancajas14fortnocs} \cite{ancajas14fortnocs} have investigated the threat of hardware trojans specifically for \glspl{noc}. They show that the usage of
third party \gls{noc} \gls{ip} is very popular due to cost efficiency, opening up a practical infection vector for hardware
trojans. Together with a software accomplice (i.e., an infected processing element) that can send commands to the trojan, this may
lead to information leaks, data corruption, or denial of service attacks.

The authors focus on mitigating \enquote{covert data theft by a compromised \gls{noc}} \cite[3]{ancajas14fortnocs}. They suggest a three-layer
approach to mitigate this threat, consisting of data scrambling, packet certification, and node obfuscation. These techniques are
implemented solely in the network interfaces, which are not provided by a third party and hence are assumed to be trustworthy. The goal of these measures
is to prevent the activation of the hardware trojan and render transmitted information unreadable to the attacker.

% Hardened NoC design
\citeauthor{frey17hardenednoc} \cite{frey17hardenednoc} also worked on mitigating the effect of hardware trojans in a \gls{noc}. Their goal is to
harden the \gls{noc} design against potential hardware trojans located inside the routers. In contrast to \citeauthor{ancajas14fortnocs}
\cite{ancajas14fortnocs}, the protective measures are implemented at the router level and not in the network interfaces. They are designed to
\enquote{complement […] the previous \gls{noc} works aiming for \gls{ni} security} \cite[16]{frey17hardenednoc} and address \gls{dos} attacks rather
than information leakage.

The idea of the authors is to detect any flit tampering right after it leaves a router. To achieve this, an error control code and dynamic flit
permutation are applied to all flits before they enter a router, and the reverse transformations are applied after they exit the router again. This
prevents (or at least detects) small and targeted modifications to the flit headers.

% AMD codes
\citeauthor{boraten16packetsecurity} pursue a very similar approach. In their 2016 paper \cite{boraten16packetsecurity}, they propose to apply
\gls{amd} and \gls{crc} codes in the network interfaces to protect packets from fault injections by a malicious \gls{noc}. The authors suggest to use
\gls{amd} codes for sensitive data and \gls{crc} for \enquote{all other non-critical traffic} \cite[2]{boraten16packetsecurity}. First introduced in 2008
\cite{cramer08amdcodes}, \gls{amd} codes are capable of \enquote{[detecting] any tampering by an adversary} \cite[1]{cramer08amdcodes}.

Following up on their previous work, the authors published a 2018 paper \cite{boraten18mitigationdos} that refines their methods to facilitate hardware
trojan detection and mitigation. In contrast to many of the previously presented works \cites(e.g.)(){ancajas14fortnocs}{frey17hardenednoc}, they aim
to first detect the presence of a potential hardware trojan before activating further security measures. The described attacker model employs fault injections to
intentionally trigger responses from the error correction codes (like the \gls{amd} code explored earlier \cite{boraten16packetsecurity}), thus
performing a \gls{dos} attack.

Trojans are discovered with a \enquote{heuristic-based fault detection model} \cite[25]{boraten18mitigationdos} that classifies faults into accidental
and intentional ones. Once this scheme detects the presence of a trojan, additional security measures are employed. First, the authors try to keep
using the malicious links in order to not degrade network performance. Before a packet is routed through an infected area, it is obfuscated to prevent
the hardware trojan from triggering. If this fails, and faults are still being injected, \enquote{links must be disabled and routers should route
around them} \cite[32]{boraten18mitigationdos}.

\subsection{Routing Strategies}
\citeauthor{stefan11enhancingnocs} \cite{stefan11enhancingnocs} explore the potential of multipath routing as a security enhancement in \glspl{noc}.
Extending the existing \textit{\AE thereal} framework \cite{goossens05aethereal}, they implemented time-division multiplexing for choosing a route at
the sender's network interface. Two variants were pursued: static, deterministic path selection and dynamic selection at run-time (e.g., by using a
hardware random number generator). Through several experiments, they conclude that the static variant has significantly less chip area overhead than
the dynamic one and thus should be the preferred method.

Also focusing on time-division multiplexed \glspl{noc}, \citeauthor{wassel13surfnoc} \cite{wassel13surfnoc} promote the usage of different domains for
the network traffic with strict non-interference requirements. Categorizing packets into such domains \enquote{help[s] prevent cascading failures}
\cite[1]{wassel13surfnoc} and hampers \gls{dos} attacks from affecting the whole system. In their work, the authors propose an efficient method of
this domain-based routing with very low latency, making this security measure practical.

\section{Network Coding For Network-On-Chip}\label{sec:ncfornoc}
The integration of network coding into \gls{noc} architectures is a relatively new field of research. It was briefly mentioned a decade ago by
\citeauthor{fragouli08ncapplications} \cite{fragouli08ncapplications}, where emerging network coding applications were discussed. However, it was
mostly seen as a method to help \enquote{simplify and minimize the length of on-chip wiring} \cite[260]{fragouli08ncapplications}, rather than as a
robustness measure. In recent years, several works have been published that investigate the applicability and practicality of network coding for
\glspl{noc}.

\citeauthor{indrusiak11ncfornocs} \cite{indrusiak11ncfornocs} has looked into the benefits of network coding in multicast settings. A traditional
example of a network topology where this can entail large performance improvements is the so-called \textit{butterfly network} \cites(cf.)(){ahlswede00networkflow}{li03linearnc}.
To transfer the advantages of this particular setup to \glspl{noc}, he focuses on finding such butterfly arrangements in 2D mesh topologies and
mapping them onto the underlying architecture by finding appropriate intermediate nodes.

Following up on this idea, \citeauthor{shalaby12nodeselection} \cite{shalaby12nodeselection} have also looked into finding butterfly arrangements in
\glspl{noc}. Their emphasis lies on enhancing the selections of the intermediate nodes and the way these choices impact network performance. While
evaluation yielded only modest improvements, their results \enquote{confirm the great potential of network coding to improve \gls{noc}
performance} \cite[5]{shalaby12nodeselection}.

\citeauthor{duongba11ncinmulticore} \cite{duongba11ncinmulticore} have examined network throughput and latency for multi-core processors when applying
network coding. In contrast to many-core systems, the processors were assumed to have relatively small \glspl{noc}, namely with six or nine nodes.
They show that network coding improves network throughput in both unicast and multicast scenarios while slightly increasing the latencies. For
multicast communication in saturated networks, however, the average latency was drastically improved.

\citeauthor{xue15ncnoc} \cite{xue15ncnoc} have integrated network coding into a \gls{noc} architecture for multicast communications. They propose to
send both encoded and regular flits of the same packets into the network, creating a redundant transmission. This introduces benefits during the
routing of the flits, like allowing to drop some of them in order to potentially prevent network congestion.

The goal of \citeauthor{vonbun13nchopcount} \cite{vonbun13nchopcount} is to improve the evaluation of the impact that network coding has on
\glspl{noc}. According to the authors, previous works mostly used specific evaluation metrics tailored to the investigated approach, such as only
considering butterfly connections. In contrast, they aim to \enquote{demonstrate the potential of network coding in generalized connection settings}
\cite[2]{vonbun13nchopcount} and have developed an extensive framework to compute appropriate metrics based on hop counts. By applying their
evaluation method to a number of different network configurations, they were able to show that network coding almost always performs better than uncoded,
deterministic routing. In the worst cases, performance stayed the same.

In a recent work at the TU Dresden, \citeauthor{moriam15manycorenc} \cite{moriam15manycorenc} have investigated the applicability of network coding in
a unicast setting. They argue that many transmission errors may occur \enquote{due to dynamic variations of voltage and temperature}
\cite[1]{moriam15manycorenc}, leading to a large number of retransmissions and thus a severe degradation of \gls{noc} performance. In their work, they
have analyzed how different variants of network coding may remedy this situation. Using a \enquote{cycle-accurate \gls{noc} traffic simulator}
\cite[3]{moriam15manycorenc}, several performance metrics such as average latency and information rate\footnote{The information rate describes the
amount of transmitted information in relation to the total number of flits.} were examined. The results show significant improvements for error-prone
transmissions, but decreased performance in reliable networks.

Following up on this research, \citeauthor{moriam18activeattackers} \cite{moriam18activeattackers} have constructed an authentication layer on top of
their network coding system. Seeking to ensure the integrity of messages, they employ \enquote{lightweight message authentication codes (\glspl{mac})
to guarantee detection of modification} \cite[1]{moriam18activeattackers}, such as faults injected by compromised routers. In case an integrity breach is
detected by a receiver, an \gls{arq} is issued back to the sender to request the retransmission of the affected flits. The infected routers are assumed to
be randomly distributed among the network and to constitute only a small portion of the routers. Their evaluation shows that network coding is able to
significantly reduce the number of required retransmissions and the residual error probability at the cost of a lower information rate.

Since this thesis seeks to build upon their work, some concepts of \citeauthor{moriam18activeattackers} can be found here as well. Their network coding
scheme was reimplemented, which is elaborated in Section \ref{sec:designnc}. Additionally, the authentication methods that they experimented with
are taken up (see Section \ref{sec:theprotocol}). The performance metrics used to evaluate the effectiveness of their approach are also adopted, as
described in Section \ref{subsec:criteria}.

\section{Lightweight Cryptographic Algorithms}\label{sec:lightweightcrypto}
In order to fulfill the protection goals discussed in Section \ref{sec:protectiongoals}, the addition of encryption and authentication to
the communication passing through a \gls{noc} suggests itself. However, care needs to be taken to stay within the latency, area, and power constraints
of this environment (cf. Section \ref{sec:networkonchipfun}). As standard cryptographic algorithms, such as \gls{aes}, are usually not efficient
enough for this task \cite[1]{bogdanov07present}, novel designs have become essential for the implementation of cryptographic protection measures.

In an analysis conducted prior to this thesis, lightweight cryptographic algorithms were thoroughly explored \cite{harttung17lightweightcrypto}. Such
algorithms are
specifically designed to enable efficient hardware implementations with low area and power requirements. In addition, they aim to have a small
computation delay while still providing an adequate level of security. Examples of such algorithms are PRESENT \cite{bogdanov07present},
mCrypton \cite{lim06mcrypton}, PRINCE \cite{borghoff12prince}, and Klein \cite{gong12klein}.

The algorithms were evaluated and compared in terms of latency and chip area requirements. A dedicated \gls{fpga} implementation for each algorithm
was used to allow for a fair comparison. PRESENT was found to occupy the least chip area, while PRINCE managed to be the fastest.
Additionally, the large discrepancy of \gls{aes} to the lightweight algorithms was confirmed, cementing its ineptitude for the task at hand.

\section{Assessment And Significance}\label{sec:assandsig}
Overall, the studies presented in the sections above corroborate the importance of both efficient and secure \glspl{noc} for \gls{mpsoc}
communications. Furthermore, the feasibility of network coding and cryptography in \gls{noc} architectures was confirmed.

In Section \ref{sec:nocsecurity}, a variety of attack vectors and proposed countermeasures was explored. While the approaches based on security zones
look very promising, they suggest a classification of communications into sensitive and nonsensitive ones and emphasize the isolation of data between these two
categories. However, such a separation seems difficult in practice, especially when third party \gls{ip} is used throughout the whole network. Thus, a
holistic approach appears to be more desirable. Still, the cryptographic protections, such as the authenticated encryption proposed by
\citeauthor{kapoor13nocauthenc} \cite{kapoor13nocauthenc}, are closely related to the ideas pursued in this thesis\footnote{To implement authenticated
encryption, they employ \gls{aes} in the \gls{gcm} mode of operation. This scheme, however, is not lightweight enough to consider it for this thesis.}.

The studies investigating mitigations of the threat of hardware trojans follow holistic approaches. They protect communications at the network
interfaces, potentially complemented by checks at the router boundaries. For this thesis, the approach of \citeauthor{ancajas14fortnocs}
\cite{ancajas14fortnocs} is particularly relevant. Their three-layer system includes data scrambling and packet certification in the network
interfaces, which closely corresponds to the encryption and authentication techniques explored in this work. Furthermore, they have confirmed the
practicality of hardware trojan attacks, which are the basis for the threat model that is analyzed in this thesis.

The investigation of different routing strategies is also highly relevant. Since a key objective of this thesis is the usage of multiple paths, several
routing strategies will consequently be explored. The comparison of deterministic and dynamic path selection that \citeauthor{stefan11enhancingnocs}
\cite{stefan11enhancingnocs} have conducted is of particular interest, as similar variants are also pursued here. Although
\citeauthor{stefan11enhancingnocs} conclude that the deterministic approach is better suited for most applications, dynamic path selections may prove
advantageous, especially in combination with network coding.

In Section \ref{sec:ncfornoc}, it was shown that network coding is applicable to \glspl{noc} with notable performance improvements. While many
researchers have explored the effect of network coding for \glspl{noc}, a lot of them pursued the mapping of butterfly networks onto 2D meshes in
order to improve multicast performance. As the experiments in this thesis solely deal with the unicast scenario, those ideas cannot be picked up
directly. However, \citeauthor{duongba11ncinmulticore} \cite{duongba11ncinmulticore} and \citeauthor{moriam15manycorenc} \cite{moriam15manycorenc}
have achieved significant performance gains in the unicast setting. The latter work is particularly relevant, as the usage of network coding in this
thesis is inspired by their concept. Furthermore, the combination of network coding and authentication techniques investigated by
\citeauthor{moriam18activeattackers} \cite{moriam18activeattackers} proves that such an approach is feasible and able to effectively provide integrity
in a partially compromised \gls{noc}. However, their implementation requires 18 parallel cryptographic modules per network interface, resulting in a
noticeable area overhead.
\pagebreak

Finally, the work preceding this thesis was presented in Section \ref{sec:lightweightcrypto}. It confirms the applicability of lightweight
cryptographic algorithms in constrained hardware environments (such as \glspl{noc}). Especially the PRINCE algorithm is of high interest due to its
low latency, which is a direct requirement for efficient communication\footnote{Since low area is also a requirement, PRESENT is of a similar
interest. However, latency was deemed more important than area. The reasoning is explained in Section \ref{subsubsec:prince}.} (cf. Section
\ref{sec:networkonchipfun}).
