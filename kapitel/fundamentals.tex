\section{Networks-On-Chip}\label{sec:networkonchipfun}
\textit{Networks-on-Chip} (or \textit{\glspl{noc}} for short) are a method of interconnecting components on a chip. Typically employed on
\textit{Multi-Processor Systems-on-Chip (\glspl{mpsoc})} \cites(e.g.)(){ivanov05nocintroduction}{biswas15routerattack}{tatas16designingnocs}, they
provide the communication infrastructure for \textit{processing elements (\glspl{pe})} and possibly other \gls{ip} cores.

The topology of a \gls{noc} can vary. Researchers usually work with a 2D mesh topology
\cites(e.g.)(){frey17hardenednoc}{kumar02networkonchip}{fernandes16nocrouting}{boraten16packetsecurity}, which will also be used throughout this thesis.
In this case, each network node is connected to its four neighbors (excluding the boundary nodes).

A node typically consists of a processing element, a network interface (\textit{\gls{ni}}), and a router. \cite{dally01routepacketsnotwires} The
router manages the connections to neighboring nodes while also allowing the local processing element to communicate with the network through the
network interface. An example of this architecture is given in Figure \vref{fig:nocexample}.

Compared to traditional bus-based interconnect systems, \glspl{noc} can provide a lot of advantages, especially for many-core systems.
\cite[5\psqq]{tatas16designingnocs} One big advantage is scalability; because the cores do not share a global bus, \enquote{local performance is not
degraded} \cite[6]{tatas16designingnocs} as more components are added, and \enquote{aggregated bandwidth scales with the network size}
\cite[6]{tatas16designingnocs}.

In addition, the absence of global connections facilitates the use of different clock domains. This enables the implementation of the
\textit{globally asynchronous, locally synchronous (\gls{gals})} paradigm, which becomes increasingly important in chip design.
\cites[3]{kumar02networkonchip}[2]{ivanov05nocintroduction}

Furthermore, with the constantly increasing design complexity of modern chips \cite{mack11mooreslaw}, specialized on-chip
interconnections become infeasible to implement. Designing such a system \enquote{would take too much time and mapping of applications to dedicated
architectures would be impossible} \cite[1]{kumar02networkonchip}. In contrast, \glspl{noc} aims to be general purpose interconnect systems; they
\enquote{facilitate […] modularity by defining a standard interface} \cite[1]{dally01routepacketsnotwires}.

\begin{figure}
    \centering
    \includegraphics[width=0.5\textwidth]{noc_3x3_colored}
    \caption[Example of a 3x3 NoC]{Example of a 3x3 Network-on-Chip. The processing elements (red) contain a network interface
    (green), through which they are connected to a router (blue). The routers are connected in a 2D mesh topology.}
    \label{fig:nocexample}
\end{figure}

\section{Flits}\label{sec:flits}
Flits (short for \textit{flow control units}) % TODO: explain that there are flit headers/body

\section{Hardware Trojans}\label{sec:hardwaretrojans}
% What are HTs, why can they get into other hardware, what are their properties
Hardware trojans are \enquote{malicious modifications of electronic hardware} \cite[1]{bhunia14hardwaretrojans} with the intent of disrupting normal
system behavior or extract sensitive information. Because the integration of third party \gls{ip} has become increasingly popular due to circuit
complexity and cost efficiency \cites[1]{ancajas14fortnocs}[2]{bhunia14hardwaretrojans}, it is possible for adversaries to introduce hardware
trojans into larger systems, such as \glspl{mpsoc}.

In order to remain undetected, attackers aim to construct hardware trojans that are \enquote{stealthy in nature} \cite[1]{bhunia14hardwaretrojans}
and \enquote{evade […] detection through conventional postmanufacturing test} \cite[1]{bhunia14hardwaretrojans}. Hardware trojans are usually in a
dormant state until they are activated by a trigger signal to carry out their task. \cites{bhunia14hardwaretrojans}{ancajas14fortnocs} While the
trojan is inactive, communications through the \gls{noc} are unaffected and the system operates normally.

Attack types: information leak/eavesdropping, DoS (→ bandwidth depletion, deadlock, livelock)
% TODO: is this a fundamental? or write this when describing our attacker model?

\section{Network Coding}\label{sec:networkcodingfun}
