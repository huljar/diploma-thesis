In this chapter, core terms and concepts that are essential to the understanding of this thesis are explained. First, networks-on-chip are
described, as this whole work builds upon and analyzes them. Afterwards, fundamental concepts of information security and their relevance for this
thesis are outlined. Finally, essential aspects of the communication strategy devised in this thesis are delineated.

\section{Networks-On-Chip}\label{sec:networkonchipfun}
\textit{Networks-on-Chip (\glspl{noc})} are a paradigm for interconnecting components on a chip. They are employed on
\textit{Multi-Processor Systems-on-Chip (\glspl{mpsoc})} \cites(e.g.)(){ivanov05nocintroduction}{biswas15routerattack}{tatas16designingnocs}, where they
provide the communication infrastructure for the \textit{processing elements} and other \textit{intellectual property (\gls{ip}) cores}\footnote{The
term \textit{\gls{ip} cores} encompasses many kinds of chip components. Besides processing elements, it includes memory nodes and hardware
accelerators, for example.}. The basic concept is to incorporate many point-to-point connections between neighboring cores, breaking with the traditional
concept of global interconnection buses.

\glspl{noc} are represented as a topology of network nodes. In recent research, the topology is usually assumed to be a 2D mesh
\cites(e.g.)(){frey17hardenednoc}{kumar02networkonchip}{fernandes16nocrouting}{boraten16packetsecurity}. While this is not a
requirement\footnote{Examples of other \gls{noc} topologies are 2D folded tori and 3D meshes \cite[2]{feero07noc3d}.}, it is both convenient
to work with and applicable in practice. Thus, for this thesis, the network nodes are also laid out in this manner.
In a 2D mesh, each node is connected to its four neighbors (excluding the ones located at the boundaries).

A node consists of a processing element\footnote{As already mentioned, other types of cores are also possible. Since only processing elements are of
relevance to this thesis, no further distinctions will be made.}, a network interface, and a router \cite{dally01routepacketsnotwires}. The processing
element executes applications that generally require it to be able to send and receive messages. The network interface facilitates this by connecting the
processing element to the router. Finally, the router manages the interconnections with neighboring nodes and the local processing element. It is also
responsible for forwarding incoming messages towards their destinations. An example of this architecture is given in Figure \vref{fig:nocexample}.

\begin{figure}
    \centering
    \includegraphics[width=0.6\textwidth]{noc-3x3-colored}
    \caption[Example of a 3x3 mesh NoC]{Example of a Network-on-Chip in a 2D mesh topology of size 3x3. The processing elements (\gls{pe}) contain a network interface
    (\gls{ni}) through which they are connected to a router (R). The routers are interconnected as a 2D mesh.}
    \label{fig:nocexample}
\end{figure}

Compared to traditional bus-based interconnect systems, \glspl{noc} provide a lot of advantages\footnote{There are also disadvantages entailed by
the use of \glspl{noc} over bus-based systems, such as incompatibility with bus-oriented \glspl{ip}, but research has shown them to be outweighed
by the benefits. \citeauthor{tatas16designingnocs} give a detailed comparison \cite[6]{tatas16designingnocs}.}, especially for many-core systems
\cite[5\psqq]{tatas16designingnocs}. A significant benefit is scalability: since the cores do not share a global bus, \enquote{local performance is not
degraded} \cite[6]{tatas16designingnocs} as more components are added, and \enquote{aggregated bandwidth scales with the network size}
\cite[6]{tatas16designingnocs}.

In addition, the absence of global interconnection wires facilitates the use of different clock domains. This enables the implementation of the
\textit{globally asynchronous, locally synchronous (\gls{gals})} paradigm, which becomes increasingly important in chip design due to the growing
number of cores and resulting timing closure issues \cites[3]{kumar02networkonchip}[2]{ivanov05nocintroduction}.

Furthermore, the constantly increasing design complexity of modern chips \cite{mack11mooreslaw} renders specialized on-chip
interconnections infeasible to implement. Designing such a system \enquote{would take too much time and mapping of applications to dedicated
architectures would be impossible} \cite[1]{kumar02networkonchip}. In contrast, \glspl{noc} are intended to be general-purpose interconnect systems; they
\enquote{facilitate […] modularity by defining a standard interface} \cite[1]{dally01routepacketsnotwires}.

When utilizing \gls{noc} architectures in practice, it is crucial that they allow for an efficient implementation and swift communications. The
former is accomplished by optimizing the components for low area requirements and power consumption, while the latter is realized
through small electronic delays and message transmission latencies. In this thesis, several approaches like specifically lightweight cryptographic
algorithms and single-cycle routing are pursued to meet these criteria. They are explained in detail in Chapters \ref{ch:protocol} and
\ref{ch:implementation}.

While the term \textit{\gls{noc}} originally refers to the interconnection paradigm, it is also often used in reference to actual on-chip networking
hardware that follows this paradigm. Throughout this thesis, it will mostly be used to denote the networking hardware.

\section{Protection Goals}\label{sec:protectiongoals}
% Confidentiality, integrity, availability
Protection goals describe desired properties of a system with regard to information security. They consist of \textit{confidentiality},
\textit{integrity}, and \textit{availability}\footnote{The goals may be further classified to include \textit{unobservability} and \textit{plausible
deniability}, for example. However, these are extraneous to this thesis.}. Confidentiality means preventing unauthorized access to information, which
can be accomplished, for example, by means of encryption. Integrity is attained by ensuring that any unintentional modifications to information, both accidental and
malicious, are detected, for instance through authentication techniques. Finally, availability refers to the reliable accessibility of information. It is
achieved via methods like backup systems and redundant hardware.

Naturally, in \gls{noc} architectures, it is desirable to fulfill these goals\footnote{The necessity for security measures was demonstrated in
related research, which is elaborated in Section \ref{sec:nocsecurity} and reiterated in Section \ref{sec:necessityofsecurity}.}. In this thesis,
confidentiality and integrity through encryption and authentication, respectively, are examined.

\section{Hardware Trojans}\label{sec:hardwaretrojans}
% What are HTs, why can they get into other hardware, what are their properties
Hardware trojans are \enquote{malicious modifications of electronic hardware} \cite[1]{bhunia14hardwaretrojans} intended to disrupt normal
system behavior or to extract sensitive information. The introduction of hardware trojans into larger systems (such as \glspl{mpsoc}) is possible
through a number of different infection vectors (see Section \ref{sec:necessityofsecurity} for details).

In order to remain undetected, adversaries aim to construct hardware trojans that are \enquote{stealthy in nature} \cite[1]{bhunia14hardwaretrojans}
and \enquote{evade […] detection through conventional postmanufacturing test} \cite[1]{bhunia14hardwaretrojans}. Hardware trojans are usually in a
dormant state until they are activated by a trigger signal to carry out their task \cites{bhunia14hardwaretrojans}{ancajas14fortnocs}. While the
trojan is inactive, communications through the \gls{noc} are unaffected and the system operates normally.

\section{Flow Control Units}\label{sec:flitsfun}
\textit{Flow control units (flits)} are small pieces of data that are sent over a network. They are usually created by breaking a larger
packet down into multiple parts to allow for their individual transmission \cite[6]{flitslecturecmu}. Each flit contains a set of header fields (such as source and
destination address, sequence number, or identifier) that are required for routing and handling by the receiver \cite[2]{flitslectureutah}.
In addition, it contains a payload that carries the actual information to be transferred.

In the context of \glspl{noc}, flits are often used as the standard unit of transmission \cite[51\psqq]{tatas16designingnocs}, which is the case for
this thesis as well. Details on how flits are employed in the devised communication protocols can be found in Chapter \ref{ch:protocol}.

\section{Automatic Repeat Requests}\label{sec:arqs}
When reliable data transmissions over an unreliable network are required, the communication protocol needs to ensure that all packets arrive unmodified
at their intended destinations. A common way to achieve this is the usage of \textit{automatic repeat requests (\glspl{arq})}.

In traditional implementations, such as \gls{tcp}, the receiver confirms the arrival of packets by answering each one with an acknowledgment.
If the sender does not receive such a confirmation within a given time span, the corresponding packet is assumed lost and retransmitted.

A \gls{noc} is considered an unreliable network in this thesis. The reason for this are compromised, malicious routers that have the potential to drop
or modify flits, intending to disrupt the normal network operations.

In the protocol devised for this thesis, there are no acknowledgments for successfully transmitted flits. Instead, the receiver informs the sender of
missing or corrupted flits by issuing an \gls{arq} flit back to the sender. Upon arrival, the sender will retransmit the flits in question. In
combination with authentication techniques, this aids in fulfilling the protection goal of integrity. This scheme is outlined in Section
\ref{sec:reliability} and further elaborated in Section \ref{subsec:arqretransmissions}.

\section{Network Coding}\label{sec:networkcodingfun}
Network coding is a technique for transmitting packets efficiently over a network. First described in 2000 by \citeauthor{ahlswede00networkflow}
\cite{ahlswede00networkflow}, the idea is to maximize the information flow through a network and achieve higher throughput than traditional transmission
schemes. It is achieved by allowing intermediate network nodes to encode incoming packets before passing them on, creating combinations of different
packets in the process. At the destinations, the received data can be decoded again to obtain the original packets. Traditionally, network coding is
used in multicast communication patterns, but it has also been applied successfully to unicast scenarios \cite[e.g.][]{moriam15manycorenc}.

A popular coding scheme, dubbed \textit{linear network coding} \cite{li03linearnc}, is to regard all packets arriving at a node from different incoming links as a vector.
Then, linear transformations can be applied to it to obtain new combinations to send out \cite[1]{li03linearnc}. To allow receivers to decode the
combinations into the original packets, they need to be aware of the encoding patterns applied at each node. Those can be \enquote{agreed upon
beforehand} \cite[1]{li03linearnc}, but this requires global knowledge of the network topology. As this is generally not the case, a practical
alternative is to attach the encoding information to the packets in the form of a \textit{global encoding vector (\gls{gev})}
\cites[2\psqq]{chou03practicalnc}[5\psq]{chou07ncforinternetandwireless} that is updated at each intermediate node to represent the current encoding
pattern. A receiver can then decode the combinations solely using the supplied \glspl{gev}.

The objective of applying network coding is to maximize the information flow and throughput in a network. Depending on the communication patterns and
network reliability, it may result in better overall performance and lower transmission delays, especially in saturated networks
\cite[7]{duongba11ncinmulticore}. Furthermore, it provides \enquote{a natural way to take advantage of multipath diversity for security against
wiretapping attacks} \cite[8]{fragouli07ncfundamentals}. In the context of \glspl{noc}, it has the potential to hinder eavesdropping attempts inside
the network, but only for limited passive attackers that merely have access to distinct parts of the network. Therefore, in this thesis, it is not
viewed as a security measure, but as a complementary technique to increase the network's performance. This approach is elucidated in Sections
\ref{sec:networkcodingover} and \ref{sec:designnc}.
