% TODO: Leser leiten, was kommt jetzt, warum gehe ich auf bestimmte Dinge ein
% TODO: ist components eindeutig/korrekt?
% TODO: Abkürzungen kursiv oder nicht?
% TODO: einmal IP ausschreiben?

\section{Networks-On-Chip}\label{sec:networkonchipfun}
\textit{Networks-on-Chip \glspl{noc})} are a paradigm for interconnecting components on a chip. Typically employed on
\textit{Multi-Processor Systems-on-Chip (\glspl{mpsoc})} \cites(e.g.)(){ivanov05nocintroduction}{biswas15routerattack}{tatas16designingnocs}, they
provide the communication infrastructure for \textit{processing elements (\glspl{pe})} and possibly other \gls{ip} cores\footnote{Besides PEs, IP
cores encompass other types of components, such as memory nodes}.
% TODO: Leser leiten, Begriffe erwähnen, die nachfolgend erklärt werden

% TODO: 2. Satz als erstes, Fußnote um zu sagen, dass HAEC 3D mesh benutzt als Gegenbeispiel?
% TODO: "which is also used throughout ..." inhaltlich korrekt?
The topology of a \gls{noc} can vary. Researchers usually work with a 2D mesh topology
\cites(e.g.)(){frey17hardenednoc}{kumar02networkonchip}{fernandes16nocrouting}{boraten16packetsecurity}, which is also used throughout this thesis.
In this case, each network node is connected to its four neighbors (excluding the boundary nodes).

% TODO: short sentence about PE and NI
A node consists of a processing element (often referred to as an \textit{\gls{ip} core} or just \textit{core}), a network interface,
and a router \cite{dally01routepacketsnotwires}. The router manages the connections to neighboring nodes and allows
the local processing element to communicate with the network through the network interface. An example of this architecture is given in Figure
\vref{fig:nocexample}.

\begin{figure}
    \centering
    \includegraphics[width=0.6\textwidth]{noc_3x3_colored}
    \caption[Example of a 3x3 mesh NoC]{Example of a Network-on-Chip in a 2D mesh topology of size 3x3. The processing elements (red) contain a network interface
    (green) through which they are connected to a router (blue). The routers are interconnected as a 2D mesh.}
    \label{fig:nocexample}
\end{figure}

% TODO: Fußnote
Compared to traditional bus-based interconnect systems, \glspl{noc} provide a lot of advantages\footnote{There are also disadvantages entailed by
the use of \glspl{noc} over bus-based systems, but the benefits outweigh them. A detailed comparison is given in 30.}, especially for many-core systems
\cite[5\psqq]{tatas16designingnocs}. A significant advantage is scalability: since the cores do not share a global bus, \enquote{local performance is not
degraded} \cite[6]{tatas16designingnocs} as more components are added, and \enquote{aggregated bandwidth scales with the network size}
\cite[6]{tatas16designingnocs}.

% TODO: fehlt hier was in der Erzählkette? Begründung für "increasingly important"?
In addition, the absence of global interconnection wires facilitates the use of different clock domains. This enables the implementation of the
\textit{globally asynchronous, locally synchronous (\gls{gals})} paradigm, which becomes increasingly important in chip design
\cites[3]{kumar02networkonchip}[2]{ivanov05nocintroduction}.

Furthermore, the constantly increasing design complexity of modern chips \cite{mack11mooreslaw} renders specialized on-chip
interconnections infeasible to implement. Designing such a system \enquote{would take too much time and mapping of applications to dedicated
architectures would be impossible} \cite[1]{kumar02networkonchip}. In contrast, \glspl{noc} are intended to be general purpose interconnect systems; they
\enquote{facilitate […] modularity by defining a standard interface} \cite[1]{dally01routepacketsnotwires}.

% TODO: talk about area and power constraints

\section{Flits}\label{sec:flits}
\textit{Flow control units (flits)} are small pieces of data that are sent over a network. They are usually created by breaking a larger
packet down into parts to allow for their individual transmission \cite[6]{flitslecturecmu}. Each flit must contain a set of header fields (such as source and
destination address, sequence number, or identifier) that are required for routing and handling by the receiver \cite[2]{flitslectureutah}.
In addition, it contains a payload that carries the actual information to be transmitted.

In the context of \glspl{noc}, flits are often used as the standard unit of transmission \cite[51\psqq]{tatas16designingnocs}. Details on how flits
are used in this thesis can be found in (insert section/chapter vref).
% TODO: insert reference
% TODO: "used in this thesis" correct?

\section{Automatic Repeat Requests}\label{sec:arqs}
When reliable data transmissions over an unreliable network are desired, the communication protocol needs to ensure that all packets arrive at their
intended destinations. A common way to achieve this is the usage of \textit{automatic repeat requests}, or \textit{\glspl{arq}} for short.

In traditional implementations, the receiver confirms the arrival of packets and informs the sender by answering each packet with an acknowledgment.
If the sender does not receive a confirmation within a given time span, the corresponding packet is assumed lost and automatically retransmitted.

A \gls{noc} is considered an unreliable network in this thesis. One reason for this is potential network congestion, which can lead to high latencies
and dropped flits. Another possibility is compromised, malicious routers that drop flits to disrupt the normal network operations, e.g. as part of a
\gls{dos} attack.

In this thesis, there are no acknowledgments for successfully transmitted flits. Instead, the receiver informs the sender of missing or corrupted
flits by issuing an \gls{arq} back to the sender. Upon arrival, the sender will retransmit the flits in question. This scheme is described in detail
in Section (insert vref here).

\section{Hardware Trojans}\label{sec:hardwaretrojans}
% What are HTs, why can they get into other hardware, what are their properties
% TODO: third party IP nicht der einzige Grund
Hardware trojans are \enquote{malicious modifications of electronic hardware} \cite[1]{bhunia14hardwaretrojans} intended to disrupt normal
system behavior or to extract sensitive information. As the integration of third party \gls{ip} has become increasingly popular due to circuit
complexity and cost efficiency \cites[1]{ancajas14fortnocs}[2]{bhunia14hardwaretrojans}, it is possible for adversaries to introduce hardware
trojans into larger systems such as \glspl{mpsoc}.

In order to remain undetected, attackers aim to construct hardware trojans that are \enquote{stealthy in nature} \cite[1]{bhunia14hardwaretrojans}
and \enquote{evade […] detection through conventional postmanufacturing test} \cite[1]{bhunia14hardwaretrojans}. Hardware trojans are usually in a
dormant state until they are activated by a trigger signal to carry out their task \cites{bhunia14hardwaretrojans}{ancajas14fortnocs}. While the
trojan is inactive, communications through the \gls{noc} are unaffected and the system operates normally.

Attack types: information leak/eavesdropping, DoS (→ bandwidth depletion, deadlock, livelock)
% TODO: is this a fundamental? or write this when describing our attacker model?

\section{Network Coding}\label{sec:networkcodingfun}
Network coding is a technique for transmitting packets efficiently over a network. First described in 2000 by \citeauthor{ahlswede00networkflow}
\cite{ahlswede00networkflow}, the idea is to maximize the information flow through a network and achieve higher throughput than traditional transmission
schemes. It is achieved by allowing intermediate network nodes to encode incoming packets before passing them on, creating combinations of different
packets in the process. At the destinations, the received data can be decoded again to obtain the original packets. Traditionally, network coding is
used in multicast communication patterns, but it has also been applied successfully to unicast scenarios \cite[e.g.][]{moriam15manycorenc}.

A popular coding scheme, dubbed \textit{linear network coding} \cite{li03linearnc}, is to regard all packets arriving at a node from different incoming links as a vector.
Then, linear transformations can be applied to it to obtain new combinations to send out \cite[1]{li03linearnc}. To allow receivers to decode the
combinations into the original packets, the encoding patterns applied at each node can be \enquote{agreed upon beforehand} \cite[1]{li03linearnc}.
However, this requires global knowledge of the network topology. A practical alternative is to attach the encoding information to the packets in the
form of a \textit{global encoding vector (\gls{gev})} \cites[2\psqq]{chou03practicalnc}[5\psq]{chou07ncforinternetandwireless} that is updated at each intermediate node to
represent the current encoding pattern.

In addition to increasing the network performance by maximizing throughput, network coding can also provide an additional layer of resilience against
malicious intermediate nodes. It facilitates \enquote{a natural way to take advantage of multipath diversity for security against wiretapping attacks}
\cite[8]{fragouli07ncfundamentals} and can also be helpful against active attackers.

During this thesis, network coding is used primarily as a means of resilience against attackers. Furthermore, only the sender nodes will compute
combinations of different flits, while intermediate nodes merely forward them. Section (insert vref here) illuminates this subject in detail.
% TODO: mention that mostly used with multicast, we do unicast
% TODO: mention PNC paper, that we use RLNC, but only senders encode and create generations ("only intra-session network coding")
% TODO: insert vref

\section{Protection Goals}\label{sec:protectiongoals}
% Confidentiality, integrity
