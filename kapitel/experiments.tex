\section{Parameters}
In the experiments, a base network injection rate of 0.2 is assumed\footnote{The actual injection rate can be higher once \glspl{arq} come into play
and retransmissions are required.}
\begin{itemize}
    \item Injection rate
        \begin{itemize}
            \item Value if 0.2 is realistic
            \item Possibility to generate pairs for fair comparison of UC/NC
        \end{itemize}
    \item Overhead von Verschlüsselung+Auth vs. nur Auth vs keins von beiden (sowohl Latenz als auch Chipfläche)
\end{itemize}

\section{Attacker Model}
\begin{itemize}
    \item Focuses on malicious modifications rather than DoS attacks
    \item Assumption: compromised routers → rely on NIs for security
    \item No protection against bandwidth depletion, but this is not the goal here
    \item Variable number of compromised routers (e.g. 8 for an 8x8 grid)
    \item Compromised routers randomly drop or modify packets (no intelligent modifications/drops)
    \item Reasoning for having only some compromised routers (in regard to the 3rd party NoC problem → why would they not make all routers the same?)
\end{itemize}

\section{Environment}
% 50,000 cycles (why?)
% Warmup and cooldown times
% Injection rate of 0.2 into the network

\section{Determining The Hyperparameters}
\subsection{ARQ Timeouts}
Yexin: 8 cycles, but this is might be too low here because we consider internal delays and data races for the crypto modules within the network
interfaces → use as starting point

\subsection{Number Of Crypto Units}

\section{Experiments}
\subsection{Average time of a flit/transmission unit in ArrivalManager until a verification result is there}
\subsection{Number of ARQs modified/dropped by routers}
\subsection{Routing strategies effects}
% Heatmap of router workload?

\iffalse
Experiment setup parameter tables:
- NC mode (UC, G2C3, G2C4)
- ...

ARQ Limit: 1, at most 2 because more ARQs allowed per transmission unit means larger retransmission buffers everywhere
\fi
