\section{Design Considerations}
2D mesh as NoC topology is very common in research (insert many cites here).
\begin{itemize}
    \item Encryption/authentication ordering
        \begin{itemize}
            \item Encrypt-then-MAC: best practice. Sequential encrypt/authenticate on sender side, but parallel decrypt/verify
                on receiver side. Advantage: MAC can be computed on sender side immediately when ciphertext arrives, even when
                MAC flit has not arrived yet (if ARQ is necessary, it can be issued right away)
            \item MAC-then-encrypt: bad. Sequential authenticate/encrypt on sender side and sequential decrypt/verify on receiver
                side.
            \item Encrypt-and-MAC: okay. Parallel encrypt/authenticate on sender side, but sequential decrypt/verify on receiver
                side (overall same latency as Encrypt-then-MAC, but without advantage of fast ARQs)
        \end{itemize}
    \item GALS (Globally Asynchronous, Locally Synchronous)
        \begin{itemize}
            \item not relevant for simulations, just for actual hardware (e.g. power spikes on active clock edges, low-power PEs etc.)
            \item simulation is only inaccurate at the link between routers
            \item we just do Globally Synchronous because it's easier
        \end{itemize}
    \item 24 bit FIDs/GIDs
        \begin{itemize}
            \item integer overflow after $2^{24}-1$
            \item this is the latest time when session keys should be changed, otherwise packet injection (repeat attacks) become
                possible
        \end{itemize}
    \item Retransmission Buffer structure and lookup times
        \begin{itemize}
            \item corresponding flits are stored consecutively (e.g. data/MAC of same FID, flits of same generation etc.)
            \item lookup time (in clock cycles) is a parameter in the simulation
            \item UC case: one cycle lookup is fine (just need to find FID, mode field determines offset in the buffer)
            \item NC case: two cycles for lookup (one to find GID, one to compare GEVs of the generation in parallel, mode determines offset)
        \end{itemize}
    \item Priorities
        \begin{itemize}
            \item retransmission buffer: ARQs have priority
            \item crypto units (→ entry guard): arriving flits have priority
        \end{itemize}
    \item Lane widths
        \begin{itemize}
            \item lanes have as many wires as flits have bits
            \item one flit can be transmitted per clock cycle per lane
            \item flit size is fixed (standard header fields + 64 bit payload)
        \end{itemize}
    \item Buffers/Queues
        \begin{itemize}
            \item App/NI/Routers have only input buffers, no output buffers
            \item Routers only route flits when the receiving router's input queue is not full
        \end{itemize}
    \item Crypto units
        \begin{itemize}
            \item Separate units for encryption and decryption
            \item Send/receive pipeline share the same set of crypto units
            \item Talk about auth. method 3: 32 bit block size, what algorithms?
        \end{itemize}
    \item Routing Strategies
        \begin{itemize}
            \item XY/YX
                \begin{itemize}
                    \item Deterministic path
                    \item Attacker controlling a single router can reliably disrupt communication between certain nodes
                    \item does not distribute flits of a generation across different paths
                \end{itemize}
            \item XY/YX + Valiant
                \begin{itemize}
                    \item Deterministic path only if fixed valiant
                \end{itemize}
            \item Random XorY
            \item Random XorY + Valiant
        \end{itemize}
    \item Injection rate
        \begin{itemize}
            \item Value if 0.2 is realistic
            \item Possibility to generate pairs for fair comparison of UC/NC
        \end{itemize}
\end{itemize}

\section{Simulation Framework}

\section{Components}
\subsection{Network Interface}
