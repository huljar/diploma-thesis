\section{Design Considerations}
2D mesh as NoC topology is very common in research (insert many cites here).
\begin{itemize}
    \item GALS (Globally Asynchronous, Locally Synchronous)
        \begin{itemize}
            \item not relevant for simulations, just for actual hardware (e.g. power spikes on active clock edges, low-power PEs etc.)
            \item simulation is only inaccurate at the link between routers
            \item we just do Globally Synchronous because it's easier
        \end{itemize}
    \item 24 bit FIDs/GIDs
        \begin{itemize}
            \item integer overflow after $2^{24}-1$
            \item this is the latest time when session keys should be changed, otherwise packet injection (repeat attacks) become
                possible
            \item look at work paper for calculations about value ranges considered and why they chose 24 bit
            \item Continuous flit IDs: advantage of detecting fully lost transmission units, but for the simulation we use globally unique IDs for
                easier statistics recording → for fully lost units no ARQ is sent
        \end{itemize}
    \item Retransmission Buffer structure and lookup times
        \begin{itemize}
            \item corresponding flits are stored consecutively (e.g. data/MAC of same FID, flits of same generation etc.)
            \item lookup time (in clock cycles) is a parameter in the simulation
            \item UC case: one cycle lookup is fine (just need to find FID, mode field determines offset in the buffer)
            \item NC case: two cycles for lookup (one to find GID, one to compare GEVs of the generation in parallel, mode determines offset)
        \end{itemize}
    \item Priorities
        \begin{itemize}
            \item retransmission buffer: ARQs have priority
            \item crypto units (→ entry guard): arriving flits have priority
        \end{itemize}
    \item Lane widths
        \begin{itemize}
            \item lanes have as many wires as flits have bits
            \item one flit can be transmitted per clock cycle per lane
            \item flit size is fixed (standard header fields + 64 bit payload)
        \end{itemize}
    \item Buffers/Queues
        \begin{itemize}
            \item App/NI/Routers have only input buffers, no output buffers
            \item Routers only route flits when the receiving router's input queue is not full
        \end{itemize}
    \item Crypto units
        \begin{itemize}
            \item Separate units for encryption and authentication
            \item Encryption units can also decrypt → very easy to see with PRINCE
            \item Send/receive pipeline share the same set of crypto units
            \item Talk about auth. method 3: 32 bit block size, what algorithms?
            \item Latency: assume PRINCE → ~35MHz FPGA, *~4 for ASIC → paper. Look at PRINCE paper and survey paper for numbers
        \end{itemize}
    \item Tracking finished IDs: prevent repeat attacks, prevent re-processing a unit due to redundant retransmissions or repeat attacks, but still
        allow out-of-order arrivals SeemsGood
    \item Routing strategies: randomness needs hardware RNG → more complex logic, more area!
\end{itemize}

\section{Simulation Framework}
% Talk about OMNeT++

\section{Components}
\subsection{Network Interface}
% 1 paragraph per component
% at the end: table with all components, their delay (cycles)
\subsection{Routers}
% Only input buffers → single cycle routing
% Routing is only performed when receiving router's input queue is not full

\section{Configurable Parameters}
