Since the dawn of multiprocessor systems, the way/design of communicating/communication has been an integral part of processor design.
Multi-core chips are/were often based on bus communication, which quickly becomes impractical for many-core chips.
To confront this scaling problem, new ways of communication were developed, and thus the Network-on-Chip principle was devised.

With the emergence of network-on-chip as a scalable solution for inter-core or inter-processor communication, ...

However, as the popularity of \glspl{noc} increases, so does the interest of adversaries to compromise \glspl{mpsoc} that implement them as their communication
backbone.

Security is an essential component to NoC design. \citeauthor{ancajas14fortnocs} have shown that it is feasible to compromise a NoC with a very
minimal area and performance overhead, and that attack vectors are accessible e.g. in cloud computing setups. \cite{ancajas14fortnocs}

In this thesis, a novel approach to secure a NoC against adversaries in the hardware is explored.

\section{Background}\label{sec:background}
% Quickly describe HAEC project

\section{Motivation}\label{sec:motivation}
% Explain importance for HAEC, show image of HAEC cube, make clear that this thesis does not only apply to HAEC

The rest/remainder of this thesis is organized as follows.
