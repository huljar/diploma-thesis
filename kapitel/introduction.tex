The complexity of integrated circuits has been growing steadily since the dawn of the digital age. The amount of transistors employed on a single chip
continues to increase to this day, consistent with Moore's Law \cite{mack11mooreslaw}. The clock frequency of the systems, however, has not
experienced substantial growth rates for about a decade \cite{intelfrequency}. One of the main limiting factors here is the heat emission of the
circuits: as clock frequencies increase, the generated heat rises as well. At too high frequencies, the whole circuit might experience a physical
meltdown \cite{intelfrequency}.

Since the desire for more performance did not subside with the decelerating clock frequency growth, the trend has shifted towards parallelized
architectures, where multiple interconnected processor cores are placed on a single die \cite[6]{kumar08parallel}. As the level of parallelism has
increased substantially over the last years, leading from multi-core to many-core systems, the importance of an efficient interconnection architecture
capable of handling highly parallelized systems has equally grown.

Traditional interconnection architectures employ a global bus that all cores are attached to. However, as their number increases, buses quickly become
a bottleneck for the overall performance \cite[6]{tatas16designingnocs}. To confront this scaling problem, new ways of communicating were developed,
and thus the Network-on-Chip paradigm was devised \cites{kumar02networkonchip}{benini02nocparadigm}.

NoCs are a novel idea, seeking to overcome the drawbacks of traditional bus-based interconnect systems.

With the emergence of network-on-chip as a scalable solution for inter-core or inter-processor communication, ...

However, as the popularity of \glspl{noc} increases, so does the interest of adversaries to compromise \glspl{mpsoc} that implement them as their communication
backbone.

Security is an essential component to NoC design. \citeauthor{ancajas14fortnocs} have shown that it is feasible to compromise a NoC with a very
minimal area and performance overhead, and that attack vectors are accessible e.g. in cloud computing setups. \cite{ancajas14fortnocs}

In this thesis, a novel approach to secure a NoC against adversaries in the hardware is explored.

\section{Background}\label{sec:background}
% Quickly describe HAEC project
The TU Dresden research on network-on-chip security originates from the HAEC project.

\section{Motivation}\label{sec:motivation}
% Explain importance for HAEC, show image of HAEC cube, make clear that this thesis does not only apply to HAEC

The rest/remainder of this thesis is organized as follows.
% TODO: give one practical example of where and how a NoC/MPSoC is used
