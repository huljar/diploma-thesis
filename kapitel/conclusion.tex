In this thesis, a novel approach for securing the communications within a \gls{noc} against routers compromised with hardware trojans was presented
and evaluated. For this purpose, a protocol was designed that achieves the protection goals of confidentiality and integrity. After examining
and assessing related work from this field of research, an in-depth explanation of said protocol was given. Several performance indicators were then
evaluated through extensive simulations with varying parameters. To conduct these experiment, a dedicated software was developed that facilitates
cycle-accurate analyses with detailed statistics for all facets of the protocol. Finally, the most suitable configuration was determined based on the
obtained results.

Using PRINCE as a lightweight cipher for all cryptographic operations has proven to be a sound choice. Its ability to process a 64-bit block within two
clock cycles (at 500 MHz clock speed) allows for low-latency encryptions and authentications. Although its hardware implementation is
rather large with around 8260 \gls{ge}, relatively little parallel crypto modules per network interface are required (3 and 9 for encryption and
authentication, respectively). In the work by \citeauthor{moriam18activeattackers}, which this thesis builds upon (cf. Section \ref{sec:ncfornoc}),
mCrypton \cite{lim06mcrypton} was employed as the underlying cipher. Requiring 13 cycles per block, it necessitates 18 parallel modules per network
interface for authentication only \cite[5]{moriam18activeattackers}. However, with a size of 2681 \gls{ge}, it is significantly smaller than PRINCE.
In total, their approach consumes about half the chip area than the one proposed in this thesis, but provides no confidentiality and is slower by a
factor of 6.5.

Network coding provided the advantages that were envisioned. In unreliable networks, it was able to reduce the network load by up to 17\% and the
number of residual errors by up to 86\% at medium attack probabilities of 0.2\footnote{The percentages correspond to the \gls{iwa}-\gls{dor} protocol
with an \gls{arq} limit of 1, comparing UC and G2C4.}. However, the redundant transmissions also reduced the information rate by around 40\%. Thus,
while being a great asset in error-prone networks, it has a negative impact on the performance in reliable networks devoid of adversaries.

The ability to request retransmissions via \glspl{arq} is crucial to provide a possibility to recover from integrity breaches and still facilitate
successful transmissions of source flits. Since the \glspl{arq} and retransmissions themselves are also vulnerable to attackers, doubling
the number of allowed \glspl{arq} further improved the resilience of the protocol. With the same configurations as above, it decreased the amount of
residual errors by 23\% at the cost of a network load increase of 7\%.

There are still many avenues of improvement that have not been explored here as they would go beyond the scope of this thesis. Nonetheless, they may
be of interest and spark inspiration for future work that draws upon the presented ideas. The following list provides a selection of such avenues:
\begin{itemize}
    \item \textbf{Burst mode.} As the flits already contain a currently unused bit to indicate burst mode, the investigation of the presented
        protocols and routing strategies in combination with this mode suggests itself.
    \item \textbf{Detecting full transmission unit loss.} As each pair of sender and receiver is assumed to use its own sequence of flit and
        generation IDs, it is possible to detect the loss of complete transmission units if the stream of IDs is discontinuous on the receiving side.
        In this case, an \gls{arq} could be issued that requests all flits associated with the missing ID.
    \item \textbf{Protecting the \gls{arq} flits.} In the designed protocol, \glspl{arq} are neither authenticated nor network coded. Hence, they are
        one of the most vulnerable parts of the protocol: in case of an attack on an \gls{arq} flit, the retransmissions fail. Adding authentication
        or redundancy to \glspl{arq} may help to decrease the residual error probabilities even further, albeit at the cost of a higher network load.
    \item \textbf{Different paths for retransmissions.} To increase the probability of successful retransmissions, it may be beneficial to ensure that
        they take different paths than the initially sent flits, even if the resulting routes are not minimal. If a specific flit is requested via an
        \gls{arq}, its original path could be prohibited for the subsequent retransmission.
    \item \textbf{Varying traffic models.} The employed traffic generation pattern uses a uniform random distribution of created flits
        among the processing elements. In real-world applications, nodes may have differing injection rates, which influences the network load
        distribution. For such scenarios, dynamic routing strategies may provide stronger beneficial effects over static \gls{dor}.
    \item \textbf{Different attacker model.} The attacker model for the evaluation in this thesis is based on random flit drops and modifications. An
        adversary, however, may for instance be interested in disrupting only specific communications with a certain source or destination, and then
        attack every matching flit.
    \item \textbf{Further network coding schemes.} While a generation size of two has proven to be beneficial over larger ones
        \cite{moriam15manycorenc}, the composition of the generations may be altered. For instance, an interesting approach would be to include
        \glspl{mac} of the source flits inside the generations instead of authenticating the encoded combinations. Then, for each source flit, a
        generation could be formed from the data and \gls{mac} flit.
    \item \textbf{Truncated \glspl{mac} for \gls{iwa}.} In the presented protocol variant \gls{iwa}, authcodes are used to authenticate flits, which
        are computed from a pseudo-random intermediate key and the data part of the payload. An alternative would be to compute a \gls{mac} over the
        header fields and the 32 payload bits and then use only the first half of the result as authentication information. This would
        save one clock cycle per authentication procedure as the final step of the authcode computation (see Figure \ref{fig:computeauthcodeuc}) could
        be omitted. However, truncating a \gls{mac} in such a manner may entail a weaker overall security.
    \item \textbf{Protecting the address header field.} The address header field of the flits, which contains a 32 bit memory address, remains unused
        for this thesis. However, if it is utilized in the future, it may be desirable to encrypt it together with the payload. Since memory layouts
        are not necessarily random, it may convey interesting information to an eavesdropping adversary.
    \item \textbf{Virtual channels.} For this thesis, virtual channels were not considered in the investigated dynamic routing strategies. For a
        real-world implementation, however, they are necessary to ensure deadlock freedom. Hence, it may be desirable to evaluate the effects of their
        presence for future work with dynamic routing.
\end{itemize}
\vspace{0.5\baselineskip}

All in all, the protection measures investigated and evaluated in this thesis provide a promising approach to secure and efficient communications for
\glspl{noc}. Encryption and authentication with a lightweight cipher are able to fulfill the goals of confidentiality and integrity for the
transmitted information while network coding adds a layer of redundancy and resilience. Furthermore, dynamic routing strategies entail the usage of
multiple paths even among the same communication partners, but their potential for performance enhancements depends on the network load distribution. There are still many open
avenues to further improve the presented schemes; the auspicious results so far seem encouraging for future research in this direction.
