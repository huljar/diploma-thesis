In this thesis, a novel approach for securing the communications within a \gls{noc} against compromised routers with hardware trojans was presented
and evaluated. For this purpose, a protocol was designed that aims to ensure the protection goals of confidentiality and integrity. After examining
and assessing related work from this field of research, an in-depth explanation of said protocol was given. Several performance indicators were then
evaluated through extensive simulations with varying parameters. To conduct these experiment, a dedicated software was developed that facilitates
cycle-accurate analyses with detailed statistics for all facets of the protocol. Finally, the most suitable configuration was determined based on the
obtained results.

Future work:
\begin{itemize}
    \item Sequential IDs for uncoded and network coded → full transmission unit loss can be detected, ARQ issued
    \item Authenticated ARQs
    \item Different attacker models/smarter attackers
    \item If attacker model changes, i.e. attackers start to drop specific/whole generations,
        how does that influence the Routing/ARQ design? → ability to detect fully lost flits/gens via continuous IDs
    \item My idea of G3C4 with generation MAC as part of the G3? → it is feasible, would require experiments → G3C4 or G3C5? → refer to Yexin's
        experiments
    \item Authenticated encryption schemes in the NIs
    \item Local network coding, local encoding vectors
    \item Burst mode (w/ head and tail flits)
    \item TDM route selecting
    \item RNG for path selection in simulator → according to \cite{stefan11enhancingnocs}, non-static selection can be very expensive in area →
        explore static routes?
    \item Intelligent attackers → modify ARQs or send own ARQs → they are unauthenticated so DoS attack by maximizing retransmissions
    \item Usage of ECC (error correcting codes) to potentially reduce number of ARQs (at least for accidental bit flips)
    \item Security of the encryption → we kind of use ECB here which is not secure → e.g. a nonce is needed? or the flits need to be chained with some
        mode of operation like CBC or CTR → CBC requires no residual errors
    \item Mixing G2C3 and G2C4 -- e.g. once a network interface detects that it receives an unusually high amount of ARQs, indicating a very
        unreliable network, network coding could be \enquote{upgraded} from G2C3 to G2C4, so it is not statically one or the other
    \item Int. auth: adapt the scheme to compute MAC over the data part as well as just use the first 32 bit of the result as the "authcode". Saves a
        clock cycle because the cogwheel lookup function is omitted and input still fits into two blocks, but maybe security implications?
    \item Encryption (and encoding) of the address field → it may convey information useful to an attacker since the memory layout of the PEs is probably not random
        (→ ASLR?)
    \item Application of NC to ARQs (interwoven variant)
    \item Head-of-line blocking in routers → future work w/ virtual channels/TDM etc.
\end{itemize}
