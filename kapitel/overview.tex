% 1. novel approach, why? protection goals and performance → security and efficiency
% 2. what precisely was done at the TUD chair → say that their research proved the potential of NC and that's why we persue it here again
% 3. NC provides no confidentiality/integrity guarantees → we implement encryption+authentication
% 3.5 recall NoC architecture → this is implemented in the network interfaces
% 4. Different variants envisioned (3 methods) + comparison with uncoded variant where applicable
% 5. Multiple paths focus → why multiple paths in the first place? → chance to avoid compromised routers, still have enough flits
%    thanks to NC
% 6. Routing strategies: deterministic vs. non-deterministic routing → XY, smart random XY, ROMM+XY, ROMM+srXY
% 7. Attacker/Threat model
% 8. Evaluation
% Always refer to the chapters that explain this in detail
As mentioned in Chapter \ref{ch:introduction}, this thesis aims to pursue a novel approach for providing secure and efficient communication in
\glspl{noc}.

This thesis follows up on previous work done at the \thechair. In 2015, the effect of network coding
on communications in a partially compromised \gls{noc} was evaluated and discussed. Now, the emphasis lies on combining network coding with
cryptographic measures to fulfill the desired protection goals (see Section (vref to fundamentals)). % TODO: move this to introduction?
% Mention that NoC will be simulated
