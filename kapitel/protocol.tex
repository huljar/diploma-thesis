To accomplish the goals 

This chapter will illuminate the details of how encryption, authentication, network coding, and ARQ management are implemented in the network
interfaces.

Recall that the architecture of a \gls{noc} consists of a router grid with each node connected to a local processing element through a network
interface (cf. Figure \vref{fig:nocexample}). The 

There are different variants of the protocol, which are explained in detail in Section (insert vref here). While the emphasis lies on network coded
variants, uncoded versions have also been implemented for comparison purposes. However, not all methods have a corresponding uncoded variant.
% TODO: this goes to the overview chapter

\section{Flit Structure}
% Figure of the flit (bar) with header fields (for coded+uncoded)
% Explain purpose of header fields

\section{ARQ Structure}

\section{Network Coding}
% Why G2C3 and G2C4? → refer to previous TUD papers

\section{Routing Strategies}
\begin{itemize}
    \item Routing Strategies
        \begin{itemize}
            \item XY/YX
                \begin{itemize}
                    \item Deterministic path
                    \item Attacker controlling a single router can reliably disrupt communication between certain nodes
                    \item does not distribute flits of a generation across different paths
                \end{itemize}
            \item XY/YX + Valiant
                \begin{itemize}
                    \item Deterministic path only if fixed valiant
                \end{itemize}
            \item Random XorY
            \item Random XorY + Valiant
            \item When writing about this: MANHATTAN DISTANCE (same for XY, YX, random XY, ROMM)
        \end{itemize}
\end{itemize}

\section{Protocol Variants}
Three different communication schemes were envisioned.

\subsection{Uncoded separate data/MAC flits}
\subsection{Network coded separate data/MAC flits}
\subsection{Uncoded data/MAC splits}
\subsection{Network coded data/MAC splits}
\subsection{Network coded data flits with generation MAC}

\section{Notes}
\begin{itemize}
    \item Encryption/authentication ordering
        \begin{itemize}
            \item Encrypt-then-MAC: best practice. Sequential encrypt/authenticate on sender side, but parallel decrypt/verify
                on receiver side. Advantage: MAC can be computed on receiver side immediately when ciphertext arrives, even when
                MAC flit has not arrived yet (if ARQ is necessary, it can be issued right away)
            \item MAC-then-encrypt: bad. Sequential authenticate/encrypt on sender side and sequential decrypt/verify on receiver
                side.
            \item Encrypt-and-MAC: okay. Parallel encrypt/authenticate on sender side, but sequential decrypt/verify on receiver
                side (overall same latency as Encrypt-then-MAC, but without advantage of fast ARQs)
        \end{itemize}
    \item Uncoded transmission
        \begin{itemize}
            \item no network coding
            \item 2 methods: 1 data flit + 1 MAC flit OR 2 data/MAC split flits
        \end{itemize}
    \item Flit structure
        \begin{itemize}
            \item burst bit, source/target address, mode, address, GID/FID, GEV, payload
            \item mode: define if data/mac/split/arq
        \end{itemize}
    \item Network coded transmission
        \begin{itemize}
            \item Number of flits: G2C3 or G2C4
            \item 3 methods: 1 data flit + 1 MAC flit OR 1 MAC flit per generation OR 2 data/MAC split flits
            \item mention that coded clits are slightly larger due to GEV being embedded → requires wider lanes
        \end{itemize}
    \item ARQs
        \begin{itemize}
            \item Limited number of ARQs per transmission unit (UC: data/MAC pair or split pair, NC: generation)
            \item Timeout of x (e.g. 8) cycles until first ARQ is sent
            \item If limit >1: start larger timeout (→ round-trip of ARQ)
            \item Many different cases, insert some flow diagrams here
            \item The higher the ARQ timeout/limit, the less likely the flit is still in retransmission buffer
            \item → ARQ timeout/limit and retransmission buffer size have to correlate
            \item In the case that we only have 1 ARQ left that we are allowed to send: Wait for any ongoing MAC verifications
                so in case they fail, the flits can be included in the ARQ
        \end{itemize}
\end{itemize}
