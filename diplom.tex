% !TeX program = lualatex
% !TeX spellcheck = en_US
\documentclass[
	paper=a4,
	%open=right, % Chapters start on right pages
	%twoside=true,
	fontsize=11pt,
	parskip=full % Space between paragraphs
]{scrreprt}

% % % Polyglossia % % %
\usepackage{polyglossia}
\setmainlanguage{german}

\usepackage{csquotes}

% % % BibLaTeX % % %
\usepackage[
	%abbreviate=false, % Don't abbreviate standard bibliography terms
	backend=biber, % Bibliography engine
	citestyle=numeric-comp, % Style for citations
	bibstyle=numeric, % Style for bibliography
	date=terse, % Shorter dates
	ibidtracker=false, idemtracker=false, opcittracker=false, citetracker=false, % Don't abbreviate when same citation twice in a row
	doi=false, % Don't print the following fields in the bibliography, unless required by the entry type
	isbn=false,
	url=false,
	giveninits=true, % Render first and middle names as initials
	uniquename=init, % Prevent using initials for authors
	maxcitenames=2, % Maximum number of authors to use in citations
	maxbibnames=99 % Print all authors in bibliography
]{biblatex}

\bibliography{quellen}

\setlength{\bibitemsep}{.7\baselineskip} % Empty lines between literature sources

\renewcommand{\labelnamepunct}{\addcolon\addspace}

% % % VarioRef % % %
\usepackage{varioref}

% % % GraphicX % % %
\usepackage{graphicx}
\graphicspath{{bilder/}}

% % % EnumItem % % %
\usepackage{enumitem}
\setitemize{itemsep=-.5\parskip, topsep=-.5\baselineskip}
\setenumerate{itemsep=-.5\parskip, topsep=-.5\baselineskip}

% % % Titling % % %
\usepackage{titling}

% % % Caption % % %
\usepackage[font={small,it}]{caption}

% % % amssymb % % %
\usepackage{amssymb}

% % % MathTools % % %
\usepackage{mathtools}

% % % ChangeCounter % % %
\usepackage{chngcntr}
\counterwithout{footnote}{chapter} % Global footnote indices

% % % EPStoPDF % % %
%\usepackage{epstopdf}

% % % Color % % %
\usepackage{color}

% % % SIunitX % % %
%\usepackage[group-separator={,}]{siunitx}

% % % Rahmendaten % % %
\author{Julian Harttung}
\title{Sichere und effiziente Datenübertragung für Network-on-Chip unter Nutzung multipler Pfade}
\newcommand{\thesubtitle}{Diplomarbeit}
\newcommand{\theuniversity}{Technische Universität Dresden}
\newcommand{\thefaculty}{Fakultät Informatik}
\newcommand{\theinstitute}{Institut für Systemarchitektur}
\newcommand{\thechair}{Professur für Datenschutz und Datensicherheit}
% % % Rahmendaten Ende % % %

\begin{document}
    \frenchspacing % Disable double spaces between sentences
	\begin{titlepage}
		\includegraphics[width=0.28\textwidth]{header_logo_tud}
		\hfill
		\includegraphics[width=0.28\textwidth]{header_logo_haec} % TODO: find HD HAEC logo
		\vspace{2\baselineskip}
		
		\begin{center}
			\textsc{\theuniversity \\
					\thefaculty \\
					\theinstitute \\
					\thechair}
			\vspace{4\baselineskip}
		
			\Huge{\thetitle}
			\vspace{.5\baselineskip}
			
			\LARGE{\thesubtitle}
		\end{center}
		
		\vfill
		
		\begin{tabular}{ll}
			Autor:           & \theauthor \\
			Studiengang:     & Diplom-Informatik \\
			Matrikelnummer:  & 3753196 \\
			Betreuer:        & Dr.-Ing. Elke Franz und Dipl.-Inf. Paul Walther \\
			Hochschullehrer: & Prof. Dr. Thorsten Strufe \\
			\multicolumn{2}{l}{ } \\
			\multicolumn{2}{l}{ } \\
			\multicolumn{2}{l}{ } \\
			\multicolumn{2}{l}{Dresden, 11.\ April 2018} % TODO: adjust date
		\end{tabular}
	\end{titlepage}
	
	
	\pagenumbering{roman}
	
	\chapter*{Aufgabenstellung}
    Lorem ipsum
	
	\chapter*{Selbstständigkeitserklärung}
	Hiermit erkläre ich, dass ich die von mir am heutigen Tag dem Prüfungsausschuss der Fakultät Informatik eingereichte Arbeit zum Thema:
	\begin{center}
		\textit{\thetitle} 
	\end{center}
	
	vollkommen selbstständig verfasst und keine anderen als die angegebenen Quellen und Hilfsmittel benutzt sowie Zitate kenntlich gemacht habe.
	
	Dresden, 11.\ April 2018 \\ % TODO: adjust date
	\theauthor
	
	
	\chapter*{Abstract}
    Lorem ipsum
	
	\tableofcontents
	
	\addtocontents{lot}{\protect\vspace{-1.4\baselineskip}}
	\addtocontents{lof}{\protect\vspace{-1.4\baselineskip}}
	
	\listoftables
	\vspace{-2.6\baselineskip}
	\begingroup
	\let\clearpage\relax
	\listoffigures
	\endgroup
	
	
	\chapter{Einleitung}\label{ch:einleitung}
	\pagenumbering{arabic}
    Lorem ipsum

    \chapter{Notizen}
    \section{NoC-Design}
    \begin{itemize}
        \item Router: in+out buffer vs. only input buffer (2 cycles vs. 1 cycle routing)
        \item GenTraffic: injection rate of 0.2 was considered in Sadia's calculations and is a good starting point
        \item Network Interface: sender and receiver share the crypto units → we need to keep track of where the flit has to go after leaving the crypto unit → flag at crypto unit itself?
        \item Speed: NoC usually runs at ~500 MHz or a bit lower, but definitely runs a lot faster than the maximum frequency of the lightweight encryption algorithms. Using multiple clock domains is hard and requires expertise
        \item Crypto algorithms: 18 units for mCrypton was planned; also possible to use multiple (but less) PRINCE units, but then the clock domain problem arises again
	
	% % % Bibliography % % %
	\nocite{*} % Put all entries in the bibliography, not only those cited in the document
	\printbibliography
\end{document}
