% !TeX program = lualatex
% !TeX spellcheck = en_US
\documentclass[
	paper=a4,
	%open=right, % Chapters start on right pages
	%twoside=true,
	fontsize=11pt,
	parskip=full % Space between paragraphs
]{scrreprt}

% % % Polyglossia % % %
\usepackage{polyglossia}
\setmainlanguage[variant=american]{english}

\usepackage{csquotes}

% % % BibLaTeX % % %
\usepackage[
	%abbreviate=false, % Don't abbreviate standard bibliography terms
	backend=biber, % Bibliography engine
	citestyle=numeric-comp, % Style for citations
	bibstyle=numeric, % Style for bibliography
	date=terse, % Shorter dates
	ibidtracker=false, idemtracker=false, opcittracker=false, citetracker=false, % Don't abbreviate when same citation twice in a row
	doi=false, % Don't print the following fields in the bibliography, unless required by the entry type
	isbn=false,
	url=false,
	giveninits=true, % Render first and middle names as initials
	uniquename=init, % Prevent using initials for authors
	maxcitenames=2, % Maximum number of authors to use in citations
	maxbibnames=99 % Print all authors in bibliography
]{biblatex}

\bibliography{quellen}

\setlength{\bibitemsep}{.7\baselineskip} % Empty lines between literature sources

\renewcommand{\labelnamepunct}{\addcolon\addspace}

% % % Bookmark % % %
\usepackage[open,openlevel=1]{bookmark}

% % % VarioRef % % %
\usepackage{varioref}

% % % GraphicX % % %
\usepackage{graphicx}
\graphicspath{{bilder/}}

% % % Glossaries % % %
\usepackage[toc,nonumberlist]{glossaries}
\newglossaryentry{mpsoc}{
    name = {MPSoC},
    description = {Multi-Processor System-on-Chip},
    plural = {MPSoCs}
}
\newglossaryentry{noc}{
    name = {NoC},
    description = {Network-on-Chip},
    plural = {NoCs}
}

\makeglossaries

% % % EnumItem % % %
\usepackage{enumitem}
\setitemize{itemsep=-.5\parskip, topsep=-.5\baselineskip}
\setenumerate{itemsep=-.5\parskip, topsep=-.5\baselineskip}

% % % Titling % % %
\usepackage{titling}

% % % Caption % % %
\usepackage[font={small,it}]{caption}

% % % amssymb % % %
\usepackage{amssymb}

% % % MathTools % % %
\usepackage{mathtools}

% % % ChangeCounter % % %
\usepackage{chngcntr}
\counterwithout{footnote}{chapter} % Global footnote indices

% % % EPStoPDF % % %
%\usepackage{epstopdf}

% % % Color % % %
\usepackage{color}

% % % SIunitX % % %
%\usepackage[group-separator={,}]{siunitx}

% % % Rahmendaten % % %
\author{Julian Harttung}
\title{Sichere und effiziente Datenübertragung für Network-on-Chip unter Nutzung multipler Pfade}
\newcommand{\thesubtitle}{Diplomarbeit}
\newcommand{\theuniversity}{Technische Universität Dresden}
\newcommand{\thefaculty}{Fakultät Informatik}
\newcommand{\theinstitute}{Institut für Systemarchitektur}
\newcommand{\thechair}{Professur für Datenschutz und Datensicherheit}
% % % Rahmendaten Ende % % %

\begin{document}
    \frenchspacing % Disable double spaces between sentences
	\begin{titlepage}
		\includegraphics[width=0.28\textwidth]{header_logo_tud}
		\hfill
		\includegraphics[width=0.28\textwidth]{header_logo_haec} % TODO: find HD HAEC logo
		\vspace{1.5\baselineskip}
		
		\begin{center}
			\textsc{\theuniversity \\
					\thefaculty \\
					\theinstitute \\
					\thechair}
			\vspace{2.5\baselineskip}
		
			\Huge{\thetitle}
			\vspace{.5\baselineskip}
			
			\LARGE{\thesubtitle}
		\end{center}
		
		\vfill
		
		\begin{tabular}{ll}
			Autor:           & \theauthor \\
			Studiengang:     & Diplom-Informatik \\
			Matrikelnummer:  & 3753196 \\
			Betreuer:        & Dr.-Ing. Elke Franz und Dipl.-Inf. Paul Walther \\
			Hochschullehrer: & Prof. Dr. Thorsten Strufe \\
			\multicolumn{2}{l}{ } \\
			\multicolumn{2}{l}{ } \\
			\multicolumn{2}{l}{ } \\
			\multicolumn{2}{l}{Dresden, 11.\ April 2018} % TODO: adjust date
		\end{tabular}
	\end{titlepage}
	
	
	\pagenumbering{roman}
	
	\chapter*{Task}
    Lorem ipsum
	
	\chapter*{Selbstständigkeitserklärung}
	Hiermit erkläre ich, dass ich die von mir am heutigen Tag dem Prüfungsausschuss der Fakultät Informatik eingereichte Arbeit zum Thema:
	\begin{center}
		\textit{\thetitle} 
	\end{center}
	
	vollkommen selbstständig verfasst und keine anderen als die angegebenen Quellen und Hilfsmittel benutzt sowie Zitate kenntlich gemacht habe.
	
	Dresden, 11.\ April 2018 \\ % TODO: adjust date
	\theauthor
	
	
	\chapter*{Abstract}
    Lorem ipsum
	
	\tableofcontents
	
	\addtocontents{lot}{\protect\vspace{-1.4\baselineskip}}
	\addtocontents{lof}{\protect\vspace{-1.4\baselineskip}}
	
	\listoftables
	\vspace{-2.6\baselineskip}
	\begingroup
	\let\clearpage\relax
	\listoffigures
	\endgroup
	
	
	\chapter{Introduction}\label{ch:introduction}
	\pagenumbering{arabic}
    Since the dawn of multiprocessor systems, the way/design of communicating/communication has been an integral part of processor design.
    Multi-core chips are/were often based on bus communication, which quickly becomes impractical for many-core chips.
    To confront this scaling problem, new ways of communication were developed, and thus the Network-on-Chip principle was devised.

    With the emergence of network-on-chip as a scalable solution for inter-core or inter-processor communication, ...

    Security is an essential component to NoC design. \citeauthor{ancajas14fortnocs} have shown that it is feasible to compromise a NoC with a very
    minimal area and performance overhead, and that attack vectors are accessible e.g. in cloud computing setups. \cite{ancajas14fortnocs}

    In this thesis, a novel approach to secure a NoC against adversaries in the hardware is explored.

    The rest/remainder of this thesis is organized as follows.

    \chapter{Fundamentals}\label{ch:fundamentals}

    \chapter{Related Work}\label{ch:relatedwork} % TODO: filter papers by number of citations and conferences

    \section{Network-on-Chip}\label{sec:networkonchip}
    Research on new and efficient ways to interconnect components on a single chip has been an important field of research for decades. The concept of
    general-purpose on-chip networks has been introduced in the early 2000s
    \cites{dally01routepacketsnotwires}{kumar02networkonchip}{benini02nocparadigm} and has quickly gained a lot of traction in the research community
    \cite[e.g.][]{ivanov05nocintroduction}. With ever-increasing design complexity of modern chips \cite{mack11mooreslaw}, specialized on-chip
    interconnections become infeasible to implement; designing such a system "would take too much time and mapping of applications to dedicated architectures would
    be impossible" \cite[1]{kumar02networkonchip}. Thus, the \gls{noc} approach was devised to "facilitate [...] modularity by defining a
    standard interface" \cite[1]{dally01routepacketsnotwires}.

    As the popularity of \glspl{noc} increases, so does the interest of adversaries to compromise \glspl{mpsoc} that implement them as their communication
    backbone. In recent research, many different attack vectors have been explored, and a variety of countermeasures has been proposed to mitigate
    attacks. 

    \Glspl{mpsoc} that utilize \glspl{noc} typically have a large number of processing elements that can run many different
    tasks in parallel. Especially in cloud computing setups, untrusted applications run in parallel (cite paper mentioning cloud here).

    \subsection{Software-Based Attacks on \glspl{mpsoc}}\label{subsec:softwareattacks}

    \subsection{Hardware Trojans in \glspl{noc}}\label{subsec:hardwaretrojans}

    \begin{itemize}
        \item \textbf{\citetitle{ivanov05nocintroduction}}
            \begin{itemize}
                \item Aus dem Jahr 2005, als SoC-Kommunikationsschwierigkeiten wichtiger wurden
                \item Wird als langfristiger Einstiegspunkt in NoC-Forschung gesehen von den Autoren
            \end{itemize}
        \item \textbf{\citetitle{sethumadhavan15trustworthyhardware}}
            \begin{itemize}
                \item Introduction into hardware design process and compromisation vectors
                \item Explains how the hardware design and fabrication chain is vulnerable to exploits/attacks
                \item Three security systems operating "in series" (next one is only coming into play if previous one has failed)
                    \begin{enumerate}
                        \item Static check that the design being used is backdoor-free
                        \item Runtime altering of inputs (→ obfuscation) to ensure backdoors are not triggered/turned on
                        \item Runtime on-chip monitoring (of instruction counts, opcode types, ...) to detect enabled backdoors
                    \end{enumerate}
            \end{itemize}
        \item \textbf{\citetitle{ancajas14fortnocs}}
            \begin{itemize}
                \item MPSoCs with 3rd party IP NoCs (i.e. the interconnect system is 3rd party)
                \item Software accomplices (malicious/infected processing elements)
                \item Attack types: eavesdropping (information leak), voluntary data corruption, denial of service
                \item Fort-NoCs: 3-layer security mechanism (hardware level protection)
                    \begin{itemize}
                        \item Lower layer data scrambling (hardware encryption to prevent covert activation sequences from AcTh to Trojan)
                        \item Middle layer packet certification (authentication tag, detect unintended destination after flit copy)
                        \item Top layer node obfuscation (migrate running applications from one node to another)
                    \end{itemize}
                \item Malicious PE must secretly communicate with hardware trojan to send commands (C\&C node)
                \item Easy to run malicious software on a PE e.g. in cloud computing setups
                \item Small area and power overhead, mostly small runtime overhead
                \item Not all layers need to be used (in lower security domains)
            \end{itemize}
        \item \textbf{\citetitle{frey15stateobfuscation}}
            \begin{itemize}
                \item Attacker model: HT is the FSM control unit of NIs (very specific HT location)
                \item Countermeasure: obfuscate the states and state transistions that the FSMs do
                \item HT modifying state transitions causes FSM to enter illegal/invalid state → HT warning
                \item High HT detection rate (for this specific type of HT)
            \end{itemize}
        \item \textbf{\citetitle{frey17hardenednoc}}
            \begin{itemize}
                \item Published two years after state obfuscation paper above
                \item Router level hardware trojans (HTs)
                \item Focuses on DoS attacks (bandwidth depletion) originating in a router (not a NI because router has more connections → more
                    feasible)
                \item Implement DoS mitigation directly in the routers, rather than NI, to prevent bandwidth depletion as quickly as possible
                \item Physically Unclonable Function (PUF): random vector generation in each router
                \item Apply random dynamic permutation (data scrambling) to flits arriving at a router input (makes modifying flits into something
                    meaningful significantly harder) before flit reaches the input queue (where the HT has access); de-permutate at output port (→
                    PUF random vectors)
                \item Apply ECC (error control code) encoding before input port; decode before output port (only critical flit bits: header, tail,
                    dest. address)
                \item Check flit integrity after leaving input queue and right before departing through the computed output port
                \item Cites lots of useful other related work
            \end{itemize}
        \item \textbf{\citetitle{fernandes16nocrouting}}
            \begin{itemize}
                \item "Attacks at MPSoC aim to extract sensitive data, modify the system behavior or denial the system operation (Denial-of-Service,
                    DoS)"
                \item Build security zones in the NoC using routing algorithm ("wrap IPs and protect sensitive information from attacker")
                \item Firewalls also possible, but may be costly (→ they implement a security policy in the NI)
                \item Aims to protect against software-based attacks (NoC is assumed to be secure)
                \item Threat model: timing and DoS attacks
                \item Security zone is e.g. the set of IP blocks that an application was mapped on
                \item Routing algorithm tries to keep the sensitive path completely inside the same security zone, if possible
            \end{itemize}
        \item \textbf{\citetitle{boraten16packetsecurity}}
            \begin{itemize}
                \item Packet-Security (P-Sec)
                \item Threat model: compromised NoC does fault injection (side channel attack)
                \item It is possible to eventually obtain encryption keys by observing how encoders and decoders react to the side channel attacks
                \item → ensure integrity of packets using error correction codes (ECCs) (→ AMD, CRC)
                \item AMD for sensitive communications (together with encryption), otherwise CRC to provide minimal fault tolerance
            \end{itemize}
        \item \textbf{\citetitle{boraten18mitigationdos}}
            \begin{itemize}
                \item Published 2 year after Packet Security paper above (builds upon previous research)
                \item HT does DoS attack: inspect packets, inject fault, trigger ECC response (ECC cannot correct error) → repeated transmissions,
                    deadlocks
                \item HT resides in links between nodes
                \item Prevention: Heuristic fault classification → discover HTs
                \item Continue using compromised links instead of rerouting → obfuscation to prevent HT triggering, optimized AMDs to detect fault
                    injections
                \item Little overhead: 2\% area, 6\% power
                \item "[...] we can classify security threats for NoCs as a subset of preexisting challenges originating from but not limited to,
                    on-chip fault tolerance, functional correctness, path diversity, isolation, and quality of service"
                \item Security measures should not be compromised themselves
            \end{itemize}
        \item \textbf{\citetitle{biswas15routerattack}}
            \begin{itemize}
                \item Survey of MPSoC attack types
                \item New attack type for routing table-based routers (i.e. reconfigurable routers as opposed to routers with fixed routing logic)
                \item Mentions survey of hardware trojan detection techniques
                \item Not about detecting HTs, but about protection from malicious users
                \item → TEEs and REEs (Trusted/Rich Execution Environments), similar to security zones
                \item It is desirable to use routing tables instead of fixed routing logic (flexibility, more complex routing algorithms)
                \item Attack scenario: routing table is loaded onto NoC at boot or runtime (by host processor or NoC controller), which is modified by
                    the attacker → unauthorized access and misrouting (routing to other environment)
            \end{itemize}
        \item \textbf{\citetitle{gebotys03securityframework}}
            \begin{itemize}
                \item Framework: protection both at network and application layer
                \item Network layer
                    \begin{itemize}
                        \item Key-keeper core: protects/distributes encryption keys to other secure cores
                        \item Each secure core has a security wrapper
                        \item Focus on key distribution and key management
                    \end{itemize}
                \item Application layer
                    \begin{itemize}
                        \item Software modifications for resistance against power (side-channel) attacks
                    \end{itemize}
                \item Higher level approach than most other papers (more protocol layer than hardware layer)
                \item Strong assumptions on trusted software and hardware
                \item No clear attacker model, paper seems more like a "framework suggestion"
            \end{itemize}
        \item \textbf{\citetitle{kapoor13nocauthenc}}
            \begin{itemize}
                \item 2 NoC zones: secure and non-secure IP cores
                \item Authenticated Encryption implemented in NIs of secure cores
                \item Secure cores can communicate with each other using permanent keys
                \item Non-secure cores can communicate with each other using plain text
                \item Hardware (NIs + routers) are assumed to be secure
                \item Secure and non-secure cores communicate with session keys and an intermediate link IP core (link can be secure or non-secure)
                \item Memory IP cores have access rights table in NI to prevent unauthorized memory accesses
                \item DoS attacks prevented by having a max number of packets allowed to be sent implemented in NI
            \end{itemize}
        \item \textbf{\citetitle{evain05nocsecurityanalysis}}
            \begin{itemize}
                \item In their context: CCM (central configuration module) is added (unique IP block → initialize and (re)configure NoC). Also CCM:
                    add supervising and defending reactions for security
                \item FPGA vs. ASIC: reconfigurability of FPGA is another potential attack vector
                \item Mixed FPGA/ASIC implementation possible: ASIC for secure zone, FPGA for insecure zone (CCM must be in secure zone)
                \item Many possible attack types → different protection strategies
                    \begin{itemize}
                        \item Bandwidth denial: virtual channels in the secure area (unsecure packets can't obstruct secure packets)
                        \item Unauthorized access: packet/path filters at zone boundaries and/or at NIs
                        \item Only encrypted/authenticated communication with the CCM
                    \end{itemize}
            \end{itemize}
        \item \textbf{\citetitle{stefan11enhancingnocs}}
            \begin{itemize}
                \item Introduce non-determinism through multipath routing
                \item Proposal is implemented on top of Aethereal framework
                \item Time-division multiplexing (TDM) for router channels
                \item Alternative path selection
                    \begin{itemize}
                        \item … based on position in the slot table at the moment of sending (static schedule)
                        \item … based on hardware RNG (dynamic at runtime)
                    \end{itemize}
            \end{itemize}
    \end{itemize}

    Different attacker/threat models in literature. Depending on the attacker model, different approaches are used to protect the system against it.
    E.g. when the underlying network architecture (the NoC itself) is assumed to be compromised, protection is implemented in the network interfaces
    of the nodes. If the attacker only has access to specific parts of the routers or specific zones of the NoC, protection can be implemented through
    the routing algorithm. → The power of the HTs differs. The more complex the HT is, the stronger it influences chip area/power consumption/runtime
    overhead and may be easier detectable → that's why HTs are often assumed to use "small" attacks like fault injection, or have access to only very
    specific components of the NoC to stay undetected/not require much chip area.

    Differentiate between methods to detect HTs (on software level, firmware level, w/ static analysis, side-channel analysis), and methods to harden
    the NoC against potential HT infections.

    NI is usually assumed to be trusted, and routers are potentially compromised because of 3rd party IP or 3rd party manufacturing/integration
    partners. Other threat model: software attacks (NoC itself is secure).

    How to get HT into hardware: rogue employee, 3rd party IP, 3rd party manufacturing/integration partners, ...

    This thesis: no hardware synthesis, use software simulation. Focus on malicious flit modification (→ attacker model) rather than DoS attacks. Our
    model assumes that the NoC routers and links may be compromised and thus relies on the NIs for the security measures → no effective protection
    against bandwidth depletion, but this is not the goal.
    Deterministic vs. static routing? No security zones or division into secure/non-secure zones/cores.
    
    The concept of security zones can be implemented in different ways. Bla et al. propose to do X, while bla enforce them through the routing protocol.

    \chapter{The HAEC Project}

    \chapter{Simulation Setup}
    \section{Node Layout}

    \section{Attacker Model}
    \begin{itemize}
        \item Variable number of compromised routers (e.g. 8 for an 8x8 grid)
        \item Compromised routers randomly drop or modify packets (no intelligent modifications/drops)
        \item Future work: If attacker model changes, i.e. attackers start to drop specific/whole generations,
            how does that influence the Routing/ARQ design?
    \end{itemize}

    \chapter{NoC Design}
    2D mesh as NoC topology is very common in research (insert many cites here).
    \begin{itemize}
        \item Encryption/authentication ordering
            \begin{itemize}
                \item Encrypt-then-MAC: best practice. Sequential encrypt/authenticate on sender side, but parallel decrypt/verify
                    on receiver side. Advantage: MAC can be computed on sender side immediately when ciphertext arrives, even when
                    MAC flit has not arrived yet (if ARQ is necessary, it can be issued right away)
                \item MAC-then-encrypt: bad. Sequential authenticate/encrypt on sender side and sequential decrypt/verify on receiver
                    side.
                \item Encrypt-and-MAC: okay. Parallel encrypt/authenticate on sender side, but sequential decrypt/verify on receiver
                    side (overall same latency as Encrypt-then-MAC, but without advantage of fast ARQs)
            \end{itemize}
        \item GALS (Globally Asynchronous, Locally Synchronous)
            \begin{itemize}
                \item not relevant for simulations, just for actual hardware (e.g. power spikes on active clock edges, low-power PEs etc.)
                \item simulation is only inaccurate at the link between routers
                \item we just do Globally Synchronous because it's easier
            \end{itemize}
        \item 24 bit FIDs/GIDs
            \begin{itemize}
                \item integer overflow after $2^{24}-1$
                \item this is the latest time when session keys should be changed, otherwise packet injection (repeat attacks) become
                    possible
            \end{itemize}
        \item Retransmission Buffer structure and lookup times
            \begin{itemize}
                \item corresponding flits are stored consecutively (e.g. data/MAC of same FID, flits of same generation etc.)
                \item lookup time (in clock cycles) is a parameter in the simulation
                \item UC case: one cycle lookup is fine (just need to find FID, mode field determines offset in the buffer)
                \item NC case: two cycles for lookup (one to find GID, one to compare GEVs of the generation in parallel, mode determines offset)
            \end{itemize}
        \item Priorities
            \begin{itemize}
                \item retransmission buffer: ARQs have priority
                \item crypto units (→ entry guard): arriving flits have priority
            \end{itemize}
        \item Lane widths
            \begin{itemize}
                \item lanes have as many wires as flits have bits
                \item one flit can be transmitted per clock cycle per lane
                \item flit size is fixed (standard header fields + 64 bit payload)
            \end{itemize}
        \item Buffers/Queues
            \begin{itemize}
                \item App/NI/Routers have only input buffers, no output buffers
                \item Routers only route flits when the receiving router's input queue is not full
            \end{itemize}
        \item Crypto units
            \begin{itemize}
                \item Separate units for encryption and decryption
                \item Send/receive pipeline share the same set of crypto units
            \end{itemize}
        \item Routing Strategies
            \begin{itemize}
                \item XY/YX
                    \begin{itemize}
                        \item Deterministic path
                        \item Attacker controlling a single router can reliably disrupt communication between certain nodes
                        \item does not distribute flits of a generation across different paths
                    \end{itemize}
                \item XY/YX + Valiant
                    \begin{itemize}
                        \item Deterministic path only if fixed valiant
                    \end{itemize}
                \item Random XorY
                \item Random XorY + Valiant
            \end{itemize}
        \item Injection rate
            \begin{itemize}
                \item Value if 0.2 is realistic
                \item Possibility to generate pairs for fair comparison of UC/NC
            \end{itemize}
    \end{itemize}

    \chapter{Communication Protocol}
    \begin{itemize}
        \item Uncoded transmission
            \begin{itemize}
                \item no network coding
                \item 2 methods: 1 data flit + 1 MAC flit OR 2 data/MAC split flits
            \end{itemize}
        \item Flit structure
            \begin{itemize}
                \item burst bit, source/target address, mode, address, GID/FID, GEV, payload
                \item mode: define if data/mac/split/arq
            \end{itemize}
        \item Network coded transmission
            \begin{itemize}
                \item Number of flits: G2C3 or G2C4
                \item 3 methods: 1 data flit + 1 MAC flit OR 1 MAC flit per generation OR 2 data/MAC split flits
            \end{itemize}
        \item ARQs
            \begin{itemize}
                \item Limited number of ARQs per transmission unit (UC: data/MAC pair or split pair, NC: generation)
                \item Timeout of x (e.g. 8) cycles until first ARQ is sent
                \item If limit >1: start larger timeout (→ round-trip of ARQ)
                \item Many different cases, insert some flow diagrams here
                \item The higher the ARQ timeout/limit, the less likely the flit is still in retransmission buffer
                \item → ARQ timeout/limit and retransmission buffer size have to correlate
                \item In the case that we only have 1 ARQ left that we are allowed to send: Wait for any ongoing MAC verifications
                    so in case they fail, the flits can be included in the ARQ
            \end{itemize}
    \end{itemize}

    \chapter{Statistics}
    \begin{itemize}
        \item Injection/acceptance rate: [0, 1] (at processing element and at network interface)
        \item Queue lengths and buffer sizes
        \item Workload of crypto units
        \item Average/max flit waiting time at entry guard
        \item Average/max hop count
    \end{itemize}

    \chapter{Implementation}
    \section{Course of Events}
    \begin{itemize}
        \item lalala
    \end{itemize}

    \section{Components}
    \subsection{Network Interface}

    \chapter{Future Work}
    \begin{itemize}
        \item Local network coding, local encoding vectors
        \item Burst mode
    \end{itemize}

    \chapter{Talk About}
    \begin{itemize}
        \item Encrypt-then-MAC vs. Encrypt-and-MAC (bei Encrypt-then-MAC encrypt+authenticate sequentiell, aber decrypt+verify parallel.
            Bei Encrypt-and-MAC encrypt+authenticate parallel, aber decrypt+verify sequentiell). Bei Encrypt-and-MAC: decrypt kann schon
            beginnen, wenn MAC noch nicht da ist, verringert die Latenz in manchen Fällen.
        \item Overhead von Verschlüsselung+Auth vs. nur Auth vs keins von beiden (sowohl Latenz als auch Chipfläche)
        \item Welche Komponenten brauchen wie viele Takte
        \item Anzahl der Enc/Auth units bei den verschiedenen Methoden (mehr enc als auth bei Methode 2)
        \item GALS design pattern - wir sind aber auf jeden Fall synchron innerhalb einer Node, d.h. wenn GALS vs GS diskutiert wird,
            ändert sich nur die Zeit der Übertragung zwischen den Routern → zur Vereinfachung wird GS angenommen
        \item Leichtgewichtige Krypto: FPGA vs. ASIC, speedup von Faktor 3-4 wahrscheinlich möglich (→ critical path delay) \cite{kuon07fpgavsasic}
            → AES und AES\_INV zum Vergleich, da ähnlicher Aufbau von kombinatorischer Logik + FlipFlops (Rundenfunktionen etc.)
        \item Statistiken: Wie viele Crypto Units für Auth+Enc sind gleichzeitig besetzt (→ Auslastung)
        \item Threat model, protection goals
        \item node-unique FIDs/GIDs have advantage that 24 bits are not full nearly as fast as with globally unique IDs
		\item large flits (128 bits) need 2 crypto units (w/ block size 64 bit)
        \item Lack of error correcting code (ECC) in simulation
        \item Deadlock/Livelock possibilities? D yes, L not
    \end{itemize}

    \section{Assumptions}
    \begin{itemize}
        \item only input buffers are used (for app, NI, router), no output buffers
        \item network coding module stores flits until enough flits with the same
            destination are available (in our case: 2)
        \item network coding (creating all 3 combinations) takes one cycle, but
            because we can only send one flit out per cycle, it takes 3 cycles until
            all combinations are sent (so in the end it's one combination per cycle)
        \item we have a variable number of authentication units and the amount of
            cycles the encryption algorithm takes can be set by a parameter
        \item if all crypto units are busy, the encoded flit combinations are held
            back in a dedicated buffer
        \item if e.g. authentication method 1 is used (1 MAC flit per data flit),
            the data flit is sent out one cycle before the MAC finished computing,
            so the MAC can be sent immediately when it's ready
        \item the encoding/authentication and decoding/verification pipelines in the
            network interface share the same crypto units (but this is not
            implemented yet)
        \item when a flit arrives at the NI from the router, the first thing that is
            done is checking if this is an ARQ. If yes, it is delivered directly to
            the retransmission buffer. Otherwise, it goes through the normal network
            decoding pipeline
        \item when an ARQ arrives at a NI, it takes one cycle to look up the correct
            flit and it will be sent out in the same cycle (not sure about this, two
            cycles might be more realistic)
        \item size of the retransmission buffer is configurable
        \item retransmission buffers use a FIFO caching strategy
        \item if an ARQ and a new data flit arrive a the retransmission buffer
            during the same clock cycle, the ARQ answer (retransmission) has
            priority to be sent out first
        \item in a router, several flits can be routed simultaneously, provided that
            they do not share an input or output port
        \item also in router: flits are held back in the input queues if the
            receiving router's input queue is full (this was discussed in an earlier
            email)
        \item in uncoded flits, the GID header contains a flit ID instead of
            generation ID, and MAC and data flit have the same ID
        \item the different values for the mode header are: normal data flit, mac
            flit, data/mac combined, ARQ
        \item entry guard can distribute a departing and an arriving flit at the same
            time, as long as units are free
        \item it does not matter whether data or mac flit is sent out first (order),
            because they can arrive out-of-order at the destination node anyway
        \item encryption units can also be used for decryption
    \end{itemize}

    \chapter{Notes}
    \section{NoC Design}
    \begin{itemize}
        \item Router: in+out buffer vs. only input buffer (2 cycles vs. 1 cycle routing)
        \item GenTraffic: injection rate of 0.2 was considered in Sadia's calculations and is a good starting point
        \item Network Interface: sender and receiver share the crypto units → we need to keep track of where the flit has to go after leaving the crypto unit → flag at crypto unit itself?
        \item Speed: \gls{noc} usually runs at ~500 MHz or a bit lower, but definitely runs a lot faster than the maximum frequency of the lightweight encryption algorithms. Using multiple clock domains is hard and requires expertise
        \item Crypto algorithms: 18 units for mCrypton was planned; also possible to use multiple (but less) PRINCE units, but then the clock domain problem arises again
    \end{itemize}

    % % % Glossary % % %
    %\glsaddall % Print all glossary entries, not only the referenced ones
    \printglossaries
	
	% % % Bibliography % % %
	%\nocite{*} % Put all entries in the bibliography, not only those cited in the document
    \printbibliography[heading=bibintoc]
\end{document}
