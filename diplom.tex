% !TeX program = lualatex
% !TeX spellcheck = en_US
\documentclass[
	paper=a4,
	%open=right, % Chapters start on right pages
	%twoside=true,
	fontsize=11pt,
	parskip=full % Space between paragraphs
]{scrreprt}

% % % Polyglossia % % %
\usepackage{polyglossia}
\setmainlanguage[variant=american]{english}

% % % csquotes % % %
\usepackage{csquotes}

% % % BibLaTeX % % %
\usepackage[
	%abbreviate=false, % Don't abbreviate standard bibliography terms
	backend=biber, % Bibliography engine
	citestyle=numeric-comp, % Style for citations
	bibstyle=numeric, % Style for bibliography
	date=terse, % Shorter dates
	ibidtracker=false, idemtracker=false, opcittracker=false, citetracker=false, % Don't abbreviate when same citation twice in a row
	doi=false, % Don't print the following fields in the bibliography, unless required by the entry type
	isbn=false,
	url=false,
	giveninits=true, % Render first and middle names as initials
	uniquename=init, % Prevent using initials for authors
	maxcitenames=2, % Maximum number of authors to use in citations
	maxbibnames=99 % Print all authors in bibliography
]{biblatex}

\bibliography{quellen}

\setlength{\bibitemsep}{.7\baselineskip} % Empty lines between literature sources

\renewcommand{\labelnamepunct}{\addcolon\addspace} % Colon between author and title in bibliography

% General settings
\setcounter{tocdepth}{2} % Show chapters, sections, and subsection in the table of contents
\interfootnotelinepenalty=10000 % Prevent breaking of footnotes across pages

% % % Bookmark % % %
\usepackage[open,openlevel=1]{bookmark}

% % % VarioRef % % %
%\usepackage{varioref}
\newcommand\vref{\ref} % TODO: disabling varioref for now because it causes unstable typesetting

% % % GraphicX % % %
\usepackage{graphicx}
\graphicspath{{bilder/}{grafiken/}}

% % % SVG % % %
% Since we use lualatex, some pdftex commands need to be imported first
%\usepackage{pdftexcmds}
%\makeatletter
%\let\pdfstrcmp\pdf@strcmp
%\let\pdffilemoddate\pdf@filemoddate
%\makeatother

%\usepackage[svgpath=grafiken/]{svg}

% % % TabularY % % %
\usepackage{tabulary}
\usepackage{multirow}
\usepackage{hhline}

% % % Glossaries % % %
\usepackage[toc,nonumberlist]{glossaries}
\newglossaryentry{mpsoc}{
    name = {MPSoC},
    description = {Multi-Processor System-on-Chip},
    plural = {MPSoCs}
}
\newglossaryentry{noc}{
    name = {NoC},
    description = {Network-on-Chip},
    plural = {NoCs}
}

\makeglossaries

% % % EnumItem % % %
\usepackage{enumitem}
\setitemize{itemsep=-.5\parskip, topsep=-.5\baselineskip}
\setenumerate{itemsep=-.5\parskip, topsep=-.5\baselineskip}

% % % Titling % % %
\usepackage{titling}

% % % Caption % % %
\usepackage[font={small,it}]{caption}

% % % math packages % % %
\usepackage{mathtools}
\usepackage{amsmath}
\usepackage{amssymb}

% % % ChangeCounter % % %
\usepackage{chngcntr}
\counterwithout{footnote}{chapter} % Global footnote indices

% % % EPStoPDF % % %
%\usepackage{epstopdf}

% % % Color % % %
\usepackage{color}

% % % SIunitX % % %
\usepackage[group-separator={,}]{siunitx}

% % % Rahmendaten % % %
\author{Julian Harttung}
\title{Secure and Efficient Communication for Network-on-Chip under the Consideration of Multiple Paths}
\newcommand{\thesubtitle}{Diploma Thesis}
\newcommand{\theuniversity}{Dresden University of Technology}
\newcommand{\thefaculty}{Faculty of Computer Science}
\newcommand{\theinstitute}{Institute of Systems Architecture}
\newcommand{\thechair}{Chair of Privacy and Data Security}
% % % Rahmendaten Ende % % %

% Hilfsmakros
\newcommand{\omnet}{OMNeT++}

\begin{document}
    \frenchspacing % Disable double spaces between sentences
	\begin{titlepage}
		\includegraphics[width=0.28\textwidth]{header_logo_tud}
		\hfill
		\includegraphics[width=0.28\textwidth]{header_logo_haec} % TODO: find HD HAEC logo
		\vspace{1.5\baselineskip}
		
		\begin{center}
			\textsc{\theuniversity \\
					\thefaculty \\
					\theinstitute \\
					\thechair}
			\vspace{2.5\baselineskip}
		
			\Huge{\thetitle}
			\vspace{.5\baselineskip}
			
			\LARGE{\thesubtitle}
		\end{center}
		
		\vfill
		
		\begin{tabular}{ll}
			Author:               & \theauthor \\
			Course of Studies:    & Diplom-Informatik (PO 2010)\\
			Matriculation Number: & 3753196 \\
			Supervisors:          & Dr.-Ing. Elke Franz and Dipl.-Inf. Paul Walther \\
			Professor:            & Prof. Dr. Thorsten Strufe \\
			\multicolumn{2}{l}{ } \\
			\multicolumn{2}{l}{ } \\
			\multicolumn{2}{l}{ } \\
			\multicolumn{2}{l}{Dresden, June 28, 2018}
		\end{tabular}
	\end{titlepage}
	
	
	\pagenumbering{roman}
	
	\chapter*{Task}
    Lorem ipsum
	
	\chapter*{Selbstständigkeitserklärung}
	Hiermit erkläre ich, dass ich die von mir am heutigen Tag dem Prüfungsausschuss der Fakultät Informatik eingereichte Arbeit zum Thema:
	\begin{center}
		\textit{\thetitle} 
	\end{center}
	
	vollkommen selbstständig verfasst und keine anderen als die angegebenen Quellen und Hilfsmittel benutzt sowie Zitate kenntlich gemacht habe.
	
	Dresden, June 28, 2018 \\
	\theauthor
	
	
	\chapter*{Abstract}
    Lorem ipsum
	
	\tableofcontents
	
	\addtocontents{lot}{\protect\vspace{-1.4\baselineskip}}
	\addtocontents{lof}{\protect\vspace{-1.4\baselineskip}}
	
	\listoftables
	\vspace{-2.6\baselineskip}
	\begingroup
	\let\clearpage\relax
	\listoffigures
	\endgroup
	
	
	\chapter{Introduction}\label{ch:introduction}
	\pagenumbering{arabic}
    The complexity of integrated circuits has been growing steadily since the dawn of the digital age. The amount of transistors employed on a single chip
continues to increase to this day, consistent with Moore's Law \cite{mack11mooreslaw}. The clock frequency of the systems, however, has not
experienced substantial growth rates for about a decade \cite{intelfrequency}. One of the main limiting factors here is the heat emission of the
circuits: as clock frequencies increase, the generated heat rises as well. At too high frequencies, the whole circuit might experience a physical
meltdown \cite{intelfrequency}.

Since the desire for more performance did not subside with the decelerating clock frequency growth, the trend has shifted towards parallelized
architectures, where multiple interconnected processor cores are placed on a single die \cite[6]{kumar08parallel}. As the level of parallelism has
increased substantially over the last years, leading from multi-core to many-core systems, the importance of an efficient interconnection architecture
capable of handling highly parallelized systems has equally grown.

Traditional interconnection architectures employ a global bus that all cores are attached to. However, as their number increases, buses quickly become
a bottleneck for the overall performance \cite[6]{tatas16designingnocs}. To confront this scaling problem, new ways of communicating were developed,
and thus the Network-on-Chip paradigm was devised \cites{kumar02networkonchip}{benini02nocparadigm}.

NoCs are a novel idea, seeking to overcome the drawbacks of traditional bus-based interconnect systems.

With the emergence of network-on-chip as a scalable solution for inter-core or inter-processor communication, ...

However, as the popularity of \glspl{noc} increases, so does the interest of adversaries to compromise \glspl{mpsoc} that implement them as their communication
backbone.

Security is an essential component to NoC design. \citeauthor{ancajas14fortnocs} have shown that it is feasible to compromise a NoC with a very
minimal area and performance overhead, and that attack vectors are accessible e.g. in cloud computing setups. \cite{ancajas14fortnocs}

In this thesis, a novel approach to secure a NoC against adversaries in the hardware is explored.

\section{Background}\label{sec:background}
% Quickly describe HAEC project
The TU Dresden research on network-on-chip security originates from the HAEC project.

\section{Motivation}\label{sec:motivation}
% Explain importance for HAEC, show image of HAEC cube, make clear that this thesis does not only apply to HAEC

The rest/remainder of this thesis is organized as follows.
% TODO: give one practical example of where and how a NoC/MPSoC is used

    
    \chapter{Fundamentals}\label{ch:fundamentals}
    \section{Networks-On-Chip}\label{sec:networkonchipfun}
\textit{Networks-on-Chip} (or \textit{\glspl{noc}} for short) are a method of interconnecting components on a chip. Typically employed on
\textit{Multi-Processor Systems-on-Chip (\glspl{mpsoc})} \cites(e.g.)(){ivanov05nocintroduction}{biswas15routerattack}{tatas16designingnocs}, they
provide the communication infrastructure for \textit{processing elements (\glspl{pe})} and possibly other \gls{ip} cores.

The topology of a \gls{noc} can vary. Researchers usually work with a 2D mesh topology
\cites(e.g.)(){frey17hardenednoc}{kumar02networkonchip}{fernandes16nocrouting}{boraten16packetsecurity}, which will also be used throughout this thesis.
In this case, each network node is connected to its four neighbors (excluding the boundary nodes).

A node typically consists of a processing element, a network interface (\textit{\gls{ni}}), and a router. \cite{dally01routepacketsnotwires} The
router manages the connections to neighboring nodes while also allowing the local processing element to communicate with the network through the
network interface. An example of this architecture is given in Figure \vref{fig:nocexample}.

Compared to traditional bus-based interconnect systems, \glspl{noc} can provide a lot of advantages, especially for many-core systems.
\cite[5\psqq]{tatas16designingnocs} One big advantage is scalability; because the cores do not share a global bus, \enquote{local performance is not
degraded} \cite[6]{tatas16designingnocs} as more components are added, and \enquote{aggregated bandwidth scales with the network size}
\cite[6]{tatas16designingnocs}.

In addition, the absence of global connections facilitates the use of different clock domains. This enables the implementation of the
\textit{globally asynchronous, locally synchronous (\gls{gals})} paradigm, which becomes increasingly important in chip design.
\cites[3]{kumar02networkonchip}[2]{ivanov05nocintroduction}

Furthermore, with the constantly increasing design complexity of modern chips \cite{mack11mooreslaw}, specialized on-chip
interconnections become infeasible to implement. Designing such a system \enquote{would take too much time and mapping of applications to dedicated
architectures would be impossible} \cite[1]{kumar02networkonchip}. In contrast, \glspl{noc} aims to be general purpose interconnect systems; they
\enquote{facilitate […] modularity by defining a standard interface} \cite[1]{dally01routepacketsnotwires}.

\begin{figure}
    \centering
    \includegraphics[width=0.5\textwidth]{noc_3x3_colored}
    \caption[Example of a 3x3 NoC]{Example of a 3x3 Network-on-Chip. The processing elements (red) contain a network interface
    (green), through which they are connected to a router (blue). The routers are connected in a 2D mesh topology.}
    \label{fig:nocexample}
\end{figure}

\section{Flits}\label{sec:flits}
Flits (short for \textit{flow control units}) % TODO: explain that there are flit headers/body

\section{Hardware Trojans}\label{sec:hardwaretrojans}
% What are HTs, why can they get into other hardware, what are their properties
Hardware trojans are \enquote{malicious modifications of electronic hardware} \cite[1]{bhunia14hardwaretrojans} with the intent of disrupting normal
system behavior or extract sensitive information. Because the integration of third party \gls{ip} has become increasingly popular due to circuit
complexity and cost efficiency \cites[1]{ancajas14fortnocs}[2]{bhunia14hardwaretrojans}, it is possible for adversaries to introduce hardware
trojans into larger systems, such as \glspl{mpsoc}.

In order to remain undetected, attackers aim to construct hardware trojans that are \enquote{stealthy in nature} \cite[1]{bhunia14hardwaretrojans}
and \enquote{evade […] detection through conventional postmanufacturing test} \cite[1]{bhunia14hardwaretrojans}. Hardware trojans are usually in a
dormant state until they are activated by a trigger signal to carry out their task. \cites{bhunia14hardwaretrojans}{ancajas14fortnocs} While the
trojan is inactive, communications through the \gls{noc} are unaffected and the system operates normally.

Attack types: information leak/eavesdropping, DoS (→ bandwidth depletion, deadlock, livelock)
% TODO: is this a fundamental? or write this when describing our attacker model?

\section{Network Coding}\label{sec:networkcodingfun}


    \chapter{Related Work}\label{ch:relatedwork}
    % TODO: tenses?
\section{Network-On-Chip Security}\label{sec:networkonchipsecurity}
Research on new and efficient ways to interconnect components on a single chip has been an important field of research for decades. The concept of
general-purpose on-chip networks has been introduced in the early 2000s
\cites{dally01routepacketsnotwires}{kumar02networkonchip}{benini02nocparadigm} and has quickly gained a lot of traction in the research community
\cite[e.g.][]{ivanov05nocintroduction}. 
%\Glspl{mpsoc} that utilize \glspl{noc} typically have a large number of processing elements that can run many different
%tasks in parallel. \citeauthor{ancajas14fortnocs} have shown that the threat is relevant, especially in cloud computing setups, where several
%untrusted applications may run at the same time \cite{ancajas14fortnocs}. % TODO: move this out of related work? also drop the untrusted applications part

In recent research, many different attack vectors on \gls{noc} architectures have been explored, and a variety of countermeasures have been proposed
to mitigate attacks. The remainder of this section will explore them in detail.

% HT Survey & building trustworthy hardware
\citeauthor{bhunia14hardwaretrojans} \cite{bhunia14hardwaretrojans} have looked thoroughly into the threat of hardware trojans and possible protection
approaches. In a survey-like paper, they provide a detailed summary of attack scenarios, countermeasures, and detection paradigms. Similarly,
\citeauthor{sethumadhavan15trustworthyhardware} \cite{sethumadhavan15trustworthyhardware} analyze the challenge of building systems from untrusted
hardware components. They explain in detail how the hardware design and fabrication chain can be adapted to significantly lower the probability of
integrating malicious components. The methods in both works are not specific to \gls{noc} architectures, but are applicable to them nonetheless.

% Security zones
The necessity of security measures as part of the design was already recognized early on in \gls{noc} research. A popular approach is to
divide the \gls{noc} into security zones with different levels of protection, where sensitive information is handled exclusively inside a
high-security zone \cites(e.g.)(){gebotys03securityframework}{fernandes16nocrouting}{kapoor13nocauthenc}.

Published in 2003, \citeauthor{gebotys03securityframework} \cite{gebotys03securityframework} were among the first to propose a \enquote{general security architecture}
\cite[1]{gebotys03securityframework} to impede attacks \enquote{at both the network level […] and at the core level}
\cite[1]{gebotys03securityframework}. At the network level, they differentiate between secure cores and other cores, thus establishing two security
zones. The secure cores are capable of encrypting and authenticating network
traffic, and are thus designed to handle sensitive user information. In addition, there is a dedicated key-keeper core that handles key distribution
amongst the secure cores. At the core level (or application level), the authors propose to use a modified implementation of elliptic curve
cryptography to facilitate encryption and authentication. Aiming to provide resistance against side channel power attacks, their modifications conceal
the power traces of the different algebraic computations during the cryptographic operations. This hinders adversaries from extracting key bits based
on power spikes.

\citeauthor{kapoor13nocauthenc} \cite{kapoor13nocauthenc} have pursued a similar approach. They also separate the cores into secure and non-secure
ones and propose to implement authenticated encryption in the network interfaces. While the secure
cores employ permanent keys to communicate with each other, the non-secure ones may use plain text messages. Additionally, in order to allow
communication between the security zones, sessions may be established with individually generated session keys. Furthermore, the authors employ
traffic limitations in the network interfaces to prevent \gls{dos} attacks by malicious cores, while access rights tables prohibit unauthorized
memory accesses.

Another work that revolves around security zones in \glspl{noc} was published by \citeauthor{fernandes16nocrouting} \cite{fernandes16nocrouting}.
However, in contrast to the papers presented above, they deem the underlying hardware to be secure. Assuming that \enquote{software attacks account
for 80\% of security incidents in embedded systems} \cite[1]{fernandes16nocrouting}, their work focuses on the defense against software-based
\gls{dos} and timing attacks.

The goal of the authors is to ensure that packets from one security zone are not routed through nodes of a different zone, if possible. To achieve
this, they have adapted the routing algorithm of the network to prioritize paths that do not cross zone boundaries. In addition, they have
refined the algorithm to guarantee deadlock freedom despite the constraints imposed by the security zones.

% Fort-NoCs
\citeauthor{ancajas14fortnocs} \cite{ancajas14fortnocs} have investigated the threat of hardware trojans for \glspl{noc}. They show that the usage of
third party \gls{noc} \gls{ip} is very popular due to cost efficiency, opening up an infection vector for hardware
trojans. Together with a software accomplice (i.e. an infected processing element) that can send commands to the trojan, this may
lead to information leaks, data corruption or denial of service attacks.

The authors focus on mitigating \enquote{covert data theft by a compromised \gls{noc}} \cite[3]{ancajas14fortnocs}. They suggest a three-layer
approach to mitigate this threat, consisting of data scrambling, packet certification, and node obfuscation. These techniques are
implemented solely in the network interfaces, which are not provided by a third party and hence assumed to be trustworthy. The goal of these measures
is to prevent activation of the hardware trojan and render transmitted information unreadable to the attacker.

% Hardened NoC design
\citeauthor{frey17hardenednoc} \cite{frey17hardenednoc} also worked on mitigating the effect of hardware trojans in a \gls{noc}. Their goal is to
harden the \gls{noc} design against potential hardware trojans located inside the routers. In contrast to \citeauthor{ancajas14fortnocs}
\cite{ancajas14fortnocs}, the protective measures are implemented at the router level and not in the network interfaces. They are designed to
\enquote{complement […] the previous \gls{noc} works aiming for \gls{ni} security} \cite[16]{frey17hardenednoc} and address \gls{dos} attacks rather
than information leakage.

The idea of the authors is to detect any flit tampering right after the flit leaves a router. To achieve this, an error control code and dynamic flit
permutation are applied to all flits before they enter a router, and the reverse transformations are applied after they exit the router again. This
prevents (or at least detects) small and targeted modifications to the flit headers. % TODO: not the best wording

% AMD codes
\citeauthor{boraten16packetsecurity} pursue a very similar approach. In their 2016 paper \cite{boraten16packetsecurity}, they propose to apply
\gls{amd} and \gls{crc} codes in the network interfaces to protect packets from fault injections by a malicious \gls{noc}. The authors suggest to use
\gls{amd} codes for sensitive data and \gls{crc} for \enquote{all other non-critical traffic} \cite[2]{boraten16packetsecurity}. First introduced in 2008
\cite{cramer08amdcodes}, \gls{amd} codes are capable of \enquote{[detecting] any tampering by an adversary} \cite[1]{cramer08amdcodes}.

Following up on their previous work, the authors published a 2018 paper \cite{boraten18mitigationdos} that refines their methods to faciliate hardware
trojan detection and mitigation. In contrast to many of the previously presented works \cites(e.g.)(){ancajas14fortnocs}{frey17hardenednoc}, they aim
to first detect a potential hardware trojan before activating further security measures. The described attacker model employs fault injections to
intentionally trigger responses from the error correction codes (like the \gls{amd} code explored earlier \cite{boraten16packetsecurity}), thus
performing a \gls{dos} attack.

Trojans are discovered with a \enquote{heuristic-based fault detection model} \cite[25]{boraten18mitigationdos} that classifies faults into accidental
and intentional ones. Once this scheme detects the presence of a trojan, additional security measures are employed. First, the authors try to keep
using the malicious links in order to not degrade network performance. Before a packet is routed through an infected area, it is obfuscated to prevent
the hardware trojan from triggering. If this fails, and faults are still being injected, \enquote{links must be disabled and routers should route
around them} \cite[32]{boraten18mitigationdos}.

\citeauthor{stefan11enhancingnocs} \cite{stefan11enhancingnocs} explore the potential of multipath routing as a security enhancement in \glspl{noc}.
Extending the existing \textit{\AE thereal} framework \cite{goossens05aethereal}, they implemented time-division multiplexing for choosing a route at
the sender's network interface. Two variants were pursued: static, deterministic path selection and dynamic selection at run-time (e.g. by using a
hardware random number generator). Through several experiments, they conclude that the static variant has significantly less chip area overhead than
the dynamic one and thus should be the preferred method.

Also focusing on time-division multiplexed \glspl{noc}, \citeauthor{wassel13surfnoc} \cite{wassel13surfnoc} promote the usage of different domains for
the network traffic with strict non-interference requirements. Categorizing packets into such domains \enquote{help[s] prevent cascading failures}
\cite[1]{wassel13surfnoc} and hampers \gls{dos} attacks from affecting the whole system. In their work, the authors propose an efficient method of
this domain-based routing with very low latency, making this security measure practical.

\section{Network Coding For Network-On-Chip}
Few attempts have been made to integrate network coding into \gls{noc} architectures. It was briefly mentioned by
\citeauthor{fragouli08ncapplications} \cite{fragouli08ncapplications}, where emerging network coding applications are discussed. However, it was
mostly seen as a method to help \enquote{simplify and minimize the length of on-chip wiring} \cite[260]{fragouli08ncapplications}, rather than as a
security measure.

\citeauthor{xue15ncnoc} \cite{xue15ncnoc} have integrated network coding into a \gls{noc} architecture for multicast communications. They propose to
send both encoded and regular flits of the same packets into the network, creating a redundant transmission. This introduces benefits during the
routing of the flits, like allowing to drop some flits to potentially prevent network congestion.

\section{Lightweight Cryptographic Algorithms}\label{sec:lightweightcrypto}
For security reasons, it is often desirable to add encryption and authentication to the communication passing through a \gls{noc}, as many of the
researchers mentioned above have done. However,
standard cryptographic algorithms such as \gls{aes} are usually not efficient enough for this task \cite[1]{bogdanov07present}.

In the work preceding this thesis, lightweight cryptographic algorithms were thoroughly explored \cite{harttung17lightweightcrypto}. Such
algorithms are
specifically designed to have efficient hardware implementations with low area and power requirements. In addition, they aim to have a small
computation delay while still providing an adequate level of security. Examples of such algorithms are PRESENT \cite{bogdanov07present},
mCrypton \cite{lim06mcrypton}, PRINCE \cite{borghoff12prince} and Klein \cite{gong12klein}. Some of them will be examined later in this
thesis; see Section (insert vref here) for details. % TODO: reference this when explaining PRINCE later on

\section{Notes}
\begin{itemize}
    \item \textbf{\citetitle{ivanov05nocintroduction}} \checkmark
        \begin{itemize}
            \item Aus dem Jahr 2005, als SoC-Kommunikationsschwierigkeiten wichtiger wurden
            \item Wird als langfristiger Einstiegspunkt in NoC-Forschung gesehen von den Autoren
        \end{itemize}
    \item \textbf{\citetitle{sethumadhavan15trustworthyhardware}} \checkmark
        \begin{itemize}
            \item Introduction into hardware design process and compromisation vectors
            \item Explains how the hardware design and fabrication chain is vulnerable to exploits/attacks
            \item Three security systems operating "in series" (next one is only coming into play if previous one has failed)
                \begin{enumerate}
                    \item Static check that the design being used is backdoor-free
                    \item Runtime altering of inputs (→ obfuscation) to ensure backdoors are not triggered/turned on
                    \item Runtime on-chip monitoring (of instruction counts, opcode types, ...) to detect enabled backdoors
                \end{enumerate}
        \end{itemize}
    \item \textbf{\citetitle{ancajas14fortnocs}} \checkmark
        \begin{itemize}
            \item MPSoCs with 3rd party IP NoCs (i.e. the interconnect system is 3rd party)
            \item Software accomplices (malicious/infected processing elements)
            \item Attack types: eavesdropping (information leak), voluntary data corruption, denial of service
            \item Fort-NoCs: 3-layer security mechanism (hardware level protection)
                \begin{itemize}
                    \item Lower layer data scrambling (hardware encryption to prevent covert activation sequences from AcTh to Trojan)
                    \item Middle layer packet certification (authentication tag, detect unintended destination after flit copy)
                    \item Top layer node obfuscation (migrate running applications from one node to another)
                \end{itemize}
            \item Malicious PE must secretly communicate with hardware trojan to send commands (C\&C node)
            \item Easy to run malicious software on a PE e.g. in cloud computing setups
            \item Small area and power overhead, mostly small runtime overhead
            \item Not all layers need to be used (in lower security domains)
        \end{itemize}
    \item \textbf{\citetitle{frey15stateobfuscation}}
        \begin{itemize}
            \item Attacker model: HT is the FSM control unit of NIs (very specific HT location)
            \item Countermeasure: obfuscate the states and state transistions that the FSMs do
            \item HT modifying state transitions causes FSM to enter illegal/invalid state → HT warning
            \item High HT detection rate (for this specific type of HT)
        \end{itemize}
    \item \textbf{\citetitle{frey17hardenednoc}} \checkmark
        \begin{itemize}
            \item Published two years after state obfuscation paper above
            \item Router level hardware trojans (HTs)
            \item Focuses on DoS attacks (bandwidth depletion) originating in a router (not a NI because router has more connections → more
                feasible)
            \item Implement DoS mitigation directly in the routers, rather than NI, to prevent bandwidth depletion as quickly as possible
            \item Physically Unclonable Function (PUF): random vector generation in each router
            \item Apply random dynamic permutation (data scrambling) to flits arriving at a router input (makes modifying flits into something
                meaningful significantly harder) before flit reaches the input queue (where the HT has access); de-permutate at output port (→
                PUF random vectors)
            \item Apply ECC (error control code) encoding before input port; decode before output port (only critical flit bits: header, tail,
                dest. address)
            \item Check flit integrity after leaving input queue and right before departing through the computed output port
            \item Cites lots of useful other related work
        \end{itemize}
    \item \textbf{\citetitle{fernandes16nocrouting}} \checkmark
        \begin{itemize}
            \item "Attacks at MPSoC aim to extract sensitive data, modify the system behavior or denial the system operation (Denial-of-Service,
                DoS)"
            \item Build security zones in the NoC using routing algorithm ("wrap IPs and protect sensitive information from attacker")
            \item Firewalls also possible, but may be costly (→ they implement a security policy in the NI)
            \item Aims to protect against software-based attacks (NoC is assumed to be secure)
            \item Threat model: timing and DoS attacks
            \item Security zone is e.g. the set of IP blocks that an application was mapped on
            \item Routing algorithm tries to keep the sensitive path completely inside the same security zone, if possible
        \end{itemize}
    \item \textbf{\citetitle{boraten16packetsecurity}} \checkmark
        \begin{itemize}
            \item Packet-Security (P-Sec)
            \item Threat model: compromised NoC does fault injection (side channel attack)
            \item It is possible to eventually obtain encryption keys by observing how encoders and decoders react to the side channel attacks
            \item → ensure integrity of packets using error correction codes (ECCs) (→ AMD, CRC)
            \item AMD for sensitive communications (together with encryption), otherwise CRC to provide minimal fault tolerance
        \end{itemize}
    \item \textbf{\citetitle{boraten18mitigationdos}} \checkmark
        \begin{itemize}
            \item Published 2 year after Packet Security paper above (builds upon previous research)
            \item HT does DoS attack: inspect packets, inject fault, trigger ECC response (ECC cannot correct error) → repeated transmissions,
                deadlocks
            \item HT resides in links between nodes
            \item Prevention: Heuristic fault classification → discover HTs
            \item Continue using compromised links instead of rerouting → obfuscation to prevent HT triggering, optimized AMDs to detect fault
                injections
            \item Little overhead: 2\% area, 6\% power
            \item "[...] we can classify security threats for NoCs as a subset of preexisting challenges originating from but not limited to,
                on-chip fault tolerance, functional correctness, path diversity, isolation, and quality of service"
            \item Security measures should not be compromised themselves
        \end{itemize}
    \item \textbf{\citetitle{biswas15routerattack}}
        \begin{itemize}
            \item Survey of MPSoC attack types
            \item New attack type for routing table-based routers (i.e. reconfigurable routers as opposed to routers with fixed routing logic)
            \item Mentions survey of hardware trojan detection techniques
            \item Not about detecting HTs, but about protection from malicious users
            \item → TEEs and REEs (Trusted/Rich Execution Environments), similar to security zones
            \item It is desirable to use routing tables instead of fixed routing logic (flexibility, more complex routing algorithms)
            \item Attack scenario: routing table is loaded onto NoC at boot or runtime (by host processor or NoC controller), which is modified by
                the attacker → unauthorized access and misrouting (routing to other environment)
        \end{itemize}
    \item \textbf{\citetitle{gebotys03securityframework}} \checkmark
        \begin{itemize}
            \item Framework: protection both at network and application layer
            \item Network layer
                \begin{itemize}
                    \item Key-keeper core: protects/distributes encryption keys to other secure cores
                    \item Each secure core has a security wrapper
                    \item Focus on key distribution and key management
                \end{itemize}
            \item Application layer
                \begin{itemize}
                    \item Software modifications for resistance against power (side-channel) attacks
                \end{itemize}
            \item Higher level approach than most other papers (more protocol layer than hardware layer)
            \item Strong assumptions on trusted software and hardware
            \item No clear attacker model, paper seems more like a "framework suggestion"
        \end{itemize}
    \item \textbf{\citetitle{kapoor13nocauthenc}} \checkmark
        \begin{itemize}
            \item 2 NoC zones: secure and non-secure IP cores
            \item Authenticated Encryption implemented in NIs of secure cores
            \item Secure cores can communicate with each other using permanent keys
            \item Non-secure cores can communicate with each other using plain text
            \item Hardware (NIs + routers) are assumed to be secure
            \item Secure and non-secure cores communicate with session keys and an intermediate link IP core (link can be secure or non-secure)
            \item Memory IP cores have access rights table in NI to prevent unauthorized memory accesses
            \item DoS attacks prevented by having a max number of packets allowed to be sent implemented in NI
        \end{itemize}
    \item \textbf{\citetitle{evain05nocsecurityanalysis}}
        \begin{itemize}
            \item In their context: CCM (central configuration module) is added (unique IP block → initialize and (re)configure NoC). Also CCM:
                add supervising and defending reactions for security
            \item FPGA vs. ASIC: reconfigurability of FPGA is another potential attack vector
            \item Mixed FPGA/ASIC implementation possible: ASIC for secure zone, FPGA for insecure zone (CCM must be in secure zone)
            \item Many possible attack types → different protection strategies
                \begin{itemize}
                    \item Bandwidth denial: virtual channels in the secure area (unsecure packets can't obstruct secure packets)
                    \item Unauthorized access: packet/path filters at zone boundaries and/or at NIs
                    \item Only encrypted/authenticated communication with the CCM
                \end{itemize}
        \end{itemize}
    \item \textbf{\citetitle{stefan11enhancingnocs}}
        \begin{itemize}
            \item Introduce non-determinism through multipath routing
            \item Proposal is implemented on top of Aethereal framework
            \item Time-division multiplexing (TDM) for router channels
            \item Alternative path selection
                \begin{itemize}
                    \item … based on position in the slot table at the moment of sending (static schedule)
                    \item … based on hardware RNG (dynamic at runtime)
                \end{itemize}
        \end{itemize}
\end{itemize}

Different attacker/threat models in literature. Depending on the attacker model, different approaches are used to protect the system against it.
E.g. when the underlying network architecture (the NoC itself) is assumed to be compromised, protection is implemented in the network interfaces
of the nodes. If the attacker only has access to specific parts of the routers or specific zones of the NoC, protection can be implemented through
the routing algorithm. → The power of the HTs differs. The more complex the HT is, the stronger it influences chip area/power consumption/runtime
overhead and may be easier detectable → that's why HTs are often assumed to use "small" attacks like fault injection, or have access to only very
specific components of the NoC to stay undetected/not require much chip area.

Differentiate between methods to detect HTs (on software level, firmware level, w/ static analysis, side-channel analysis), and methods to harden
the NoC against potential HT infections.

NI is usually assumed to be trusted, and routers are potentially compromised because of 3rd party IP or 3rd party manufacturing/integration
partners. Other threat model: software attacks (NoC itself is secure).

How to get HT into hardware: rogue employee, 3rd party IP, 3rd party manufacturing/integration partners, ...

This thesis: no hardware synthesis, use software simulation. Focus on malicious flit modification (→ attacker model) rather than DoS attacks. Our
model assumes that the NoC routers and links may be compromised and thus relies on the NIs for the security measures → no effective protection
against bandwidth depletion, but this is not the goal.
Deterministic vs. static routing? No security zones or division into secure/non-secure zones/cores.

The concept of security zones can be implemented in different ways. Bla et al. propose to do X, while bla enforce them through the routing protocol.



    \chapter{Overview}\label{ch:overview}
    % 1. novel approach, why? protection goals and performance → security and efficiency
% 2. what precisely was done at the TUD chair → say that their research proved the potential of NC and that's why we persue it here again
% 3. NC provides no confidentiality/integrity guarantees → we implement encryption+authentication
% 3.1 Availability only partially adressed
% 3.2 recall NoC architecture → this is implemented in the network interfaces
% 4. Different variants envisioned (3 methods) + comparison with uncoded variant where applicable
% 5. Multiple paths focus → why multiple paths in the first place? → chance to avoid compromised routers, still have enough flits
%    thanks to NC
% 6. Routing strategies: deterministic vs. non-deterministic routing → XY, smart random XY, ROMM+XY, ROMM+srXY
% 7. Attacker/Threat model
% 8. Evaluation
% Always refer to the chapters that explain this in detail
In this thesis, a novel approach for securing the communications in a \gls{noc}-based \gls{mpsoc} is pursued. The goal is to design and implement a protocol that
remedies both accidental and malicious modifications to the transmitted data as much as possible while meeting the performance requirements of a \gls{noc} (cf. Section
\ref{sec:networkonchipfun}). Furthermore, confidentiality shall be assured to prevent attackers from accessing the transmitted information. In this
thesis, a scheme is envisioned and investigated that attempts to satisfy these ambitions by combining encryption and authentication techniques with
network coding and multipath routing.

This thesis follows up on previous research performed at the TU Dresden. In 2015, the effect of network coding on communications in a partially
compromised \gls{noc} was evaluated and discussed \cite{moriam15manycorenc}. In 2018, this approach was combined with authentication
\cite{moriam18activeattackers}. Now, the emphasis lies on fusing network coding and authentication with encryption to fulfill the desired protection
goals (see Section \ref{sec:protectiongoals}).

In the next section, the necessity of such security measures is corroborated. Subsequently, the attacker model that the schemes explored in this
thesis aim to defend against is illustrated. Afterwards, an overview of the utilized techniques and how they are
integrated into a \gls{noc} architecture is given. Furthermore, different variants of the envisioned protocol are introduced. Finally, the
methodology for evaluating the design through in-depth simulations is presented.

\section{Necessity Of Security Measures}\label{sec:necessityofsecurity}
In Section \ref{sec:nocsecurity}, it was shown that attacks specifically tailored for \gls{noc} architectures are feasible and practical. In
particular, hardware trojans pose a potent threat to \glspl{mpsoc} that incorporate \glspl{noc} as their communication backbone. Pure software attacks
originating from a processing element are feasible as well \cites(e.g.)(){biswas15routerattack}{kocher04embeddedsecurity}, but will not be considered
during this thesis. % TODO: schöner formulieren?

There are a number of possible infection vectors that adversaries may try to exploit in order to covertly introduce a hardware trojan into a \gls{noc}
implemented in an \gls{mpsoc}\footnote{Theoretically, the aforementioned infection vectors are not specific to \glspl{mpsoc} and \glspl{noc}, but
this is the relevant constellation for the scope of this thesis.}. One such vector is the integration of third party \gls{ip}\footnote{An example of
this is the Arteris FlexNoC interconnect, which in \citeyear{ancajas14fortnocs} was used by \enquote{four out of the top five Chinese fabless
semiconductor \gls{oem} companies}
\cite[2]{ancajas14fortnocs}.}, which has become increasingly popular due to cost efficiency and growing circuit complexity
\cites[1]{ancajas14fortnocs}[2]{bhunia14hardwaretrojans}. These third parties may have an interest in equipping their \gls{ip} with a hardware trojan,
and their trustworthiness is usually not guaranteed \cite[3]{sethumadhavan15trustworthyhardware}.
Another practical scenario is a rogue employee \enquote{subvert[ing] the design} \cite[3]{sethumadhavan15trustworthyhardware} of his
company's product. For example, a hardware designer \enquote{participating in the design process} \cite[3]{sethumadhavan15trustworthyhardware} may
secretly introduce a hardware trojan at one point. 

The illustrated scenarios are not an exhaustive list of infection vectors, but they clearly corroborate the need for countermeasures. The ones pursued
for this thesis will be presented in the following sections.

\section{Attacker Model}
For the experiments in this thesis, the threat of a compromised \gls{noc} is explored. More specifically, the routers may be infected by a hardware
trojan, while the network interfaces are considered trustworthy. This is the same attacker model as the one used by \citeauthor{moriam18activeattackers}
\cite{moriam18activeattackers} in the work preceding this thesis. It is based on the assumption that routers and their interconnections are
more likely to be obtained from third party vendors than network interfaces, making them more susceptible to concealed hardware trojans. The reasoning
for this is that \enquote{routers have a deterministic functionality} \cite[2]{moriam18activeattackers} that does not depend on the peculiarities of
a specific system. In contrast, network interfaces often contain product-specific logic and are thus \enquote{rather developed in house and under
control} \cite[2]{moriam18activeattackers}, eliminating the attack vector of third party \gls{ip}. This model is visualized in Figure
\vref{fig:noctrustboundaries}.

\begin{figure}
    \centering
    \includegraphics[width=0.6\textwidth]{noc-trust-boundaries}
    \caption[Trust boundaries in a NoC]{Visualization of the trusted and untrusted hardware components in a NoC. The processing elements and network
    interfaces, which are assumed to be trusted, are marked green. The network itself, which is comprised of the routers and their interconnections,
    is not trustworthy and thus marked red. The dotted lines at the local connections mark the trust boundaries.}
    \label{fig:noctrustboundaries}
\end{figure}

\section{Encryption And Authentication}\label{sec:encandauth}
The intent of integrating encryption and authentication is to provide confidentiality and integrity to messages passing through a \gls{noc}. To
implement this, a novel network interface design is proposed. Since all flits that enter the \gls{noc} must pass through a network interface, this is the
ideal location to implement cryptographic protections. In the proposed design, it encrypts all outgoing flits, which can only be decrypted by the
receiver. In addition, a \gls{mac} is computed that is sent together with the encrypted data. On the receiver side, the flits are decrypted and the
\gls{mac} is verified. This scheme is illustrated in Figure \vref{fig:nocflitencauth} and further explained in Section (insert ref here).

\begin{figure}
    \centering
    \includegraphics[width=0.9\textwidth]{noc-message-enc-auth}
    \caption[Flit in the NoC with encryption and authentication]{Example of a flit being transmitted through a NoC. After a processing element sends a flit (1),
    encryption is applied (2) and a MAC is computed (3) in the sender's network interface. The encrypted flit and the MAC are then routed to the
    destination (4). There, the receiver's network interface verifies the MAC (5) and decrypts the flit (6). Finally, the flit is passed to the
    receiving processing element (7). This scheme corresponds to the uncoded individual authentication protocol variant.}
    \label{fig:nocflitencauth}
\end{figure}

The constraints imposed by the \gls{noc} environment (cf. Section \ref{sec:networkonchipfun}) make the implementation of security measures a
challenging task. To meet them, only symmetric ciphers were investigated, as they allow for fast computations and low-area implementations.
Furthermore, the produced \glspl{mac} are short enough to fit into a single flit. However, their usage implies that each pair of
sender and receiver needs to possess two shared secret keys (one for encryption and one for authentication). To obtain or renegotiate
such keys in a secure manner, a key exchange algorithm is required
\footnote{One way to realize a secure key exchange in the context of \glspl{noc} is the utilization of central key-keeper cores
\cite{gebotys03securityframework} (see also Section \ref{subsec:securityzones}). Another option is physical layer key generation, as suggested in the
\gls{haec} presentation paper \cite[4]{matthiesen17haec}.}.
As this exceeds the scope of this thesis, each pair of sender and receiver is assumed to have access to the necessary keys. The alternative, asymmetric
cryptography, is not considered since it \enquote{implies a high computational effort} \cite[3]{moriam18activeattackers}. Additionally, this class of
algorithms produces signatures instead of \glspl{mac}, which are \enquote{too long to be included in a flit} \cite[3]{moriam18activeattackers}.

\section{Network Coding}\label{sec:networkcodingover}
A promising approach to improve the performance of \glspl{noc} is network coding. \citeauthor{moriam15manycorenc} have shown that it is particularly
effective in error-prone networks, decreasing latency up to 95\% \cite[7]{moriam15manycorenc}. In the context of this thesis, \glspl{noc} are assumed
to be unreliable: compromised routers may deliberately inject faults or drop flits. Hence, this approach is taken up here to improve the network
performance.

While network coding provides robustness against sporadic flit loss, it does not offer any guarantees on the integrity of flits. Thus, modifications
during the transmission of the coded data are not detected directly, potentially leading to faulty decodings. This deficit is remedied by the cryptographic
layer described above. To evaluate the consequences of the integration of network coding, uncoded versions of the protocol were implemented and
examined as well (see Section \ref{sec:protocolvariants}).

\section{Multipath Routing}
The exploration of different routing strategies is a central aspect of this thesis. Both static and dynamic strategies were
examined and evaluated. With a static strategy, there is a predetermined, time-invariant path that a flit will take to reach a certain destination
from a particular sender. In contrast, dynamic strategies implement random factors that influence the routing decisions for each
transmission. The emphasis of this thesis lies on the latter in order to capture the envisioned benefits of multipath routing.

As a deterministic strategy, dimension order routing is used. On the dynamic side, \gls{dm} and \gls{romm} routing are explored. The properties and
details of these strategies are presented in depth in Section (insert ref here).% TODO: rewrite this a bit, focus on dynamic

\section{Reliability}\label{sec:reliability}
In case the proposed techniques fail to achieve flawless transmission of a flit or generation, it is crucial to provide a method for requesting
their retransmission. This occurs when the integrity check via \gls{mac} verification fails, indicating a modification in one or more of the
associated flits. Furthermore, it is necessary when not enough flits arrive at the destination in time to perform the verifications in the first place.
Additionally, the network coded variants require at least partially intact transmissions in order to successfully decode a generation, necessitating
retransmissions otherwise.

A retransmission scheme is integrated into the protocol by means of \glspl{arq}. If one or more of the aforementioned events occur, the affected
receiver will issue an \gls{arq} back to the sender, who will then resend the flits in question.

To render retransmissions possible, a copy of each flit that is sent through the network must be kept by the sender in order to answer a potential
\gls{arq} arriving later on. This is facilitated by a retransmission buffer that is included in every network interface. Each flit that is sent to the
network (except for the \glspl{arq} themselves) must pass through this buffer, where a copy of it is stored. When an \gls{arq} arrives from another
node, the requested flits are retrieved from the buffer and resent. As the retransmission buffer cannot grow infinitely in size, old flits are
replaced in a \gls{fifo} manner once maximum size is reached.

\section{Protocol Variants}\label{sec:protocolvariants}
To unearth and analyze the most efficient way of applying the ideas outlined above, different variants of the protocol were implemented. They differ
in the way the authentication \glspl{mac} are included in the flit transmissions. Furthermore, both network coded and uncoded versions are compared to
examine the effects of network coding in combination with the cryptographic procedures. The emphasis of this thesis, however, is put on the network
coded forms.

The examined variants are \textit{individual authentication}, \textit{interwoven authentication}, and \textit{full-generation authentication}. For the
first two, both uncoded and network coded variants are implemented. For the latter, only a network coded version is feasible: the authentication
scheme relies on the existance of generations. A comprehensive description of all variants is given in Section
\ref{sec:theprotocol}. % TODO: wiederholung mit "both uncoded and network coded versions/variants"

\section{Analysis And Evaluation}
The aforementioned protocol variants were thoroughly analyzed to empirically determine their quality and suitability for the task at hand. Many
experiments were conducted through software simulation of an \gls{mpsoc} with a \gls{noc} using varying parameters, component layouts, and design
decisions.

The experiments were performed with a simulator specifically crafted for this thesis. Based on the free and open source framework \omnet{} \cite{omnet}, it allows
for cycle-accurate simulations with fully customizable statistics recording. Furthermore, its modular design allows for quickly swapping and
reordering components, which is required to test the different protocol variants. The details of the implementation are described in Chapter
\ref{ch:implementation}.

The workflow of the evaluation was as follows: at first, the hyperparameters were determined once, with the resulting values being used in all
subsequent experiments. These hyperparameters include, e.g., the number of required parallel crypto modules\footnote{The term \textit{crypto modules} is
used to refer to both encryption and authentication modules.} per network interface and the \gls{arq} timeouts. Afterwards, the remaining parameters are
varied to find the optimal configuration for each protocol variant. Finally, the most promising of these variants is identified. The full evaluation
process is elaborated in Chapter \ref{ch:evaluation}.


    \chapter{Communication Protocol}\label{ch:protocol}
    \begin{itemize}
    \item Uncoded transmission
        \begin{itemize}
            \item no network coding
            \item 2 methods: 1 data flit + 1 MAC flit OR 2 data/MAC split flits
        \end{itemize}
    \item Flit structure
        \begin{itemize}
            \item burst bit, source/target address, mode, address, GID/FID, GEV, payload
            \item mode: define if data/mac/split/arq
        \end{itemize}
    \item Network coded transmission
        \begin{itemize}
            \item Number of flits: G2C3 or G2C4
            \item 3 methods: 1 data flit + 1 MAC flit OR 1 MAC flit per generation OR 2 data/MAC split flits
        \end{itemize}
    \item ARQs
        \begin{itemize}
            \item Limited number of ARQs per transmission unit (UC: data/MAC pair or split pair, NC: generation)
            \item Timeout of x (e.g. 8) cycles until first ARQ is sent
            \item If limit >1: start larger timeout (→ round-trip of ARQ)
            \item Many different cases, insert some flow diagrams here
            \item The higher the ARQ timeout/limit, the less likely the flit is still in retransmission buffer
            \item → ARQ timeout/limit and retransmission buffer size have to correlate
            \item In the case that we only have 1 ARQ left that we are allowed to send: Wait for any ongoing MAC verifications
                so in case they fail, the flits can be included in the ARQ
        \end{itemize}
\end{itemize}


    \chapter{Implementation}\label{ch:implementation}
    Following the design phase, it is crucial to test the protocol in a practical environment in order to verify that it works as intended. Furthermore,
extensive performance evaluations need to be conducted to assure the viability of the devised techniques. Both of these tasks are achieved through
thorough software simulation. For this purpose, a dedicated, cycle-accurate simulator was implemented that supports the designed protocol with all the
features described in Chapter \ref{ch:protocol}.

The simulator is written in C++ and based on the \textit{\omnet{}} discrete event simulation framework \cite{omnet}. \omnet{} provides a general-purpose
network simulation architecture and promotes a modular design. In general, it allows the programmer to define the components of the network via a
special scripting language called \textit{NED}. With NED files, modules may be specified as they appear from the outside, by means of parameters,
gates, and connections. Furthermore, a module is allowed to have any number of submodules, facilitating a hierarchical structure. A C++
class is assigned to each module that controls the component's internal behavior together with the parameters. The gates define the ports of the
module through which it sends and receives messages. The connections link the gates of different modules (or submodules) together and thus define the
message flows and the network's topology.

The simulator makes extensive use of the hierarchical and modular approach of \omnet{}. The processing elements, network interfaces, and routers are
defined as modules and instanced as many times as required, depending on the \gls{noc} dimensions. They are interconnected through their gates and
arranged in a way mirroring the structure of the \gls{noc}. To create a simulation as realistic as possible, they are made up of a number of
submodules, such as queues, buffers, and crypto modules, resembling the actual hardware layout.

In Section \ref{sec:generalass}, fundamental assumptions about the simulation environment are presented. Subsequently, Section
\ref{sec:componentstructure} elaborates on the hierarchical structure of the components and their layout. Finally, an overview of the various
configurable parameters of the simulator is given in Section \ref{sec:confparams}.

\section{General Assumptions}\label{sec:generalass}
\subsection{The GALS Paradigm}
In Section \ref{sec:networkonchipfun}, it was mentioned that \glspl{noc} lend themselves well to the implementation of the \gls{gals} design paradigm.
For the simulation, it was not considered for three reasons. First, its usage requires the existance of multiple clock domains: each core runs with the
frequency best suited for itself and is not synchronized with other cores in the network (\enquote{globally asynchronous}). This, however, is very
finical to implement correctly. Second, since the cores themselves each operate under a single clock (\enquote{locally synchronous}), \gls{gals} would
only affect the transmissions from one router to another. Third, the cores are assumed to be identical and hence run with the same frequency anyway.
On these grounds, a globally synchronous architecture with a single clock driving all components is used in the simulator.

\subsection{Clock Frequency}
The \gls{noc}'s global clock runs at a frequency of 500 MHz as this is a common speed in \gls{noc}-related research
\cites{frey15stateobfuscation}{frey17hardenednoc}. Unfortunately, the PRINCE cipher that is employed for all cryptographic operations is unable to run at
such a high frequency. However, in Section \ref{subsubsec:prince} it was shown that the algorithm runs flawlessly at frequencies of up to 250 MHz.
Thus, the crypto modules operate can operate at half the frequency of the \gls{noc}, with every two cycles of the global clock corresponding to one
cycle in the crypto modules. Hence, the processing of a single block is assumed to take two clock cycles in the simulator.

It was mentioned above that the integration of multiple clock domains is very finical. However, when they are not independent and the global clock's
frequency is evenly divisible by the required frequency, as is the case for the crypto modules, this becomes significantly easier. Here, a new clock
is derived from the global clock that simply omits every other tick.

\subsection{Single Cycle Routing}
only input queues instead of in- and output queues

Pros: 1 instead of 2 cycles per hop, less buffers
Cons: if flit has to wait because target port is congested, then it blocks all further flits in the same queue
→ we like low latencies and low chip area, thus single cycle routing

\subsection{ARQ Timeouts}
- Fixed timeout for initial timer, after ARQ: initial timer + 2x distance

\section{Component Structure}\label{sec:componentstructure}
% screenshot of the Qtenv NoC
\subsection{Network Interface}
% 1 paragraph per component
% at the end: table with all components, their delay (cycles)
Network interface was modeled as accurately as possible to be able to evaluate internal congestions like competition of multiple flits over the
available crypto modules.

Figure X shows the internal structure of the network interface as rendered by \omnet{}'s graphical interface.

\subsubsection{Crypto Modules}
- Access Priorities (1. sender in order to not tear apart a generation and provoke timeouts at the receiver, 2. receiver)

Encryption/Decryption:
1 input block = 2 cycles for all methods

Authentication:
Ind. Auth 3 input blocks → 6 cycles
Int. Auth 2 input blocks → 4 cycles + 1 cycle for authcode generation → 5 cycles (2x as many authcodes generated as for ind. auth)
Gen. Auth 5 input blocks → 10 cycles (half as many MACs generated as for ind. auth)

\begin{table}
    \centering
    \begin{tabulary}{\textwidth}{R|L}
        Component & Delay in cycles \\\hline
        Encryption/decryption & 2 \\
        Encoding (G2C3) & 3 (1 per combination) \\
        Encoding (G2C4) & 4 (1 per combination) \\
        Authentication (individual) & 6 \\
        Authentication (interwoven) & 5 \\
        Authentication (full gen.) & 10 \\
        Storing flit in \gls{rtb} & 1 \\
        Decoding & 2 (1 per flit) \\
        Comparing \glspl{mac} & 1 \\
        Composing an \gls{arq} & 1 \\
        \Gls{rtb} lookup & 2 (uncoded), 3 (coded)
    \end{tabulary}
    \caption[short]{long}
    \label{tab:processinglatencies}
\end{table}

\subsection{Routers}
% Only input buffers → single cycle routing
% Routing is only performed when receiving router's input queue is not full

\begin{table}
    \centering
    \begin{tabulary}{\textwidth}{R|L}
        Transmission & Delay in cycles \\\hline
        PE ←→ NI & 1 \\
        NI ←→ Router & 1 \\
        Router ←→ Router & 1
    \end{tabulary}
    \caption[short]{long}
    \label{tab:transmissionlatencies}
\end{table}

\section{Configurable Parameters}\label{sec:confparams}

\iffalse
\begin{itemize}
    \item Retransmission Buffer structure and lookup times
        \begin{itemize}
            \item corresponding flits are stored consecutively (e.g. data/MAC of same FID, flits of same generation etc.)
            \item lookup time (in clock cycles) is a parameter in the simulation
            \item UC case: one cycle lookup is fine (just need to find FID, mode field determines offset in the buffer)
            \item NC case: two cycles for lookup (one to find GID, one to compare GEVs of the generation in parallel, mode determines offset)
        \end{itemize}
    \item Priorities
        \begin{itemize}
            \item retransmission buffer: ARQs have priority
            \item crypto units (→ entry guard): arriving flits have priority
        \end{itemize}
    \item Buffers/Queues
        \begin{itemize}
            \item App/NI/Routers have only input buffers, no output buffers
            \item Routers only route flits when the receiving router's input queue is not full
        \end{itemize}
    \item Crypto units
        \begin{itemize}
            \item Separate units for encryption and authentication
            \item Encryption units can also decrypt → very easy to see with PRINCE
            \item Send/receive pipeline share the same set of crypto units
            \item Talk about auth. method 3: 32 bit block size, what algorithms?
            \item Latency: assume PRINCE → ~35MHz FPGA, *~4 for ASIC → paper. Look at PRINCE paper and survey paper for numbers
        \end{itemize}
    \item Tracking finished IDs: prevent repeat attacks, prevent re-processing a unit due to redundant retransmissions or repeat attacks, but still
        allow out-of-order arrivals SeemsGood
    \item Routing strategies: randomness needs hardware RNG → more complex logic, more area!

    \item Timeout of x (e.g. 8) cycles until first ARQ is sent
    \item The higher the ARQ timeout/limit, the less likely the flit is still in retransmission buffer
    \item → ARQ timeout/limit and retransmission buffer size have to correlate
\end{itemize}
\fi


    \chapter{Experiments}\label{ch:experiments}
    \section{Parameters}
In the experiments, a base network injection rate of 0.2 is assumed\footnote{The actual injection rate can be higher once \glspl{arq} come into play
and retransmissions are required.}
\begin{itemize}
    \item Injection rate
        \begin{itemize}
            \item Value if 0.2 is realistic
            \item Possibility to generate pairs for fair comparison of UC/NC
        \end{itemize}
\end{itemize}

\section{Attacker Model}
\begin{itemize}
    \item Variable number of compromised routers (e.g. 8 for an 8x8 grid)
    \item Compromised routers randomly drop or modify packets (no intelligent modifications/drops)
    \item Reasoning for having only some compromised routers (in regard to the 3rd party NoC problem → why would they not make all routers the same?)
\end{itemize}

\section{Environment}
% 50,000 cycles (why?)
% Warmup and cooldown times

\section{Determining The Hyperparameters}
\subsection{ARQ Timeouts}
\subsection{Number Of Crypto Units}

\section{Experiments}
\subsection{Average time of a flit/transmission unit in ArrivalManager until a verification result is there}


    \chapter{Evaluation}\label{ch:evaluation}
    After the protocol design phase (Chapter \ref{ch:protocol}) and the implementation with the simulator (Chapter \ref{ch:implementation}), the next step
is to evaluate the different approaches. As the primary goal of this thesis is to provide protection and resilience against malicious routers within
the \gls{noc}, the protocol variants were tested for varying attacker positions, attack probabilities, and routing strategies. The quality of a scheme
is determined through several metrics that indicate its performance for a given scenario, which are explained below.

The next section establishes notations and key terms that are used throughout this chapter. Afterwards, the environment and frame conditions for the
simulations are delineated (Section \ref{sec:environmenteval}). Following this, the employed attacker model is discussed (Section
\ref{sec:attackermodeleval}). Subsequently, the hyperparameters are fixed through a series of representative experiments (Section
\ref{sec:hyperparamseval}), followed by the main evaluation where the different protocol variants and routing strategies are compared (Section
\ref{sec:perfcompeval}).

\section{Notation}\label{sec:notationeval}
To improve readability, abbreviations are introduced for the protocol variants and network coding modes. This facilitates their representation within
charts, diagrams, and tables.
\begin{itemize}
    \item \textbf{\Gls{ida}} stands for the \textit{individual authentication} protocol version
    \item \textbf{\Gls{iwa}} represents the \textit{interwoven authentication} approach
    \item \textbf{\Gls{fga}} denotes the \textit{full-generation authentication} variant
    \item \textbf{UC} is short for \textit{uncoded}, i.e., the lack of network coding
    \item \textbf{G2C3} and \textbf{G2C4} are used for the two network coding variants that were explained in Section \ref{sec:designnc}.
\end{itemize}
\vspace{0.5\baselineskip}

These terms are often combined with a routing strategy when the setup of an experiment is described. For instance, \gls{ida}-UC-\gls{dor} stands for uncoded
individual authentication using dimension order routing.

%Terms "source flit", router "port queues", "local queues", "creation rate"

\section{Environment}\label{sec:environmenteval}
In this section, the general setup for all conducted simulations is elaborated, such as flit generation patterns and simulation runtime. Table
\vref{tab:fixedparams} provides a concise overview of these invariant parameters and their values.

\begin{table}
    \centering
    \begin{tabulary}{\textwidth}{L|L}
        Parameter name & Value \\\hline
        \Gls{noc} dimensions & 8x8 \\
        Clock frequency & 500 MHz \\
        Simulation runtime & \num{50000} cycles \\
        Warmup/cooldown time & 500 cycles \\
        Network base injection rate & 0.2 \\
        Flit destination selection & uniform random \\
        Pair generation & yes \\
    \end{tabulary}
    \caption[short]{long}
    \label{tab:fixedparams}
\end{table}

When a simulation is executed, the first 500 cycles are considered warmup time. During this period, no statistics are recorded. This ensures that the
network is in a steady state and already saturated with flits once the recording starts. It is followed by the main simulation of \num{50000}
cycles where statistics are recorded normally. Finally, there is another 500 cooldown cycles at the end. This was added to ensure that those flits
generated near the end of the main simulation are not classified as lost flits simply because they were not granted enough time to reach their
destination. Thus, flits arriving at processing elements during the cooldown period are considered successfully transmitted, while flits generated
during this period are not included in the statistics.

The network base injection rate describes the average number of flits injected into the network per clock cycle at each node, excluding \gls{arq}
flits and retransmissions. The creation rate of flits at the processing elements is adjusted accordingly for each protocol variant to keep the
injection rate constant. For instance, with \gls{ida}-G2C3, every flit created at a particular processing element results in three flits injected
into the network due to network coding and the additional \gls{mac} flits. Hence, the creation rate would be $\frac{0.2}{3}$ for this scenario.

The processing elements create flits independently from each other. All processing elements use the same creation rate. The selection of the flits'
destinations is performed randomly with a uniform probability distibution over all network nodes (excluding the sender's own node).

To allow for a fair comparison of the protocol versions, flits are always generated in pairs. This means that when a particular processing
element creates a flit, another one with the same destination is guaranteed to be created on the next clock cycle. This prevents long periods of
buffering for network coded \gls{ida} and \gls{fga} in the sender's network interface: with a pair of flits, a generation can be formed immediately. This creation
pattern is a realistic assumption as in practical applications, flits are usually generated from breaking a data packet down into smaller parts. These
flits would then enter the network consecutively and have the same destination. The creation rate of flits is adjusted accordingly to reflect this
pattern: for example, with \gls{ida}-G2C3, the probability to create a pair would be set to $\frac{0.2}{6}$ to keep the base network injection rate at
0.2.

Some of the parameters were adopted from the simulation setup that \citeauthor{moriam18activeattackers} \cite{moriam18activeattackers} have employed
in their experiments to render results comparable with their evaluations. These parameters include the \gls{noc} dimensions, the base network
injection rate, and the simulation runtime. The rationale for a clock speed of 500 MHz was outlined in Section \ref{subsec:clockfrequency}.

In the experiments, a base network injection rate of 0.2 is assumed. This is the same value that \citeauthor{moriam18activeattackers} have chosen
\cite[2]{moriam18activeattackers} in order for the results to be comparable with their analyses. The actual injection rate may be higher as the base rate does not include the
issuance of \glspl{arq} and the resulting retransmissions.

\begin{table}
    \centering
    \begin{tabulary}{\textwidth}{C|C|C}
        Protocol & Ratio & Creation rate \\\hline
        \gls{ida}-UC & 1:2 & \\
    \end{tabulary}
    \caption[short]{long}
    \label{tab:creationrates}
\end{table}

\begin{itemize}
    \item Injection rate
        \begin{itemize}
            \item Value if 0.2 is realistic
            \item Possibility to generate pairs for fair comparison of UC/NC
            \item Base network injection rate of 0.2 is used for all experiments. The source flit creation rate is adjusted accordingly for the
                protocol variants to ensure that they have the same injection rate.
        \end{itemize}
    \item Overhead von Verschlüsselung+Auth vs. nur Auth vs keins von beiden (sowohl Latenz als auch Chipfläche)
\end{itemize}

The simulator has a variety of parameters that can be altered to influence its behavior and consequently the outcome of the simulations.


\begin{table}
    \centering
    \begin{tabulary}{\textwidth}{C|C|C}
        Parameter name & Value range & Placeholder \\\hline
        Network coding & $\{\mathit{UC}, \mathit{G2C3}, \mathit{G2C4}\}$ & \pNCMode{} \\
        No. of encryption modules & $\mathbb{N}^*$ & \pEncMods{} \\
        No. of authentication modules & $\mathbb{N}^*$ & \pAuthMods{} \\
        \Gls{arq} limit & $[1, 2]$ & \pARQLimit{} \\
        \Gls{arq} timeout & $\mathbb{N}^*$ cycles & \pARQTimeout{} \\
        Routing strategy & $\{\mathit{\gls{dor}}, \mathit{\gls{dm}}, \mathit{\gls{romm}}, \mathit{\gls{ramm}}\}$ & \pRStrat{} \\
    \end{tabulary}
    \caption[short]{long}
    \label{tab:inputparams}
\end{table}% TODO: finish table

\section{Attacker Model}\label{sec:attackermodeleval}
\begin{itemize}
    \item Focuses on malicious modifications rather than DoS attacks
    \item Assumption: compromised routers → rely on NIs for security
    \item No protection against bandwidth depletion, but this is not the goal here
    \item Variable number of compromised routers (e.g. 8 for an 8x8 grid)
    \item Compromised routers randomly drop or modify packets (no intelligent modifications/drops)
    \item Reasoning for having only some compromised routers (in regard to the 3rd party NoC problem → why would they not make all routers the same?)
        → HT implies more logic in routers → more area and draws more power → might attract attention if all routers have this and overall NoC
        parameters diverge considerably from the expectations
\end{itemize}

Compromised routers: 8 randomly selected routers (uniform distribution), 3 different sets, average over them

\begin{figure}
    \includegraphics[width=0.3\textwidth]{attacker-positions-1}\hfill
    \includegraphics[width=0.3\textwidth]{attacker-positions-2}\hfill
    \includegraphics[width=0.3\textwidth]{attacker-positions-3}
    \caption[Malicious router distributions]{The three distributions of 8 malicious routers over the 8x8 \gls{noc} that were used in experiments.}
    \label{fig:attackerpositions}
\end{figure}

\section{Determining The Hyperparameters}\label{sec:hyperparamseval}
\subsection{ARQ Timeouts}\label{subsec:arqtimeouts}
In Section \ref{subsec:arqretransmissions}, the concept of \glspl{arq} and timeouts was introduced: receivers issue \glspl{arq} when the temporal gap
between the arrival of flits belonging to the same transmission unit becomes too large, i.e. when a timeout occurs. Its value is determined through
experiments and measured in clock cycles.

\begin{figure}
    \centering
    \includegraphics[width=0.9\textwidth]{arq-timeouts-calc}
    \caption[Example of ARQ timeout calculation]{Example for the calculation of \gls{arq} timeouts. With a Manhattan distance of 4 between source $S$ and
    destination $D$ and a given inter-arrival timeout $t_1$, the \gls{arq} timeout $t_2(S, D)$ is computed as $t_1 + 2 \cdot 4 = t_1 + 8$ cycles.}
    \label{fig:arqtimeoutscalc}
\end{figure}

In addition to the inter-arrival timeout, there is another, higher value that is used after an \gls{arq} was issued to await the answer, as explained
in Section \ref{subsec:arqretransmissions}. The \gls{arq} answer timeout depends on the inter-arrival timeout and the Manhattan distance between the
two affected communication partners. More precisely, if $t_1$ is the inter-arrival timeout, $t_2(S, D)$ is the \gls{arq} answer timeout for a
particular source $S$ and destination $D$, and $d$ is the Manhattan distance between the two nodes, then $t_2(S, D) = t_1 + 2 \cdot d$. Figure
\vref{fig:arqtimeoutscalc} illustrates this calculation.

Hence, only $t_1$ needs to be determined through experiments. Since this is the first parameter to be fixed, the other input values for the simulation
are estimated. The tests are independent of the protocol variant as the same injection rate is used for all of them. Table \vref{tab:setuparqtimeouts}
presents how the simulator is set up.

\begin{table}
    \centering
    \begin{tabulary}{\textwidth}{C|C|C|C|C|C|C|C|C|C}
        \pProtVar{} & \pNCMode{} & \pEncMods{} & \pAuthMods{} & \pRQSize{} & \pARQLimit{} & \pARQTimeout{} & \pRStrat{} & \pNumAttackers{} & \pAttackProb{} \\\hline
        \gls{ida}   & varying    & 5           & 15           & unlimited  & 1            & varying        & \gls{dor}  & 0                & 0 \\
    \end{tabulary}
    \caption[Input parameters for ARQ timeouts experiment]{long}
    \label{tab:setuparqtimeouts}
\end{table}
% Why Ind. Auth.? Most flits per transmission unit (8 with G2C4)
% Why this number of enc./auth. units? Large number so there are definitely no internal congestions → area doesn't matter for this experiment
% Why ARQs per source flit and not per transmission unit? because size of trans. units varies considerably with prot. variant

To determine the timeout value, an uncompromised \gls{noc} (i.e., with zero malicious routers) is used. Ideally, no \glspl{arq} are issued in this
scenario. However, due to random deviations from the average flit injection rate, the network load varies over time and congestions in some routers
may occur. The resulting increased transmission delays may be high enough to trigger timeouts even when an \gls{arq} is not necessary. Sporadic
occurances of such cases cannot be ruled out, but should rarely happen. However, simply increasing the timeout until these cases vanish is undesirable
for two reasons. First, a high timeout directly corresponds to high latencies when flit losses necessitate \glspl{arq}. Second, the later an \gls{arq}
is issued, the larger the \gls{rtb} of the communication partner needs to be so that the flits in question are not already overwritten when the
\gls{arq} arrives. The goal of this experiment is to find a reasonable middle ground: the smallest timeout that does not entail a significant number of
unnecessary \glspl{arq} will be used for the subsequent evaluations.

\begin{figure}
    \centering
    % GNUPLOT: LaTeX picture with Postscript
\begingroup
\newcommand{\ft}[0]{\footnotesize}\newcommand{\ty}[0]{\tiny}
  \makeatletter
  \providecommand\color[2][]{%
    \GenericError{(gnuplot) \space\space\space\@spaces}{%
      Package color not loaded in conjunction with
      terminal option `colourtext'%
    }{See the gnuplot documentation for explanation.%
    }{Either use 'blacktext' in gnuplot or load the package
      color.sty in LaTeX.}%
    \renewcommand\color[2][]{}%
  }%
  \providecommand\includegraphics[2][]{%
    \GenericError{(gnuplot) \space\space\space\@spaces}{%
      Package graphicx or graphics not loaded%
    }{See the gnuplot documentation for explanation.%
    }{The gnuplot epslatex terminal needs graphicx.sty or graphics.sty.}%
    \renewcommand\includegraphics[2][]{}%
  }%
  \providecommand\rotatebox[2]{#2}%
  \@ifundefined{ifGPcolor}{%
    \newif\ifGPcolor
    \GPcolortrue
  }{}%
  \@ifundefined{ifGPblacktext}{%
    \newif\ifGPblacktext
    \GPblacktextfalse
  }{}%
  % define a \g@addto@macro without @ in the name:
  \let\gplgaddtomacro\g@addto@macro
  % define empty templates for all commands taking text:
  \gdef\gplbacktext{}%
  \gdef\gplfronttext{}%
  \makeatother
  \ifGPblacktext
    % no textcolor at all
    \def\colorrgb#1{}%
    \def\colorgray#1{}%
  \else
    % gray or color?
    \ifGPcolor
      \def\colorrgb#1{\color[rgb]{#1}}%
      \def\colorgray#1{\color[gray]{#1}}%
      \expandafter\def\csname LTw\endcsname{\color{white}}%
      \expandafter\def\csname LTb\endcsname{\color{black}}%
      \expandafter\def\csname LTa\endcsname{\color{black}}%
      \expandafter\def\csname LT0\endcsname{\color[rgb]{1,0,0}}%
      \expandafter\def\csname LT1\endcsname{\color[rgb]{0,1,0}}%
      \expandafter\def\csname LT2\endcsname{\color[rgb]{0,0,1}}%
      \expandafter\def\csname LT3\endcsname{\color[rgb]{1,0,1}}%
      \expandafter\def\csname LT4\endcsname{\color[rgb]{0,1,1}}%
      \expandafter\def\csname LT5\endcsname{\color[rgb]{1,1,0}}%
      \expandafter\def\csname LT6\endcsname{\color[rgb]{0,0,0}}%
      \expandafter\def\csname LT7\endcsname{\color[rgb]{1,0.3,0}}%
      \expandafter\def\csname LT8\endcsname{\color[rgb]{0.5,0.5,0.5}}%
    \else
      % gray
      \def\colorrgb#1{\color{black}}%
      \def\colorgray#1{\color[gray]{#1}}%
      \expandafter\def\csname LTw\endcsname{\color{white}}%
      \expandafter\def\csname LTb\endcsname{\color{black}}%
      \expandafter\def\csname LTa\endcsname{\color{black}}%
      \expandafter\def\csname LT0\endcsname{\color{black}}%
      \expandafter\def\csname LT1\endcsname{\color{black}}%
      \expandafter\def\csname LT2\endcsname{\color{black}}%
      \expandafter\def\csname LT3\endcsname{\color{black}}%
      \expandafter\def\csname LT4\endcsname{\color{black}}%
      \expandafter\def\csname LT5\endcsname{\color{black}}%
      \expandafter\def\csname LT6\endcsname{\color{black}}%
      \expandafter\def\csname LT7\endcsname{\color{black}}%
      \expandafter\def\csname LT8\endcsname{\color{black}}%
    \fi
  \fi
    \setlength{\unitlength}{0.0500bp}%
    \ifx\gptboxheight\undefined%
      \newlength{\gptboxheight}%
      \newlength{\gptboxwidth}%
      \newsavebox{\gptboxtext}%
    \fi%
    \setlength{\fboxrule}{0.5pt}%
    \setlength{\fboxsep}{1pt}%
\begin{picture}(7200.00,3168.00)%
    \gplgaddtomacro\gplbacktext{%
      \csname LTb\endcsname%
      \put(660,660){\makebox(0,0)[r]{\strut{}\ft 0}}%
      \csname LTb\endcsname%
      \put(660,1109){\makebox(0,0)[r]{\strut{}\ft 0.05}}%
      \csname LTb\endcsname%
      \put(660,1557){\makebox(0,0)[r]{\strut{}\ft 0.1}}%
      \csname LTb\endcsname%
      \put(660,2006){\makebox(0,0)[r]{\strut{}\ft 0.15}}%
      \csname LTb\endcsname%
      \put(660,2454){\makebox(0,0)[r]{\strut{}\ft 0.2}}%
      \csname LTb\endcsname%
      \put(660,2903){\makebox(0,0)[r]{\strut{}\ft 0.25}}%
      \put(792,440){\makebox(0,0){\strut{}\ft 3}}%
      \put(1293,440){\makebox(0,0){\strut{}\ft 4}}%
      \put(1794,440){\makebox(0,0){\strut{}\ft 5}}%
      \put(2295,440){\makebox(0,0){\strut{}\ft 6}}%
      \put(2796,440){\makebox(0,0){\strut{}\ft 7}}%
      \put(3297,440){\makebox(0,0){\strut{}\ft 8}}%
      \put(3798,440){\makebox(0,0){\strut{}\ft 9}}%
      \put(4298,440){\makebox(0,0){\strut{}\ft 10}}%
      \put(4799,440){\makebox(0,0){\strut{}\ft 11}}%
      \put(5300,440){\makebox(0,0){\strut{}\ft 12}}%
      \put(5801,440){\makebox(0,0){\strut{}\ft 13}}%
      \put(6302,440){\makebox(0,0){\strut{}\ft 14}}%
      \put(6803,440){\makebox(0,0){\strut{}\ft 15}}%
    }%
    \gplgaddtomacro\gplfronttext{%
      \csname LTb\endcsname%
      \put(22,1781){\rotatebox{-270}{\makebox(0,0){\strut{}\ft ARQs per source flit}}}%
      \put(3797,154){\makebox(0,0){\strut{}\ft Timeout $t_1$ in cycles}}%
      \csname LTb\endcsname%
      \put(6068,2765){\makebox(0,0)[r]{\strut{}\ty IDA-UC}}%
      \csname LTb\endcsname%
      \put(6068,2615){\makebox(0,0)[r]{\strut{}\ty G2C3}}%
      \csname LTb\endcsname%
      \put(6068,2465){\makebox(0,0)[r]{\strut{}\ty G2C4}}%
    }%
    \gplbacktext
    \put(0,0){\includegraphics{../plots/arqtimeouts}}%
    \gplfronttext
  \end{picture}%
\endgroup

    \caption[Results for ARQ timeouts experiment]{long}
    \label{fig:resultsarqtimeouts}
\end{figure}

Figure \vref{fig:resultsarqtimeouts} shows the results for timeout values ranging from 3 to 15 cycles.

- Use sadias formula for RTB size estimation

We choose value of 12 because then for all NC variants, it is less than 1 ARQ per 100 source flits. Higher timeout would entail even less ARQs, but
increase latency and RTB sizes.

\subsection{Number Of Crypto Modules}
\begin{table}
    \centering
    \begin{tabulary}{\textwidth}{C|C|C|C|C|C|C|C|C|C}
        \pProtVar{} & \pNCMode{} & \pEncMods{} & \pAuthMods{} & \pRQSize{} & \pARQLimit{} & \pARQTimeout{} & \pRStrat{} & \pNumAttackers{} & \pAttackProb{} \\\hline
        varying     & varying    & varying     & varying      & unlimited  & 1            & 12             & \gls{dor}  & 8                & 0.2 \\
    \end{tabulary}
    \caption[Input parameters for number of crypto modules experiment]{long}
    \label{tab:setupnumcrypto}
\end{table}
- congestions in the queues in front of the crypto modules should be minimal
- less modules are better if possible because less chip area
- relatively high attack probabilities because more ARQs means more ver.+dec. retries at receivers, and network should be able to deal with that scenario
  "network needs to be equipped to deal with/handle periods of high traffic volumes"
- max. required enc. units: 4, because max. 2 flits can be drawn from the queues per cycle, so max. 4 flits active at the same time
- max. required auth. units: 12 (ind. auth), 10 (int. auth), 11 (gen. auth (2 flits = 11 cycles busy))
- criterion: max wait time needs to be 5 or less cycles

\begin{figure}
    \centering
    \begin{tabular}{ll}
        \input{gnuplot/plots/encmodules-ida} & \input{gnuplot/plots/authmodules-ida.tex} \\
        \input{gnuplot/plots/encmodules-iwa} & % GNUPLOT: LaTeX picture with Postscript
\begingroup
\newcommand{\ft}[0]{\footnotesize}\newcommand{\ty}[0]{\tiny}
  \makeatletter
  \providecommand\color[2][]{%
    \GenericError{(gnuplot) \space\space\space\@spaces}{%
      Package color not loaded in conjunction with
      terminal option `colourtext'%
    }{See the gnuplot documentation for explanation.%
    }{Either use 'blacktext' in gnuplot or load the package
      color.sty in LaTeX.}%
    \renewcommand\color[2][]{}%
  }%
  \providecommand\includegraphics[2][]{%
    \GenericError{(gnuplot) \space\space\space\@spaces}{%
      Package graphicx or graphics not loaded%
    }{See the gnuplot documentation for explanation.%
    }{The gnuplot epslatex terminal needs graphicx.sty or graphics.sty.}%
    \renewcommand\includegraphics[2][]{}%
  }%
  \providecommand\rotatebox[2]{#2}%
  \@ifundefined{ifGPcolor}{%
    \newif\ifGPcolor
    \GPcolortrue
  }{}%
  \@ifundefined{ifGPblacktext}{%
    \newif\ifGPblacktext
    \GPblacktextfalse
  }{}%
  % define a \g@addto@macro without @ in the name:
  \let\gplgaddtomacro\g@addto@macro
  % define empty templates for all commands taking text:
  \gdef\gplbacktext{}%
  \gdef\gplfronttext{}%
  \makeatother
  \ifGPblacktext
    % no textcolor at all
    \def\colorrgb#1{}%
    \def\colorgray#1{}%
  \else
    % gray or color?
    \ifGPcolor
      \def\colorrgb#1{\color[rgb]{#1}}%
      \def\colorgray#1{\color[gray]{#1}}%
      \expandafter\def\csname LTw\endcsname{\color{white}}%
      \expandafter\def\csname LTb\endcsname{\color{black}}%
      \expandafter\def\csname LTa\endcsname{\color{black}}%
      \expandafter\def\csname LT0\endcsname{\color[rgb]{1,0,0}}%
      \expandafter\def\csname LT1\endcsname{\color[rgb]{0,1,0}}%
      \expandafter\def\csname LT2\endcsname{\color[rgb]{0,0,1}}%
      \expandafter\def\csname LT3\endcsname{\color[rgb]{1,0,1}}%
      \expandafter\def\csname LT4\endcsname{\color[rgb]{0,1,1}}%
      \expandafter\def\csname LT5\endcsname{\color[rgb]{1,1,0}}%
      \expandafter\def\csname LT6\endcsname{\color[rgb]{0,0,0}}%
      \expandafter\def\csname LT7\endcsname{\color[rgb]{1,0.3,0}}%
      \expandafter\def\csname LT8\endcsname{\color[rgb]{0.5,0.5,0.5}}%
    \else
      % gray
      \def\colorrgb#1{\color{black}}%
      \def\colorgray#1{\color[gray]{#1}}%
      \expandafter\def\csname LTw\endcsname{\color{white}}%
      \expandafter\def\csname LTb\endcsname{\color{black}}%
      \expandafter\def\csname LTa\endcsname{\color{black}}%
      \expandafter\def\csname LT0\endcsname{\color{black}}%
      \expandafter\def\csname LT1\endcsname{\color{black}}%
      \expandafter\def\csname LT2\endcsname{\color{black}}%
      \expandafter\def\csname LT3\endcsname{\color{black}}%
      \expandafter\def\csname LT4\endcsname{\color{black}}%
      \expandafter\def\csname LT5\endcsname{\color{black}}%
      \expandafter\def\csname LT6\endcsname{\color{black}}%
      \expandafter\def\csname LT7\endcsname{\color{black}}%
      \expandafter\def\csname LT8\endcsname{\color{black}}%
    \fi
  \fi
    \setlength{\unitlength}{0.0500bp}%
    \ifx\gptboxheight\undefined%
      \newlength{\gptboxheight}%
      \newlength{\gptboxwidth}%
      \newsavebox{\gptboxtext}%
    \fi%
    \setlength{\fboxrule}{0.5pt}%
    \setlength{\fboxsep}{1pt}%
\begin{picture}(4320.00,3310.00)%
    \gplgaddtomacro\gplbacktext{%
      \csname LTb\endcsname%
      \put(330,660){\makebox(0,0)[r]{\strut{}\ft 1}}%
      \csname LTb\endcsname%
      \put(330,1853){\makebox(0,0)[r]{\strut{}\ft 10}}%
      \csname LTb\endcsname%
      \put(330,3045){\makebox(0,0)[r]{\strut{}\ft 100}}%
      \put(462,440){\makebox(0,0){\strut{}\ft 1}}%
      \put(777,440){\makebox(0,0){\strut{}\ft 2}}%
      \put(1091,440){\makebox(0,0){\strut{}\ft 3}}%
      \put(1406,440){\makebox(0,0){\strut{}\ft 4}}%
      \put(1721,440){\makebox(0,0){\strut{}\ft 5}}%
      \put(2035,440){\makebox(0,0){\strut{}\ft 6}}%
      \put(2350,440){\makebox(0,0){\strut{}\ft 7}}%
      \put(2664,440){\makebox(0,0){\strut{}\ft 8}}%
      \put(2979,440){\makebox(0,0){\strut{}\ft 9}}%
      \put(3294,440){\makebox(0,0){\strut{}\ft 10}}%
      \put(3608,440){\makebox(0,0){\strut{}\ft 11}}%
      \put(3923,440){\makebox(0,0){\strut{}\ft 12}}%
    }%
    \gplgaddtomacro\gplfronttext{%
      \csname LTb\endcsname%
      \put(-137,1852){\rotatebox{-270}{\makebox(0,0){\strut{}\ft Enqueued time in cycles}}}%
      \put(2192,154){\makebox(0,0){\strut{}\ft No. of authentication modules}}%
      \csname LTb\endcsname%
      \put(3224,2912){\makebox(0,0)[r]{\strut{}\ty IWA-UC max}}%
      \csname LTb\endcsname%
      \put(3224,2772){\makebox(0,0)[r]{\strut{}\ty avg}}%
      \csname LTb\endcsname%
      \put(3224,2632){\makebox(0,0)[r]{\strut{}\ty G2C3 max}}%
      \csname LTb\endcsname%
      \put(3224,2492){\makebox(0,0)[r]{\strut{}\ty avg}}%
      \csname LTb\endcsname%
      \put(3224,2352){\makebox(0,0)[r]{\strut{}\ty G2C4 max}}%
      \csname LTb\endcsname%
      \put(3224,2212){\makebox(0,0)[r]{\strut{}\ty avg}}%
    }%
    \gplbacktext
    \put(0,0){\includegraphics{../plots/authmodules-iwa}}%
    \gplfronttext
  \end{picture}%
\endgroup
 \\
        \input{gnuplot/plots/encmodules-fga} & \input{gnuplot/plots/authmodules-fga.tex}
    \end{tabular}
    \caption[Results for number of crypto modules experiment]{The number of encryption modules (left column) and authentication modules (right column)
    is shown in relation to the maximum and average wait times of enqueued flits. Each row represents one protocol variant.}
    \label{fig:resultscryptomodules}
\end{figure}

\subsection{Router Input Queue Sizes}
\begin{table}
    \centering
    \begin{tabulary}{\textwidth}{C|C|C|C|C|C|C|C|C|C}
        \pProtVar{} & \pNCMode{} & \pEncMods{} & \pAuthMods{} & \pRQSize{} & \pARQLimit{} & \pARQTimeout{} & \pRStrat{} & \pNumAttackers{} & \pAttackProb{} \\\hline
        \gls{ida} & varying & 3 & 9 & varying & 1 & 12 & \gls{dor} & 8 & 0.2 \\
    \end{tabulary}
    \caption[Input parameters for router queue sizes experiment]{long}
    \label{tab:setupqueuesizes}
\end{table}
- only node input queues; local input queues are left unlimited for simulation purposes to avoid any potential flit drops due to such a queue being full
- why IDA? largest transmission units (G2C4) and one of the smallest (UC)
- goal: as usual: as small as possible (→ less chip area) but not increase latencies too much
- explored as last hyperparameter because with static DOR it only shifts where flits wait to be routed, congestions do not affect route
  - will be crucial for the adaptive strategies though
- chosen value: 6 (increasing it does not substantially lower max/avg times and all avgs are below 1 or just very slightly above it)

\begin{figure}
    \centering
    \begin{tabular}{ll}
        % GNUPLOT: LaTeX picture with Postscript
\begingroup
\newcommand{\ft}[0]{\footnotesize}\newcommand{\ty}[0]{\tiny}
  \makeatletter
  \providecommand\color[2][]{%
    \GenericError{(gnuplot) \space\space\space\@spaces}{%
      Package color not loaded in conjunction with
      terminal option `colourtext'%
    }{See the gnuplot documentation for explanation.%
    }{Either use 'blacktext' in gnuplot or load the package
      color.sty in LaTeX.}%
    \renewcommand\color[2][]{}%
  }%
  \providecommand\includegraphics[2][]{%
    \GenericError{(gnuplot) \space\space\space\@spaces}{%
      Package graphicx or graphics not loaded%
    }{See the gnuplot documentation for explanation.%
    }{The gnuplot epslatex terminal needs graphicx.sty or graphics.sty.}%
    \renewcommand\includegraphics[2][]{}%
  }%
  \providecommand\rotatebox[2]{#2}%
  \@ifundefined{ifGPcolor}{%
    \newif\ifGPcolor
    \GPcolortrue
  }{}%
  \@ifundefined{ifGPblacktext}{%
    \newif\ifGPblacktext
    \GPblacktextfalse
  }{}%
  % define a \g@addto@macro without @ in the name:
  \let\gplgaddtomacro\g@addto@macro
  % define empty templates for all commands taking text:
  \gdef\gplbacktext{}%
  \gdef\gplfronttext{}%
  \makeatother
  \ifGPblacktext
    % no textcolor at all
    \def\colorrgb#1{}%
    \def\colorgray#1{}%
  \else
    % gray or color?
    \ifGPcolor
      \def\colorrgb#1{\color[rgb]{#1}}%
      \def\colorgray#1{\color[gray]{#1}}%
      \expandafter\def\csname LTw\endcsname{\color{white}}%
      \expandafter\def\csname LTb\endcsname{\color{black}}%
      \expandafter\def\csname LTa\endcsname{\color{black}}%
      \expandafter\def\csname LT0\endcsname{\color[rgb]{1,0,0}}%
      \expandafter\def\csname LT1\endcsname{\color[rgb]{0,1,0}}%
      \expandafter\def\csname LT2\endcsname{\color[rgb]{0,0,1}}%
      \expandafter\def\csname LT3\endcsname{\color[rgb]{1,0,1}}%
      \expandafter\def\csname LT4\endcsname{\color[rgb]{0,1,1}}%
      \expandafter\def\csname LT5\endcsname{\color[rgb]{1,1,0}}%
      \expandafter\def\csname LT6\endcsname{\color[rgb]{0,0,0}}%
      \expandafter\def\csname LT7\endcsname{\color[rgb]{1,0.3,0}}%
      \expandafter\def\csname LT8\endcsname{\color[rgb]{0.5,0.5,0.5}}%
    \else
      % gray
      \def\colorrgb#1{\color{black}}%
      \def\colorgray#1{\color[gray]{#1}}%
      \expandafter\def\csname LTw\endcsname{\color{white}}%
      \expandafter\def\csname LTb\endcsname{\color{black}}%
      \expandafter\def\csname LTa\endcsname{\color{black}}%
      \expandafter\def\csname LT0\endcsname{\color{black}}%
      \expandafter\def\csname LT1\endcsname{\color{black}}%
      \expandafter\def\csname LT2\endcsname{\color{black}}%
      \expandafter\def\csname LT3\endcsname{\color{black}}%
      \expandafter\def\csname LT4\endcsname{\color{black}}%
      \expandafter\def\csname LT5\endcsname{\color{black}}%
      \expandafter\def\csname LT6\endcsname{\color{black}}%
      \expandafter\def\csname LT7\endcsname{\color{black}}%
      \expandafter\def\csname LT8\endcsname{\color{black}}%
    \fi
  \fi
    \setlength{\unitlength}{0.0500bp}%
    \ifx\gptboxheight\undefined%
      \newlength{\gptboxheight}%
      \newlength{\gptboxwidth}%
      \newsavebox{\gptboxtext}%
    \fi%
    \setlength{\fboxrule}{0.5pt}%
    \setlength{\fboxsep}{1pt}%
\begin{picture}(4030.00,3600.00)%
    \gplgaddtomacro\gplbacktext{%
      \csname LTb\endcsname%
      \put(990,660){\makebox(0,0)[r]{\strut{}\ft 0.1}}%
      \put(990,1216){\makebox(0,0)[r]{\strut{}\ft 1}}%
      \put(990,1773){\makebox(0,0)[r]{\strut{}\ft 10}}%
      \put(990,2329){\makebox(0,0)[r]{\strut{}\ft 100}}%
      \put(990,2885){\makebox(0,0)[r]{\strut{}\ft 1000}}%
      \put(3633,440){\makebox(0,0){\strut{}$\infty$}}%
      \put(1122,440){\makebox(0,0){\strut{}\ft 1}}%
      \put(1373,440){\makebox(0,0){\strut{}\ft 2}}%
      \put(1624,440){\makebox(0,0){\strut{}\ft 3}}%
      \put(1875,440){\makebox(0,0){\strut{}\ft 4}}%
      \put(2126,440){\makebox(0,0){\strut{}\ft 5}}%
      \put(2378,440){\makebox(0,0){\strut{}\ft 6}}%
      \put(2629,440){\makebox(0,0){\strut{}\ft 7}}%
      \put(2880,440){\makebox(0,0){\strut{}\ft 8}}%
      \put(3131,440){\makebox(0,0){\strut{}\ft 9}}%
      \put(3382,440){\makebox(0,0){\strut{}\ft 10}}%
    }%
    \gplgaddtomacro\gplfronttext{%
      \csname LTb\endcsname%
      \put(352,1772){\rotatebox{-270}{\makebox(0,0){\strut{}\ft Port queue lengths in flits}}}%
      \put(2377,154){\makebox(0,0){\strut{}\ft Maximum port queue size}}%
      \csname LTb\endcsname%
      \put(1774,3462){\makebox(0,0)[r]{\strut{}\ty IDA-UC max}}%
      \csname LTb\endcsname%
      \put(1774,3312){\makebox(0,0)[r]{\strut{}\ty G2C3 max}}%
      \csname LTb\endcsname%
      \put(1774,3162){\makebox(0,0)[r]{\strut{}\ty G2C4 max}}%
      \csname LTb\endcsname%
      \put(3457,3462){\makebox(0,0)[r]{\strut{}\ty avg}}%
      \csname LTb\endcsname%
      \put(3457,3312){\makebox(0,0)[r]{\strut{}\ty avg}}%
      \csname LTb\endcsname%
      \put(3457,3162){\makebox(0,0)[r]{\strut{}\ty avg}}%
    }%
    \gplbacktext
    \put(0,0){\includegraphics{../plots/queuelengths-ports}}%
    \gplfronttext
  \end{picture}%
\endgroup
 & % GNUPLOT: LaTeX picture with Postscript
\begingroup
\newcommand{\ft}[0]{\footnotesize}\newcommand{\ty}[0]{\tiny}
  \makeatletter
  \providecommand\color[2][]{%
    \GenericError{(gnuplot) \space\space\space\@spaces}{%
      Package color not loaded in conjunction with
      terminal option `colourtext'%
    }{See the gnuplot documentation for explanation.%
    }{Either use 'blacktext' in gnuplot or load the package
      color.sty in LaTeX.}%
    \renewcommand\color[2][]{}%
  }%
  \providecommand\includegraphics[2][]{%
    \GenericError{(gnuplot) \space\space\space\@spaces}{%
      Package graphicx or graphics not loaded%
    }{See the gnuplot documentation for explanation.%
    }{The gnuplot epslatex terminal needs graphicx.sty or graphics.sty.}%
    \renewcommand\includegraphics[2][]{}%
  }%
  \providecommand\rotatebox[2]{#2}%
  \@ifundefined{ifGPcolor}{%
    \newif\ifGPcolor
    \GPcolortrue
  }{}%
  \@ifundefined{ifGPblacktext}{%
    \newif\ifGPblacktext
    \GPblacktextfalse
  }{}%
  % define a \g@addto@macro without @ in the name:
  \let\gplgaddtomacro\g@addto@macro
  % define empty templates for all commands taking text:
  \gdef\gplbacktext{}%
  \gdef\gplfronttext{}%
  \makeatother
  \ifGPblacktext
    % no textcolor at all
    \def\colorrgb#1{}%
    \def\colorgray#1{}%
  \else
    % gray or color?
    \ifGPcolor
      \def\colorrgb#1{\color[rgb]{#1}}%
      \def\colorgray#1{\color[gray]{#1}}%
      \expandafter\def\csname LTw\endcsname{\color{white}}%
      \expandafter\def\csname LTb\endcsname{\color{black}}%
      \expandafter\def\csname LTa\endcsname{\color{black}}%
      \expandafter\def\csname LT0\endcsname{\color[rgb]{1,0,0}}%
      \expandafter\def\csname LT1\endcsname{\color[rgb]{0,1,0}}%
      \expandafter\def\csname LT2\endcsname{\color[rgb]{0,0,1}}%
      \expandafter\def\csname LT3\endcsname{\color[rgb]{1,0,1}}%
      \expandafter\def\csname LT4\endcsname{\color[rgb]{0,1,1}}%
      \expandafter\def\csname LT5\endcsname{\color[rgb]{1,1,0}}%
      \expandafter\def\csname LT6\endcsname{\color[rgb]{0,0,0}}%
      \expandafter\def\csname LT7\endcsname{\color[rgb]{1,0.3,0}}%
      \expandafter\def\csname LT8\endcsname{\color[rgb]{0.5,0.5,0.5}}%
    \else
      % gray
      \def\colorrgb#1{\color{black}}%
      \def\colorgray#1{\color[gray]{#1}}%
      \expandafter\def\csname LTw\endcsname{\color{white}}%
      \expandafter\def\csname LTb\endcsname{\color{black}}%
      \expandafter\def\csname LTa\endcsname{\color{black}}%
      \expandafter\def\csname LT0\endcsname{\color{black}}%
      \expandafter\def\csname LT1\endcsname{\color{black}}%
      \expandafter\def\csname LT2\endcsname{\color{black}}%
      \expandafter\def\csname LT3\endcsname{\color{black}}%
      \expandafter\def\csname LT4\endcsname{\color{black}}%
      \expandafter\def\csname LT5\endcsname{\color{black}}%
      \expandafter\def\csname LT6\endcsname{\color{black}}%
      \expandafter\def\csname LT7\endcsname{\color{black}}%
      \expandafter\def\csname LT8\endcsname{\color{black}}%
    \fi
  \fi
    \setlength{\unitlength}{0.0500bp}%
    \ifx\gptboxheight\undefined%
      \newlength{\gptboxheight}%
      \newlength{\gptboxwidth}%
      \newsavebox{\gptboxtext}%
    \fi%
    \setlength{\fboxrule}{0.5pt}%
    \setlength{\fboxsep}{1pt}%
\begin{picture}(3888.00,3600.00)%
    \gplgaddtomacro\gplbacktext{%
      \csname LTb\endcsname%
      \put(369,638){\makebox(0,0)[r]{\strut{}\ft 0.1}}%
      \csname LTb\endcsname%
      \put(369,1207){\makebox(0,0)[r]{\strut{}\ft 1}}%
      \csname LTb\endcsname%
      \put(369,1777){\makebox(0,0)[r]{\strut{}\ft 10}}%
      \csname LTb\endcsname%
      \put(369,2346){\makebox(0,0)[r]{\strut{}\ft 100}}%
      \csname LTb\endcsname%
      \put(369,2915){\makebox(0,0)[r]{\strut{}\ft 1000}}%
      \put(3491,418){\makebox(0,0){\strut{}$\infty$}}%
      \put(501,418){\makebox(0,0){\strut{}\ft 1}}%
      \put(800,418){\makebox(0,0){\strut{}\ft 2}}%
      \put(1099,418){\makebox(0,0){\strut{}\ft 3}}%
      \put(1398,418){\makebox(0,0){\strut{}\ft 4}}%
      \put(1697,418){\makebox(0,0){\strut{}\ft 5}}%
      \put(1996,418){\makebox(0,0){\strut{}\ft 6}}%
      \put(2295,418){\makebox(0,0){\strut{}\ft 7}}%
      \put(2594,418){\makebox(0,0){\strut{}\ft 8}}%
      \put(2893,418){\makebox(0,0){\strut{}\ft 9}}%
      \put(3192,418){\makebox(0,0){\strut{}\ft 10}}%
    }%
    \gplgaddtomacro\gplfronttext{%
      \csname LTb\endcsname%
      \put(-203,1776){\rotatebox{-270}{\makebox(0,0){\strut{}\ft Local queue lengths in flits}}}%
      \put(1996,154){\makebox(0,0){\strut{}\ft Maximum port queue size}}%
      \csname LTb\endcsname%
      \put(1429,3467){\makebox(0,0)[r]{\strut{}\ty IDA-UC max}}%
      \csname LTb\endcsname%
      \put(1429,3327){\makebox(0,0)[r]{\strut{}\ty G2C3 max}}%
      \csname LTb\endcsname%
      \put(1429,3187){\makebox(0,0)[r]{\strut{}\ty G2C4 max}}%
      \csname LTb\endcsname%
      \put(2500,3467){\makebox(0,0)[r]{\strut{}\ty avg}}%
      \csname LTb\endcsname%
      \put(2500,3327){\makebox(0,0)[r]{\strut{}\ty avg}}%
      \csname LTb\endcsname%
      \put(2500,3187){\makebox(0,0)[r]{\strut{}\ty avg}}%
    }%
    \gplbacktext
    \put(0,0){\includegraphics{../plots/queuelengths-local}}%
    \gplfronttext
  \end{picture}%
\endgroup

    \end{tabular}
    \caption[Results for router queue lengths experiment]{long}
    \label{fig:resultsqueuelengths}
\end{figure}

\section{Performance Comparisons}\label{sec:perfcompeval}
\subsection{Performances Of The Protocol Variants}
Record:
- acceptance rate (aka actual injection rate)
- information rate (source flits / total flits)
- residual error probability
- end-to-end latency

\begin{table}
    \centering
    \begin{tabulary}{\textwidth}{C|C|C|C|C|C|C|C|C|C}
        \pProtVar{} & \pNCMode{} & \pEncMods{} & \pAuthMods{} & \pRQSize{} & \pARQLimit{} & \pARQTimeout{} & \pRStrat{} & \pNumAttackers{} & \pAttackProb{} \\\hline
        varying & varying & 3 & 9 & 6 & 1 & 12 & \gls{dor} & 8 & varying \\
    \end{tabulary}
    \caption[Input parameters for protocol variant experiment]{long}
    \label{tab:setupprotvarexperiment}
\end{table}

\begin{figure}
    \centering
    \begin{tabular}{ll}
        % GNUPLOT: LaTeX picture with Postscript
\begingroup
\newcommand{\ft}[0]{\footnotesize}\newcommand{\ty}[0]{\tiny}
  \makeatletter
  \providecommand\color[2][]{%
    \GenericError{(gnuplot) \space\space\space\@spaces}{%
      Package color not loaded in conjunction with
      terminal option `colourtext'%
    }{See the gnuplot documentation for explanation.%
    }{Either use 'blacktext' in gnuplot or load the package
      color.sty in LaTeX.}%
    \renewcommand\color[2][]{}%
  }%
  \providecommand\includegraphics[2][]{%
    \GenericError{(gnuplot) \space\space\space\@spaces}{%
      Package graphicx or graphics not loaded%
    }{See the gnuplot documentation for explanation.%
    }{The gnuplot epslatex terminal needs graphicx.sty or graphics.sty.}%
    \renewcommand\includegraphics[2][]{}%
  }%
  \providecommand\rotatebox[2]{#2}%
  \@ifundefined{ifGPcolor}{%
    \newif\ifGPcolor
    \GPcolortrue
  }{}%
  \@ifundefined{ifGPblacktext}{%
    \newif\ifGPblacktext
    \GPblacktextfalse
  }{}%
  % define a \g@addto@macro without @ in the name:
  \let\gplgaddtomacro\g@addto@macro
  % define empty templates for all commands taking text:
  \gdef\gplbacktext{}%
  \gdef\gplfronttext{}%
  \makeatother
  \ifGPblacktext
    % no textcolor at all
    \def\colorrgb#1{}%
    \def\colorgray#1{}%
  \else
    % gray or color?
    \ifGPcolor
      \def\colorrgb#1{\color[rgb]{#1}}%
      \def\colorgray#1{\color[gray]{#1}}%
      \expandafter\def\csname LTw\endcsname{\color{white}}%
      \expandafter\def\csname LTb\endcsname{\color{black}}%
      \expandafter\def\csname LTa\endcsname{\color{black}}%
      \expandafter\def\csname LT0\endcsname{\color[rgb]{1,0,0}}%
      \expandafter\def\csname LT1\endcsname{\color[rgb]{0,1,0}}%
      \expandafter\def\csname LT2\endcsname{\color[rgb]{0,0,1}}%
      \expandafter\def\csname LT3\endcsname{\color[rgb]{1,0,1}}%
      \expandafter\def\csname LT4\endcsname{\color[rgb]{0,1,1}}%
      \expandafter\def\csname LT5\endcsname{\color[rgb]{1,1,0}}%
      \expandafter\def\csname LT6\endcsname{\color[rgb]{0,0,0}}%
      \expandafter\def\csname LT7\endcsname{\color[rgb]{1,0.3,0}}%
      \expandafter\def\csname LT8\endcsname{\color[rgb]{0.5,0.5,0.5}}%
    \else
      % gray
      \def\colorrgb#1{\color{black}}%
      \def\colorgray#1{\color[gray]{#1}}%
      \expandafter\def\csname LTw\endcsname{\color{white}}%
      \expandafter\def\csname LTb\endcsname{\color{black}}%
      \expandafter\def\csname LTa\endcsname{\color{black}}%
      \expandafter\def\csname LT0\endcsname{\color{black}}%
      \expandafter\def\csname LT1\endcsname{\color{black}}%
      \expandafter\def\csname LT2\endcsname{\color{black}}%
      \expandafter\def\csname LT3\endcsname{\color{black}}%
      \expandafter\def\csname LT4\endcsname{\color{black}}%
      \expandafter\def\csname LT5\endcsname{\color{black}}%
      \expandafter\def\csname LT6\endcsname{\color{black}}%
      \expandafter\def\csname LT7\endcsname{\color{black}}%
      \expandafter\def\csname LT8\endcsname{\color{black}}%
    \fi
  \fi
    \setlength{\unitlength}{0.0500bp}%
    \ifx\gptboxheight\undefined%
      \newlength{\gptboxheight}%
      \newlength{\gptboxwidth}%
      \newsavebox{\gptboxtext}%
    \fi%
    \setlength{\fboxrule}{0.5pt}%
    \setlength{\fboxsep}{1pt}%
\begin{picture}(4030.00,4320.00)%
    \gplgaddtomacro\gplbacktext{%
      \csname LTb\endcsname%
      \put(528,638){\makebox(0,0)[r]{\strut{}\ft 0.2}}%
      \csname LTb\endcsname%
      \put(528,955){\makebox(0,0)[r]{\strut{}\ft 0.21}}%
      \csname LTb\endcsname%
      \put(528,1273){\makebox(0,0)[r]{\strut{}\ft 0.22}}%
      \csname LTb\endcsname%
      \put(528,1590){\makebox(0,0)[r]{\strut{}\ft 0.23}}%
      \csname LTb\endcsname%
      \put(528,1908){\makebox(0,0)[r]{\strut{}\ft 0.24}}%
      \csname LTb\endcsname%
      \put(528,2225){\makebox(0,0)[r]{\strut{}\ft 0.25}}%
      \csname LTb\endcsname%
      \put(528,2543){\makebox(0,0)[r]{\strut{}\ft 0.26}}%
      \csname LTb\endcsname%
      \put(528,2860){\makebox(0,0)[r]{\strut{}\ft 0.27}}%
      \csname LTb\endcsname%
      \put(528,3178){\makebox(0,0)[r]{\strut{}\ft 0.28}}%
      \csname LTb\endcsname%
      \put(528,3495){\makebox(0,0)[r]{\strut{}\ft 0.29}}%
      \put(660,418){\makebox(0,0){\strut{}\ft 0}}%
      \put(1778,418){\makebox(0,0){\strut{}\ft 0.1}}%
      \put(2897,418){\makebox(0,0){\strut{}\ft 0.2}}%
      \put(3121,418){\makebox(0,0){\strut{}}}%
      \put(3344,418){\makebox(0,0){\strut{}}}%
      \put(3568,418){\makebox(0,0){\strut{}\ft 0.5}}%
    }%
    \gplgaddtomacro\gplfronttext{%
      \csname LTb\endcsname%
      \put(-44,2066){\rotatebox{-270}{\makebox(0,0){\strut{}\ft Acceptance rate}}}%
      \put(2114,154){\makebox(0,0){\strut{}\ft Attack probabilities}}%
      \csname LTb\endcsname%
      \put(1547,4187){\makebox(0,0)[r]{\strut{}\ty IDA-UC}}%
      \csname LTb\endcsname%
      \put(1547,4047){\makebox(0,0)[r]{\strut{}\ty G2C3}}%
      \csname LTb\endcsname%
      \put(1547,3907){\makebox(0,0)[r]{\strut{}\ty G2C4}}%
      \csname LTb\endcsname%
      \put(1547,3767){\makebox(0,0)[r]{\strut{}\ty IWA-UC}}%
      \csname LTb\endcsname%
      \put(2954,4187){\makebox(0,0)[r]{\strut{}\ty G2C3}}%
      \csname LTb\endcsname%
      \put(2954,4047){\makebox(0,0)[r]{\strut{}\ty G2C4}}%
      \csname LTb\endcsname%
      \put(2954,3907){\makebox(0,0)[r]{\strut{}\ty FGA-G2C3}}%
      \csname LTb\endcsname%
      \put(2954,3767){\makebox(0,0)[r]{\strut{}\ty G2C4}}%
    }%
    \gplbacktext
    \put(0,0){\includegraphics{../plots/main-acceptancerate}}%
    \gplfronttext
  \end{picture}%
\endgroup
 & \input{gnuplot/plots/main-informationrate} \\
        \input{gnuplot/plots/main-residualerror} & \input{gnuplot/plots/main-endtoendlatency}
    \end{tabular}
    \caption[Results for protocol variant experiment]{long}
    \label{fig:resultsprotvarexperiment}
\end{figure}

Observations:
- NC does not improve (=decrease) residual error. why? might be because you have 1 ARQ per generation (=0.5ARQs per source flit) for NC, but 1 ARQ per
source flit for UC (IDA protocol)
  - how to confirm? UC with ARQ limit 1 vs. NC with ARQ limit 2 (then same amount of ARQs per source flit)
  - ARQ flit itself is not coded → hence when it gets attacked, all retransmissions fail → for NC, on average a larger number of flits is affected by
    this → how to confirm? set routers to not attack ARQ flits
  - may be exciting to see if this is the case as well for IWA because there you have 1 ARQ per source flit for both UC and NC

\subsection{Performances Of The Routing Strategies}
Take: best protocol variant (IWA), use it to find the best routing strategy

\begin{table}
    \centering
    \begin{tabulary}{\textwidth}{C|C|C|C|C|C|C|C|C|C}
        \pProtVar{} & \pNCMode{} & \pEncMods{} & \pAuthMods{} & \pRQSize{} & \pARQLimit{} & \pARQTimeout{} & \pRStrat{} & \pNumAttackers{} & \pAttackProb{} \\\hline
        \gls{iwa} & varying & 3 & 9 & 6 & 1 & 12 & varying & 8 & varying \\
    \end{tabulary}
    \caption[Input parameters for main experiment]{long}
    \label{tab:setuproutingstratexperiment}
\end{table}


% Heatmap of router workload?
\begin{figure}
    \input{gnuplot/plots/heatmap-dor}
    \caption[short]{Heatmaps}
    \label{fig:resultsroutingstratheatmaps}
\end{figure}

\iffalse
Experiment setup parameter tables:
- NC mode (UC, G2C3, G2C4)
- ...

ARQ Limit: 1, at most 2 because more ARQs allowed per transmission unit means larger retransmission buffers everywhere

Do one experiment with the 8 routers with attack probability one and compare the routing strategies

"Results for IWA-NC confirm the alleged positive effect of NC in unreliable networks"

"Creation rate adjustment to keep the base injection rate constant is reflected in the differing information rates"

"IDA: 4 flits need to arrive unharmed for successful transmission unit, 3 for FGA, and 2 for IWA"

"NC proves to provide a performance improvement, although more in the sense of less residual errors than a decrease in latency since ARQs are still
limited to 1 per transmission unit across all variants."

\section{Statistics}
\begin{itemize}
    \item Injection/acceptance rate: [0, 1] (at processing element and at network interface)
    \item Queue lengths and buffer sizes
    \item Workload of crypto units
    \item Average/max flit waiting time at entry guard
    \item Average/max hop count
\end{itemize}

\section{Area Overhead}
\fi


    \chapter{Conclusion}\label{ch:conclusion}
    In this thesis, a novel approach for securing the communications within a \gls{noc} against routers compromised with hardware trojans was presented
and evaluated. For this purpose, a protocol was designed that achieves the protection goals of confidentiality and integrity. After examining
and assessing related work from this field of research, an in-depth explanation of said protocol was given. Several performance indicators were then
evaluated through extensive simulations with varying parameters. To conduct these experiment, a dedicated software was developed that facilitates
cycle-accurate analyses with detailed statistics for all facets of the protocol. Finally, the most suitable configuration was determined based on the
obtained results.

Using PRINCE as a lightweight cipher for all cryptographic operations has proven to be a sound choice. Its ability to process a 64-bit block within two
clock cycles (at 500 MHz clock speed) allows for low-latency encryptions and authentications. Although its hardware implementation is
rather large with around 8260 \gls{ge}, relatively little parallel crypto modules per network interface are required (3 and 9 for encryption and
authentication, respectively). In the work by \citeauthor{moriam18activeattackers}, which this thesis builds upon (cf. Section \ref{sec:ncfornoc}),
mCrypton \cite{lim06mcrypton} was employed as the underlying cipher. Requiring 13 cycles per block, it necessitates 18 parallel modules per network
interface for authentication only \cite[5]{moriam18activeattackers}. However, with a size of 2681 \gls{ge}, it is significantly smaller than PRINCE.
In total, their approach consumes about half the chip area than the one proposed in this thesis, but provides no confidentiality and is slower by a
factor of 6.5.

Network coding provided the advantages that were envisioned. In unreliable networks, it was able to reduce the network load by up to 17\% and the
number of residual errors by up to 86\% at medium attack probabilities of 0.2\footnote{The percentages correspond to the \gls{iwa}-\gls{dor} protocol
with an \gls{arq} limit of 1, comparing UC and G2C4.}. However, the redundant transmissions also reduced the information rate by around 40\%. Thus,
while being a great asset in error-prone networks, it has a negative impact on the performance in reliable networks devoid of adversaries.

The ability to request retransmissions via \glspl{arq} is crucial to provide a possibility to recover from integrity breaches and still facilitate
successful transmissions of source flits. Since the \glspl{arq} and retransmissions themselves are also vulnerable to attackers, doubling
the number of allowed \glspl{arq} further improved the resilience of the protocol. With the same configurations as above, it decreased the amount of
residual errors by 23\% at the cost of a network load increase of 7\%.

There are still many avenues of improvement that have not been explored here as they would go beyond the scope of this thesis. Nonetheless, they may
be of interest and spark inspiration for future work that draws upon the presented ideas. The following list provides a selection of such avenues:
\begin{itemize}
    \item \textbf{Burst mode.} As the flits already contain a currently unused bit to indicate burst mode, the investigation of the presented
        protocols and routing strategies in combination with this mode suggests itself.
    \item \textbf{Detecting full transmission unit loss.} As each pair of sender and receiver is assumed to use its own sequence of flit and
        generation IDs, it is possible to detect the loss of complete transmission units if the stream of IDs is discontinuous on the receiving side.
        In this case, an \gls{arq} could be issued that requests all flits associated with the missing ID.
    \item \textbf{Protecting the \gls{arq} flits.} In the designed protocol, \glspl{arq} are neither authenticated nor network coded. Hence, they are
        one of the most vulnerable parts of the protocol: in case of an attack on an \gls{arq} flit, the retransmissions fail. Adding authentication
        or redundancy to \glspl{arq} may help to decrease the residual error probabilities even further, albeit at the cost of a higher network load.
    \item \textbf{Different paths for retransmissions.} To increase the probability of successful retransmissions, it may be beneficial to ensure that
        they take different paths than the initially sent flits, even if the resulting routes are not minimal. If a specific flit is requested via an
        \gls{arq}, its original path could be prohibited for the subsequent retransmission.
    \item \textbf{Varying traffic models.} The employed traffic generation pattern uses a uniform random distribution of created flits
        among the processing elements. In real-world applications, nodes may have differing injection rates, which influences the network load
        distribution. For such scenarios, dynamic routing strategies may provide stronger beneficial effects over static \gls{dor}.
    \item \textbf{Different attacker model.} The attacker model for the evaluation in this thesis is based on random flit drops and modifications. An
        adversary, however, may for instance be interested in disrupting only specific communications with a certain source or destination, and then
        attack every matching flit.
    \item \textbf{Further network coding schemes.} While a generation size of two has proven to be beneficial over larger ones
        \cite{moriam15manycorenc}, the composition of the generations may be altered. For instance, an interesting approach would be to include
        \glspl{mac} of the source flits inside the generations instead of authenticating the encoded combinations. Then, for each source flit, a
        generation could be formed from the data and \gls{mac} flit.
    \item \textbf{Truncated \glspl{mac} for \gls{iwa}.} In the presented protocol variant \gls{iwa}, authcodes are used to authenticate flits, which
        are computed from a pseudo-random intermediate key and the data part of the payload. An alternative would be to compute a \gls{mac} over the
        header fields and the 32 payload bits and then use only the first half of the result as authentication information. This would
        save one clock cycle per authentication procedure as the final step of the authcode computation (see Figure \ref{fig:computeauthcodeuc}) could
        be omitted. However, truncating a \gls{mac} in such a manner may entail a weaker overall security.
    \item \textbf{Protecting the address header field.} The address header field of the flits, which contains a 32 bit memory address, remains unused
        for this thesis. However, if it is utilized in the future, it may be desirable to encrypt it together with the payload. Since memory layouts
        are not necessarily random, it may convey interesting information to an eavesdropping adversary.
    \item \textbf{Virtual channels.} For this thesis, virtual channels were not considered in the investigated dynamic routing strategies. For a
        real-world implementation, however, they are necessary to ensure deadlock freedom. Hence, it may be desirable to evaluate the effects of their
        presence for future work with dynamic routing.
\end{itemize}
\vspace{0.5\baselineskip}

All in all, the protection measures investigated and evaluated in this thesis provide a promising approach to secure and efficient communications for
\glspl{noc}. Encryption and authentication with a lightweight cipher are able to fulfill the goals of confidentiality and integrity for the
transmitted information while network coding adds a layer of redundancy and resilience. Furthermore, dynamic routing strategies entail the usage of
multiple paths even among the same communication partners, but their potential for performance enhancements depends on the network load distribution. There are still many open
avenues to further improve the presented schemes; the auspicious results so far seem encouraging for future research in this direction.


    \chapter{Future Work}\label{ch:futurework}
    \begin{itemize}
    \item Sequential IDs for uncoded and network coded → full transmission unit loss can be detected, ARQ issued
    \item Authenticated ARQs
    \item Different attacker models/smarter attackers
    \item If attacker model changes, i.e. attackers start to drop specific/whole generations,
        how does that influence the Routing/ARQ design?
    \item My idea of G3C4 with generation MAC as part of the G3? → ask if this is feasible
    \item Authenticated encryption schemes in the NIs
    \item Local network coding, local encoding vectors
    \item Burst mode (w/ head and tail flits)
    \item TDM route selecting
    \item RNG for path selection in simulator → according to \cite{stefan11enhancingnocs}, non-static selection can be very expensive in area →
        explore static routes?
    \item Intelligent attackers → modify ARQs or send own ARQs → they are unauthenticated so DoS attack by maximizing retransmissions
\end{itemize}




    \chapter{Talk About}
    \begin{itemize}
        \item Welche Komponenten brauchen wie viele Takte
        \item Anzahl der Enc/Auth units bei den verschiedenen Methoden (mehr enc als auth bei Methode 2)
        \item GALS design pattern - wir sind aber auf jeden Fall synchron innerhalb einer Node, d.h. wenn GALS vs GS diskutiert wird,
            ändert sich nur die Zeit der Übertragung zwischen den Routern → zur Vereinfachung wird GS angenommen
        \item Leichtgewichtige Krypto: FPGA vs. ASIC, speedup von Faktor 3-4 wahrscheinlich möglich (→ critical path delay) \cite{kuon07fpgavsasic}
            → AES und AES\_INV zum Vergleich, da ähnlicher Aufbau von kombinatorischer Logik + FlipFlops (Rundenfunktionen etc.)
        \item Statistiken: Wie viele Crypto Units für Auth+Enc sind gleichzeitig besetzt (→ Auslastung)
        \item Threat model, protection goals
        \item node-unique FIDs/GIDs have advantage that 24 bits are not full nearly as fast as with globally unique IDs
        \item Deadlock/Livelock possibilities? D yes, L not
        \item Unicast/Multicast
        \item Head-of-line blocking in routers → future work w/ virtual channels/TDM etc.
        \item Säulendiagramm (mit verschiedenfarbigen Abschnitten?): Latenz der (einzelnen Phasen der) Flitübertragung im Vergleich: mit NC, mit
            Enc/auth methoden, mit beidem, ohne alles
        \item Tortendiagramm: Für NC+Enc/Auth Methoden, welche Phasen dauern wie lange anteilig
    \end{itemize}

    \section{Assumptions}
    \begin{itemize}
        \item only input buffers are used (for app, NI, router), no output buffers
        \item network coding module stores flits until enough flits with the same
            destination are available (in our case: 2)
        \item network coding (creating all 3 combinations) takes one cycle, but
            because we can only send one flit out per cycle, it takes 3 cycles until
            all combinations are sent (so in the end it's one combination per cycle)
        \item we have a variable number of authentication units and the amount of
            cycles the encryption algorithm takes can be set by a parameter
        \item if all crypto units are busy, the encoded flit combinations are held
            back in a dedicated buffer
        \item if e.g. authentication method 1 is used (1 MAC flit per data flit),
            the data flit is sent out one cycle before the MAC finished computing,
            so the MAC can be sent immediately when it's ready
        \item the encoding/authentication and decoding/verification pipelines in the
            network interface share the same crypto units (but this is not
            implemented yet)
        \item when a flit arrives at the NI from the router, the first thing that is
            done is checking if this is an ARQ. If yes, it is delivered directly to
            the retransmission buffer. Otherwise, it goes through the normal network
            decoding pipeline
        \item when an ARQ arrives at a NI, it takes one cycle to look up the correct
            flit and it will be sent out in the same cycle (not sure about this, two
            cycles might be more realistic)
        \item size of the retransmission buffer is configurable
        \item retransmission buffers use a FIFO caching strategy
        \item if an ARQ and a new data flit arrive a the retransmission buffer
            during the same clock cycle, the ARQ answer (retransmission) has
            priority to be sent out first
        \item in a router, several flits can be routed simultaneously, provided that
            they do not share an input or output port
        \item also in router: flits are held back in the input queues if the
            receiving router's input queue is full (this was discussed in an earlier
            email)
        \item in uncoded flits, the GID header contains a flit ID instead of
            generation ID, and MAC and data flit have the same ID
        \item the different values for the mode header are: normal data flit, mac
            flit, data/mac combined, ARQ
        \item entry guard can distribute a departing and an arriving flit at the same
            time, as long as units are free
        \item it does not matter whether data or mac flit is sent out first (order),
            because they can arrive out-of-order at the destination node anyway
        \item encryption units can also be used for decryption
    \end{itemize}

    \chapter{Notes}
    \section{NoC Design}
    \begin{itemize}
        \item Router: in+out buffer vs. only input buffer (2 cycles vs. 1 cycle routing)
        \item GenTraffic: injection rate of 0.2 was considered in Sadia's calculations and is a good starting point
        \item Network Interface: sender and receiver share the crypto units → we need to keep track of where the flit has to go after leaving the crypto unit → flag at crypto unit itself?
        \item Speed: \gls{noc} usually runs at ~500 MHz or a bit lower, but definitely runs a lot faster than the maximum frequency of the lightweight encryption algorithms. Using multiple clock domains is hard and requires expertise
        \item Crypto algorithms: 18 units for mCrypton was planned; also possible to use multiple (but less) PRINCE units, but then the clock domain problem arises again
    \end{itemize}

    % % % Glossary % % %
    %\glsaddall % Print all glossary entries, not only the referenced ones
    % TODO: add specific entries (the ones only occuring in images) explicitly
    \printglossaries
	
	% % % Bibliography % % %
	%\nocite{*} % Put all entries in the bibliography, not only those cited in the document
    \printbibliography[heading=bibintoc]
\end{document}
